\documentclass[12pt]{article}

\usepackage[margin=1in]{geometry}
\usepackage[utf8]{inputenc}
\usepackage[T1]{fontenc}
\usepackage{textcomp}
\usepackage{ae,aecompl}
\usepackage{amsmath,amsbsy,amssymb,amsthm} % Combined ams packages
\usepackage{eurosym}
\usepackage{graphicx}
\usepackage[english]{babel}
\usepackage{longtable} % Loaded once
\usepackage{appendix}
\usepackage{float}
\usepackage{placeins}
\usepackage[hang, flushmargin, bottom]{footmisc}
\usepackage{multicol}
\usepackage{adjustbox}
%\usepackage{cite} % REMOVED - Conflicts with natbib
\usepackage{color}
\usepackage{setspace}
\usepackage{caption}
\usepackage{ifthen}
\usepackage{pdflscape}
\usepackage{lscape}
\usepackage{enumerate}
\usepackage[all]{xy}
\usepackage[TABTOPCAP]{subfigure}
\usepackage{fancyhdr}
\usepackage{siunitx}
\usepackage{booktabs}
\usepackage{verbatim}
\usepackage{tikz}
\usepackage{epstopdf}
\usepackage{titling,lipsum}
\usepackage{changepage}

% --- CITATION AND HYPERLINKS SETUP ---
% Load natbib for author-year citations
\usepackage{natbib} 

% Load hyperref LAST (with a few exceptions) and with all options
\usepackage[hyperfootnotes=false]{hyperref} 
\usepackage[all]{hypcap} % hypcap should be loaded after hyperref

% Configure hyperref settings
\hypersetup{
    colorlinks=true,
    citecolor=blue,
    urlcolor=blue,
    linkcolor=black % It's good practice to define linkcolor too
}

%TCIDATA comments are usually from scientific software and can be kept or removed
%TCIDATA{OutputFilter=LATEX.DLL}
%TCIDATA{Version=5.00.0.2552}
%TCIDATA{<META NAME="SaveForMode" CONTENT="1">}
%TCIDATA{LastRevised=Friday, March 14, 2014 19:54:11}
%TCIDATA{<META NAME="GraphicsSave" CONTENT="32">}

\xyoption{frame}
\setlength\parindent{24pt}
\setcounter{MaxMatrixCols}{10}
\captionsetup{format=default,font=small}

\pagenumbering{arabic}
\bibliographystyle{apalike}
\doublespacing


\title{\normalsize{\textbf{\Large{Wojciech Hrycenko Master Thesis Synopsis:\\ Privatization’s Unequal Toll: Explaining Cross-Country Variation in Mortality and Health Outcomes After Transition From Comunism}}} }
\date{\vspace{-5ex}}


\begin{document}

\begin{titlingpage}


\maketitle



\end{titlingpage}

\clearpage

% \section*{\Large{\textbf{Introduction}}}

\section*{Aim of the Study}

\begin{itemize}
    \item To investigate why post-communist countries experienced divergent mortality and health outcomes following privatization programs in the 1990s.
    \item The investigation focuses not only on differences in the privatization programs themselves but also on the interplay of health systems, social policies, and cultural factors.
    \item The study aims to identify why some countries $\mathbf{mitigated}$ privatization's negative health effects (e.g., Poland, Czech Republic) while others $\mathbf{suffered}$ mortality crises (e.g., Russia, Latvia).
\end{itemize}

\hrule
\section*{Research Problem}

\begin{itemize}
    \item To what extent can the divergent privatization strategies (mass versus gradual) implemented in post-communist countries during the 1990s account for the variation in mortality outcomes, and what was the moderating role of national social and health policy systems?
    \item Under what institutional and social conditions did mass privatization contribute to $\mathbf{divergent}$ mortality and health outcomes across post-communist countries?
    \item Why did some states (e.g., Russia, Latvia) experience severe mortality crises following rapid privatization, while others (e.g., Poland, Czech Republic) avoided such outcomes despite similar economic reforms?
    \item How did $\mathbf{mediating~factors}$ -- such as pre-existing health infrastructure, social safety nets, and alcohol regulation -- shape privatization's role in these divergent trajectories?
\end{itemize}

\hrule
\section*{Theses / Hypotheses}

\begin{itemize}
    \item \textbf{Main Hypothesis:} The \textbf{speed and method of privatization} were primary determinants of the post-communist mortality crisis. Specifically, countries that implemented \textbf{rapid mass privatization} (e.g., Russia) experienced significantly larger increases in mortality rates compared to countries that pursued a more \textbf{gradual privatization} strategy (e.g., Poland). The core mechanism is that rapid reforms created widespread social disruption, mass unemployment, and psychosocial stress, which directly led to negative health outcomes.

    \item \textbf{Hypothesis 2 (The Social Safety Net Moderator):} The adverse health impact of rapid privatization was significantly \textbf{moderated by the strength of the national social safety net}. In countries with more robust and pre-existing social protection systems (e.g., generous unemployment benefits, accessible social assistance), the statistical link between mass privatization and increased mortality was substantially weaker. These policies acted as a crucial buffer, mitigating the economic and social shocks for the most vulnerable segments of the population.

    \item \textbf{Hypothesis 3 (The Public Health System Moderator):} The resilience and concurrent funding of the \textbf{public health system} played a critical moderating role. The mortality crisis following rapid privatization was most severe in countries where the public health infrastructure simultaneously degraded or was defunded. Conversely, countries that successfully maintained or reformed their public health systems were able to counteract the negative health pressures induced by the economic transformation.

\end{itemize}
% \section*{Work Plan}

% \begin{enumerate}
%     \item Literature Review
%     \item Data Collection
%     \item Quantitative Analysis
%     \item Drafting
%     \item Final Revisions
% \end{enumerate}

\section*{Work Plan}
\begin{enumerate}

    \item \textbf{Literature Review} \\
    \textit{Timeline: -- November 15, 2025} \\
    This stage focuses on a comprehensive analysis of the existing research to establish the theoretical framework. Key activities include finalizing the research problem, formulating hypotheses, and writing the theoretical chapter. \\
    \textbf{Milestone:} An advanced draft of the theoretical chapter will be completed by \textbf{November 15, 2025}.

    \item \textbf{Data Collection} \\
    \textit{Timeline: November 1 -- December 20, 2025} \\
    This phase, running partly in parallel with the literature review, is dedicated to gathering all necessary empirical data. The process includes creating a master file and cleaning the dataset. \\
    \textbf{Milestone:} A complete, clean, and analysis-ready dataset will be finalized by \textbf{December 20, 2025}.

    \item \textbf{Quantitative Analysis} \\
    \textit{Timeline: December 21, 2025 -- March 15, 2026} \\
    This is the core research phase where the data is analyzed to test the hypotheses. It is divided into two parts: a preliminary analysis for the initial model (by Jan 31) and an in-depth analysis to follow. \\
    \textbf{Milestones:} The preliminary model, including initial findings, will be ready by \textbf{January 31, 2026}. All quantitative analysis will be completed by \textbf{March 15, 2026}.

    \item \textbf{Drafting} \\
    \textit{Timeline: January -- April 20, 2026} \\
    The writing process will occur alongside the analysis. This stage involves drafting all chapters (Introduction, Methodology, Results, Discussion, and Conclusion) and integrating them into a coherent manuscript. \\
    \textbf{Milestone:} The first complete draft of the entire thesis will be submitted to the Supervisor for review by \textbf{April 20, 2026}.

    \item \textbf{Final Revisions} \\
    \textit{Timeline: April 21 -- May 21, 2026} \\
    The final stage is dedicated to perfecting the manuscript based on the Supervisor's feedback. It includes proofreading, editing, and technical formatting of the bibliography, tables, and figures. A one-week buffer is reserved for any unforeseen issues. \\
    \textbf{Milestone:} The final thesis will be submitted by the official deadline of \textbf{May 21, 2026}.

\end{enumerate}

\hrule
\section*{Data Sources}

\begin{itemize}
    \item Human Mortality Database (HMD) -- Comprehensive mortality and population data. Link: \url{https://www.mortality.org/}
    \item WHO European Health for All Database (HFA-DB) -- Mortality data with split to cause of death. Link: \url{https://gateway.euro.who.int/en/datasets/#hfamdb}
    \item EBRD Transition Report -- Data on privatization methods and economic reforms. Link: \url{https://www.ebrd.com/transition-report}
    \item World Bank -- Private Sector GDI Share. Link: \url{https://data.worldbank.org/indicator/NE.GDI.FPRV.ZS}
    \item World Bank -- Unemployment Data. Link: \url{https://data.worldbank.org/indicator/SL.UEM.TOTL.ZS}
    \item Russian Health Care Spending -- Data on healthcare expenditure in Russia. Link: \url{https://pmc.ncbi.nlm.nih.gov/articles/PMC6571548/#B17-ijerph-16-01848} \url{https://nap.nationalacademies.org/read/5852/chapter/18#330}
    \item Health Care Expenditure Data -- OECD Health Statistics. Link: \url{https://www.cms.gov/research-statistics-data-and-systems/research/healthcarefinancingreview/downloads/03fallpg1.pdf} \url{https://ourworldindata.org/financing-healthcare}
\end{itemize}


\section*{Source Articles}



\begin{enumerate}
    \item Mass privatisation and the post-communist mortality crisis: a cross-national analysis -- David Stuckler, Lawrence King, Martin McKee
    \item Mass Privatisation and the Post-Communist Mortality Crisis: Is There Really a Relationship? -- John S. Earle, Scott Gehlbach
    \item Did Post-communist Privatization Increase Mortality -- John S Earle \& Scott Gehlbach
    \item Privatization and State Capacity in Postcommunist Society -- Lawrence King and Patrick Hamm
    \item The effect of rapid privatisation on mortality in mono-industrial towns in post-Soviet Russia: a retrospective cohort study -- Aytalina Azarova, Darja Irdam, Alexi Gugushvili, Mihaly Fazekas, Gábor Scheiring, Pia Horvat, Denes Stefler, Irina Kolesnikova, Vladimir Popov, Ivan Szelenyi, David Stuckler, Michael Marmot, Michael Murphy, Martin McKee, Martin Bobak, Lawrence King
    \item Who Wants to Revise Privatization and Why? Evidence from 28 Post-Communist Countries
    \item Politics and Policies in Post-Communist Transition: Primary and Secondary Privatisation in Central Europe and the Former Soviet Union -- Károly Attila Soós
\end{enumerate}

\end{document}


