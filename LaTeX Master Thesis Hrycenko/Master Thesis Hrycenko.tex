\documentclass[12pt]{article}

\usepackage[margin=1in]{geometry}
\usepackage[utf8]{inputenc}
\usepackage[T1]{fontenc}
\usepackage{textcomp}
\usepackage{ae,aecompl}
\usepackage{amsmath,amsbsy,amssymb,amsthm} % Combined ams packages
\usepackage{eurosym}
\usepackage{graphicx}
\usepackage[english]{babel}
\usepackage{longtable} % Loaded once
\usepackage{appendix}
\usepackage{float}
\usepackage{placeins}
\usepackage[hang, flushmargin, bottom]{footmisc}
\usepackage{multicol}
\usepackage{adjustbox}
%\usepackage{cite} % REMOVED - Conflicts with natbib
\usepackage{color}
\usepackage{setspace}
\usepackage{caption}
\usepackage{ifthen}
\usepackage{pdflscape}
\usepackage{lscape}
\usepackage{enumerate}
\usepackage[all]{xy}
\usepackage[TABTOPCAP]{subfigure}
\usepackage{fancyhdr}
\usepackage{siunitx}
\usepackage{booktabs}
\usepackage{verbatim}
\usepackage{tikz}
\usepackage{epstopdf}
\usepackage{titling,lipsum}
\usepackage{changepage}

% --- CITATION AND HYPERLINKS SETUP ---
% ZMIANA 1: Dodaj opcję [numbers] do pakietu natbib
\usepackage[numbers]{natbib} 

% Load hyperref LAST (with a few exceptions) and with all options
\usepackage[hyperfootnotes=false]{hyperref} 
\usepackage[all]{hypcap} % hypcap should be loaded after hyperref

% Configure hyperref settings
\hypersetup{
    colorlinks=true,
    citecolor=blue,
    urlcolor=blue,
    linkcolor=black % It's good practice to define linkcolor too
}

%TCIDATA comments are usually from scientific software and can be kept or removed
%TCIDATA{OutputFilter=LATEX.DLL}
%TCIDATA{Version=5.00.0.2552}
%TCIDATA{<META NAME="SaveForMode" CONTENT="1">}
%TCIDATA{LastRevised=Friday, March 14, 2014 19:54:11}
%TCIDATA{<META NAME="GraphicsSave" CONTENT="32">}

\xyoption{frame}
\setlength\parindent{24pt}
\setcounter{MaxMatrixCols}{10}
\captionsetup{format=default,font=small}

\pagenumbering{arabic}

% ZMIANA 2: Zmień styl bibliografii na numeryczny (np. plainnat)
\bibliographystyle{unsrtnat}
\doublespacing


\title{\normalsize{\textbf{\Large{Privatization’s Unequal Toll: Explaining Cross-Country
Variation in Mortality and Health Outcomes After Transition From Comunism}}} }
\date{\vspace{-5ex}}


\begin{document}

\begin{titlingpage}


\maketitle



\end{titlingpage}

\clearpage

% \section*{\Large{\textbf{Introduction}}}

\section*{Introduction}\

The collapse of communism in Central and Eastern Europe as well as in the former Soviet Union in the late 1980s 
and early 1990s marked a significant turning point in global political and economic history. 
The complexity of carrying out political, territorial, and economic reforms simultaneously is often termed 
the "dilemma of synchronicity" \citep{A_Ramet2010} . A cornerstone of this economic paradigm 
shift was privatization, which entailed the replacement of planned economies with free-market principles.
That process was expected to lead to increased efficiency, economic growth, and improved living standards. 
However, the outcomes of this transition have been far from uniform across countries, with some nations experiencing 
significant improvements in health and mortality rates, while others have faced deteriorating conditions \citep{B_Brainerd1998}.

This paper draws on ...
% do wykorzystania przy literaturze:
% into distinct types: models of "negotiated system reform" (Poland), 
% "enforced reform" (Czech Republic, Croatia), and "reform without negotiation" or 
% "imposed transformation" (Bulgaria, Serbia), a category that often includes Russia 
% and other former Soviet states, which also experienced systemic "implosion" \citep{BujwidKurek_Transformacja2015}. 

\section*{Relevant Literature}\

\subsection*{}\

\subsection*{}\

\subsection*{}\

\subsection*{}\

% The sources classify these transitions in various ways \citep{C_BujwidKurek_Transformacja2015}, however the most commonly referenced
% typology of transformation processes divides countries into three categories based on the nature of their reforms:
% "transformation" (e.g., Soviet Union, Bulgaria, Hungary), "transplacement" (e.g., Czechoslovakia, Poland), and "replacement" 
% (e.g., East Germany) \citep{D_Huntington1991}. These categories reflect the degree of domestic consensus and external pressure 
% influencing the reform processes. The privatization methods varied widely across the region. In Poland, 
% privatization occurred primarily through "capital" and "liquidation" tracks. 
% In the Czech Republic, the campaign of "mass privatization" was implemented but later criticized for contributing 
% to modest economic performance, financial scandals, a weak banking sector, and a loss of political orientation. 
% In Russia, a high percentage of enterprises chose "Option 2," leading to insider dominance, where employee ownership 
% was the majority in many privatized companies by early 1994. 
% These diverse methods and outcomes illustrate that privatization was a process that often diverged steeply from 
% original expectations.

% ... Tutaj reszta tekstu ...


%%% --- DODANO SEKCJA BIBLIOGRAFII ---
\clearpage 
\singlespacing 
\bibliography{myreferences} 

\end{document}