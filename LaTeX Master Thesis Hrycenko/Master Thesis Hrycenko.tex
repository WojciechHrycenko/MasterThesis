\documentclass[12pt]{article}

\usepackage[margin=1in]{geometry}
\usepackage[utf8]{inputenc}
\usepackage[T1]{fontenc}
\usepackage{textcomp}
\usepackage{ae,aecompl}
\usepackage{amsmath,amsbsy,amssymb,amsthm} % Combined ams packages
\usepackage{eurosym}
\usepackage{graphicx}
\usepackage[english]{babel}
\usepackage{longtable} % Loaded once
\usepackage{appendix}
\usepackage{float}
\usepackage{placeins}
\usepackage[hang, flushmargin, bottom]{footmisc}
\usepackage{multicol}
\usepackage{adjustbox}
%\usepackage{cite} % REMOVED - Conflicts with natbib
\usepackage{color}
\usepackage{setspace}
\usepackage{caption}
\usepackage{ifthen}
\usepackage{pdflscape}
\usepackage{lscape}
\usepackage{enumerate}
\usepackage[all]{xy}
\usepackage[TABTOPCAP]{subfigure}
\usepackage{fancyhdr}
\usepackage{siunitx}
\usepackage{booktabs}
\usepackage{verbatim}
\usepackage{tikz}
\usepackage{epstopdf}
\usepackage{titling,lipsum}
\usepackage{changepage}

% --- CITATION AND HYPERLINKS SETUP ---
% ZMIANA 1: Dodaj opcję [numbers] do pakietu natbib
\usepackage[numbers]{natbib} 

% Load hyperref LAST (with a few exceptions) and with all options
\usepackage[hyperfootnotes=false]{hyperref} 
\usepackage[all]{hypcap} % hypcap should be loaded after hyperref

% Configure hyperref settings
\hypersetup{
    colorlinks=true,
    citecolor=blue,
    urlcolor=blue,
    linkcolor=black % It's good practice to define linkcolor too
}

%TCIDATA comments are usually from scientific software and can be kept or removed
%TCIDATA{OutputFilter=LATEX.DLL}
%TCIDATA{Version=5.00.0.2552}
%TCIDATA{<META NAME="SaveForMode" CONTENT="1">}
%TCIDATA{LastRevised=Friday, March 14, 2014 19:54:11}
%TCIDATA{<META NAME="GraphicsSave" CONTENT="32">}

\xyoption{frame}
\setlength\parindent{24pt}
\setcounter{MaxMatrixCols}{10}
\captionsetup{format=default,font=small}

\pagenumbering{arabic}

% ZMIANA 2: Zmień styl bibliografii na numeryczny (np. plainnat)
\bibliographystyle{unsrtnat}
\doublespacing


\title{\normalsize{\textbf{\Large{Privatization’s Unequal Toll: Explaining Cross-Country
Variation in Mortality and Health Outcomes After Transition From Comunism}}} }
\date{\vspace{-5ex}}


\begin{document}

\begin{titlingpage}


\maketitle



\end{titlingpage}

\clearpage

% \section*{\Large{\textbf{Introduction}}}

\section*{Introduction}\

The collapse of communism in Central and Eastern Europe as well as in the former Soviet Union in the late 1980s 
and early 1990s marked a significant turning point in global political and economic history. 
The complexity of carrying out political, territorial, and economic reforms simultaneously is often termed 
the "dilemma of synchronicity" \citep{A_Ramet2010}. A cornerstone of this economic paradigm 
shift was privatization, which entailed the replacement of planned economies with free-market principles.
That process was expected to lead to increased efficiency, economic growth, and improved living standards. 
However, the outcomes of this transition have been far from uniform across countries, with some nations experiencing 
significant improvements in health and mortality rates, while others have faced deteriorating conditions \citep{B_Brainerd1998}.

This paper draws on ...
% do wykorzystania przy literaturze:
% into distinct types: models of "negotiated system reform" (Poland), 
% "enforced reform" (Czech Republic, Croatia), and "reform without negotiation" or 
% "imposed transformation" (Bulgaria, Serbia), a category that often includes Russia 
% and other former Soviet states, which also experienced systemic "implosion" \citep{BujwidKurek_Transformacja2015}. 

\section*{Relevant Literature}\
\subsection*{Transition Types}\

The widespread institutional collapse across Central and Eastern Europe and the Former Soviet Union in the late 1980s 
led to an imperative for simultaneous political, territorial, and economic restructuring. 
Within the academic field of transitology, literature seeks to define and categorize these complex processes in various ways \citep{C_BujwidKurek_Transformacja2015}. 
Probably the most commonly referenced typology of transformation processes \citep{C_BujwidKurek_Transformacja2015} of Samuel Huntington divides countries into three categories based on the nature of their reforms \citep{D_Huntington1991}:
\begin{itemize}
    \item "transformation" (e.g., Soviet Union, Bulgaria, Hungary) - when reformators group is originated in government elites, 
    \item "transplacement" (e.g., Czechoslovakia, Poland) - when the reforms are imposed by both government elites and opposition forces, as it happened during the round table talks,
    \item "replacement" (e.g., East Germany, Romania) - when oposition has a significant role in reforms.
\end{itemize}
These categories reflect the degree of domestic consensus and external pressure influencing the reform processes. 
It is also worth mentioning that both power camp and oposition are divided, government into reformators and "hard-headed", oposition into moderates and radicals.
In case of other classifications, they mostly match the above-mentioned typology, sometimes however some differences in countries grouping can be observed.
Sometimes, a separate category is being created for countries created from multi-national states, like Czechoslovakia or Yugoslavia, due to merged effect 
of divisions on the level of top leadership and the nationalistic pressure from below. \citep{Wiatr2020}.
Worth to mention is also the division proposed by Herbert Kitschelt due to underlining group of countries where preemptive reforms were the idea. Examples are Soviet Union Gorbachev’s initial 
innovations, as well as the regime changes in Bulgaria, Romania etc. \citep{Kitschelt2001}.


% The foundational political literature provides various typologies for these regime changes, 
% classifying states based on the role of the former communist elite and the opposition, thus establishing the initial political trajectory:
% • Negotiated System Reform: Characterized by pacts and agreements between the opposition and reformist elements of the communist party, exemplified by Poland and Hungary.
% • Enforced Reform: Occurring when the old regime was weak and opposition was strong, as seen in the Czech Republic and Croatia.
% • Reform without Negotiation / Imposed Transformation: Where reforms were controlled from above by existing elites, often resulting in systemic "implosion," a category frequently applied to Bulgaria, Serbia, Russia, and other FSU states.



% The long-term success of these transitions is assessed by the quality of the new political institutions. 
% Research in this area uses the institutional definition of the political system (encompassing state apparatus, parties, and legal framework) to analyze development. 
% A key finding in the literature is that initial political reforms frequently led to "facades of democratic institutions". 
% In these defective regimes, formal legal norms (often copied from consolidated Western democracies) lacked the "moral and cultural infrastructure" for effective functioning. 
% This weakness allowed political elite dominance over codified law and provided fertile ground for corruption, particularly in states where the political process was far from accountable and transparent. 
% The success of political reform was often predicted by the outcome of the first competitive election; clear opposition victories facilitated the "bundling" of liberalization and democratic consolidation, 
% as seen in Poland and the Czech Republic

%they are often also contrasting the short-term "transition" with the long-term, multi-dimensional "transformation" 
%required for institutionalization. The complexity has led some scholars to categorize the changes as a "refolution", combining peaceful reform with revolutionary scope.




%Transition process. Once the structural economic crisis of communist
%  regimes set in, the repressiveness of communist rulers and the (virtual)
%  resourcefulness of the opposition shaped the transitions to democracy
%  in the late 1980s. Under bureaucratic-authoritarian communism the
%  ruling parties were intransigent to reform and clung on to power until
%  the bitter end, but an urban middle-class opposition was potentially
%  resourceful against the backdrop of historical memories and practical
%  experiences in the interwar period. Communism here disappeared at a
%  late stage by implosion when the ruling parties could no longer contain
%  demands for fundamental change. The collapse of East Germany and of
%  the Czechoslovak regime in November 1989 exemplify this case.
%  Under national-accommodative communism, where the incumbent
%  party was more flexible in granting reforms and limited civil rights and
%  thus allowed opposition groups to become comparatively resourceful,
%  democracy evolved through a negotiated transition between rulers and
%  representatives of the opposition. Prolonged bargaining characterized
%  the transitions in Hungary, Poland, Slovenia, and possibly in Croatia
%  and the Baltic countries, where the communist party leaderships began
%  to accept oppositional representation before the communist regimes
%  collapsed.
%  Under patrimonial communism, by contrast, rulers had relied on
%  repression and cooptation and opposition forces remained weak and
%  isolated, thus never creating a serious threat to the political incumbents.
%  When communism was crumbling in one country after the other, elements
%  of the ruling communist parties themselves chose preemptive reform to
%  salvage as much of their power as possible into a new postcommunist
%  era in which they expected to continue their domination over a passive
%  civil society. Preemptive reform was the idea behind Gorbachev’s initial
%  innovations, as well as the regime changes in Bulgaria, Romania, some
%  of the Yugoslav republics, and many of the Soviet republics in 1990-91 \citep{Kitschelt2001}.


\subsection*{Economic Transformation and the Privatization Debate}\
% Privatization was universally deemed a central component of the economic transformation, necessary to replace the inefficient command economy with market principles. 
% The literature of the 1990s was dominated by a dichotomy regarding the pace and method of implementation:
% 2.2.1. Shock Therapy vs. Gradualism
% Neoliberal advisors championed "shock therapy," advocating for the rapid and simultaneous implementation of price and trade liberalization, stabilization, 
% and mass privatization. This approach was intended to create an irreversible shift to the market. Critics, often described as gradualist economists or neo-institutionalists, 
% argued that hasty reform damaged the state and favored a slower process to allow sufficient time for essential governing institutions (e.g., corporate law, capital markets) 
% to develop.
% 2.2.2. Privatization Methods and Consequences
% The specific methods chosen across the CEE and FSU yielded varied results and attracted distinct academic critiques:
% • Mass (Voucher) Privatization: While the econometric results suggest that mass privatization, overall, had a significant positive effect on growth after the initial 
% recession (post-1995), its implementation, particularly in the Czech Republic, was criticized for leading to "financial scandals, a weak banking sector, and a loss of 
% political orientation".
% • Insider Privatization: Models that favored management and employees (e.g., Russia's "Option 2") led to "insider dominance" [implied by FSU, 327] and were less 
% effective at restructuring than market methods. This approach was associated with political continuity and was largely avoided in the Czech Republic and restricted 
% in Poland, where hard political transition had weakened the political positions of former communist managers.
% • Corruption and Fiscal Shock: The literature confirms that the transition created new incentives for corruption due to the massive transfer of property, 
% coupled with weak states and underdeveloped civil societies. Research shows that privatization's general growth effect hinges on the quality of the institutional setting, 
% often proxied by the level of corruption. Furthermore, the implementation of mass privatization programs is empirically demonstrated to have created a massive fiscal 
% shock for post-communist governments, which undermined state capacity and protection of property rights, thereby exacerbating the recession.
% The general dissatisfaction with the privatization process stemmed from its non-transparent outcomes, rather than opposition to private property itself. 
% The popular demand for revising privatization was strongly correlated with individuals who experienced hardships during the transition, such as wage cuts, 
% unemployment, or poor self-rated health.




















\subsection*{Mortality and Health Outcomes}\

\subsection*{Research Gaps: Quantifying the Unequal Toll}\

% The sources classify these transitions in various ways \citep{C_BujwidKurek_Transformacja2015}, however the most commonly referenced
% typology of transformation processes divides countries into three categories based on the nature of their reforms:
% "transformation" (e.g., Soviet Union, Bulgaria, Hungary), "transplacement" (e.g., Czechoslovakia, Poland), and "replacement" 
% (e.g., East Germany) \citep{D_Huntington1991}. These categories reflect the degree of domestic consensus and external pressure 
% influencing the reform processes. 



% The privatization methods varied widely across the region. In Poland, 
% privatization occurred primarily through "capital" and "liquidation" tracks. 
% In the Czech Republic, the campaign of "mass privatization" was implemented but later criticized for contributing 
% to modest economic performance, financial scandals, a weak banking sector, and a loss of political orientation. 
% In Russia, a high percentage of enterprises chose "Option 2," leading to insider dominance, where employee ownership 
% was the majority in many privatized companies by early 1994. 
% These diverse methods and outcomes illustrate that privatization was a process that often diverged steeply from 
% original expectations.

% ... Tutaj reszta tekstu ...
\section*{Data and Methodology}\

\section*{Results}\

\section*{Conclusion}\

%%% --- DODANO SEKCJA BIBLIOGRAFII ---
\clearpage 
\singlespacing 
\bibliography{myreferences} 

\end{document}