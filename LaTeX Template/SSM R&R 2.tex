

\documentclass[12pt]{article}
%%%%%%%%%%%%%%%%%%%%%%%%%%%%%%%%%%%%%%%%%%%%%%%%%%%%%%%%%%%%%%%%%%%%%%%%%%%%%%%%%%%%%%%%%%%%%%%%%%%%%%%%%%%%%%%%%%%%%%%%%%%%%%%%%%%%%%%%%%%%%%%%%%%%%%%%%%%%%%%%%%%%%%%%%%%%%%%%%%%%%%%%%%%%%%%%%%%%%%%%%%%%%%%%%%%%%%%%%%%%%%%%%%%%%%%%%%%%%%%%%%%%%%%%%%%%
\usepackage[margin=1in]{geometry}
\usepackage[latin1]{inputenc}
\usepackage[T1]{fontenc}
\usepackage{textcomp}
\usepackage{ae,aecompl,amsmath,amsbsy,amssymb,eurosym}
\usepackage{graphicx}
\usepackage[english]{babel}
\usepackage{longtable}
\usepackage{appendix}
\usepackage{float}
\usepackage{natbib,hyperref}
\usepackage{placeins}
\usepackage[hang, flushmargin, bottom]{footmisc}
\usepackage{graphics}
\usepackage{multicol}
\usepackage{adjustbox}
\usepackage{cite}
\usepackage{color}
\usepackage{setspace}
\usepackage{caption}
\usepackage{ifthen}
\usepackage{pdflscape}
\usepackage{enumerate}
\usepackage{longtable}
\usepackage[all]{xy}
\usepackage[latin1]{inputenc}
\usepackage[TABTOPCAP]{subfigure}
\usepackage{amsthm,amsmath,natbib}
\usepackage[hyperfootnotes=false]{hyperref}
\usepackage[all]{hypcap}
\usepackage{fancyhdr}
\usepackage{siunitx}
\usepackage{booktabs}
\usepackage{verbatim}
\usepackage{tikz}
\usepackage{epstopdf}
%\usepackage[capposition=top]{floatrow}
\usepackage{titling,lipsum}
\usepackage{lscape}
\setlength\parindent{24pt}
\setcounter{MaxMatrixCols}{10}
\captionsetup{format=default,font=small}
\usepackage{changepage}

%TCIDATA{OutputFilter=LATEX.DLL}
%TCIDATA{Version=5.00.0.2552}
%TCIDATA{<META NAME="SaveForMode" CONTENT="1">}
%TCIDATA{LastRevised=Friday, March 14, 2014 19:54:11}
%TCIDATA{<META NAME="GraphicsSave" CONTENT="32">}

\xyoption{frame}
\hypersetup{
 colorlinks=true,
 citecolor=blue,
 urlcolor=blue}
\pagenumbering{arabic}
%\numberwithin{equation}{section}
%\linespread{2}
%\linespread{1.5}
\bibliographystyle{apalike}

\doublespacing

\title{\normalsize{\textbf{\Large{Equal before luck?  Well-being consequences of personal deprivation and transition}}} }
\date{\vspace{-5ex}}


\begin{document}

\begin{titlingpage}


\maketitle



\begin{abstract}
Past trauma resulting from personal life shocks, especially during periods of particular volatility such as regime transition (or regime change), can give rise to significant long-lasting effects on people's health and well-being. We study this question by drawing on longitudinal and retrospective data to examine the effect of past exposure to major individual-level shocks (specifically hunger, persecution, dispossession, and exceptional stress) on current measures of an individual's health and mental well-being. We study the effect of the timing of the personal shocks, alongside the additional effect of `institutional uncertainty' of regime change in post-communist European countries. Our findings are as follows: First, we document evidence of the detrimental effects of shocks on a series of relevant health and well-being outcomes. Second, we show evidence of more pronounced detrimental consequences of such personal shocks experienced by individuals living in formerly communist countries (which accrue to about 8\% and 10\% in the case of hunger and persecution, respectively) than in non-communist countries. The effects are robust and take place in addition to the direct effects of regime change and shocks. \\
 
\noindent \textbf{Keywords}: transition shocks; Soviet communism; later life health; health care system \\
\noindent \textbf{JEL codes}: I18, H75, H79\\
\end{abstract}
\noindent \textbf{Statements and Declarations}: Authors state no Ethics Approval was required in this research.


\end{titlingpage}

\clearpage

\section*{\Large{\textbf{Introduction}}}

Later-life health can be significantly shaped by early life experiences and specifically, detrimental events  \citep{almond2018childhood}, such as hardship, hunger, and economic deprivation or natural disasters and war \citep{haas2007long, adhvaryu2019early, dinkelman2017long}. However, such effects vary widely across individuals and rather than being homogeneous, they can be influenced by the institutional environment in which they take place, and in particular, by the politico-economic regime which individuals can be exposed to as well as by reduced access to and affordability of health care services over the life course \citep{grossman1972concept, grossman2005education}. The adoption of healthy behaviours can influence a series of potential outcomes such as an individual's mental health, overweight and obesity alongside other measures of health \citep{cui2020early}. However, to date, we still know little about the effects of constraints and incentives resulting from early life shocks, especially about the institutional determinants of health after major regime upheavals. 

This paper draws on the evidence from countries that were exposed to Soviet communism and experienced a regime change from communism to capitalism, to examine how the exposure to past personal shocks giving rise to deprivation (such as destitution or persecution) in different regimes affect the current health and well-being of older adults, and the extent to which the transition period plays a role in exacerbating the effect of such shocks. We contribute to the literature by showing the effect of the experience of hunger resulting from war experience on an individual's height, prevalence of obesity, and hypertension alongside blood pressure \citep{van2016instrumental}. Furthermore, the paper contributes to a growing literature on the effect of famines on later-life health \citep{kim2017long}. Similarly, the paper contributes to a wider literature on institutional or social determinants of health. This is a literature that has established that early life deprivation and adverse events exert later life undesirable effects such a higher rates of disability, which generally impacts negatively on healthy ageing. Finally, the paper contributes to still small but growing literature on the health effects of exposure to Soviet communism \citep{costa2023soviet}.  

A paradigmatic example of the importance of institutions in later life health lies in understanding the effects of the deep variation in health systems' organisation and constraints to preventative health behaviours between communist and non-communist countries in the last century. This paper draws on this evidence to study the heterogeneity of health and well-being effects of adverse life events between those countries that were exposed to Soviet communism and other European countries. However, given that the period of transition from communism to the free market regimes encompassed new institutional and economic shocks (such as re-privatization, inflation, structural unemployment, changes in the balance of payments and foreign direct investment, reforms to the health care system, etc.), we provide a specific examination of the effect of shocks during such a period. All these circumstances might have affected citizens directly and indirectly, e.g. through improved access to education, that is associated with health \citep{cutler2006education, conti2010education}, and through a shift in family and social ties towards a stronger focus on individuals needs \citep{balcerowicz1994understanding, sachs1994structural}.  This is important as regime change encompassed political and economic instability, which might have increased levels of stress \citep{lipowicz2016biological}. 

We examine the cross-country heterogeneity in the exposure to shocks in communist countries given that although most countries from the Soviet bloc (i.e. with centrally planned economies and communist politico-economic regimes) in Central and Eastern Europe experienced communism collapse around the same time (1989-1991), the process varied deeply across countries \citep{moller2009post}. While in some countries the transition occurred peacefully and relatively fast due to so called `shock therapy' \citep{sachs1995shock}, in other countries the changes introduced by the government met with weaker support from the citizens and the transition period was much more turbulent. Furthermore, the economic conditions and the links to Western Europe varied between member states of the Soviet Union (USSR) and other countries aligned with USSR during and prior to the transformation \citep{wedel2000us}. Similarly, despite the tight control of the public institutions in the Soviet bloc by the First Secretary of the Russian Communist Party, there was substantial level of variation in their performance and design between the communist countries, to the point in which some researchers make distinction between the `communist' regime in USSR and `socialist' regimes operating in other countries from the Soviet bloc before 1990's, see e.g. \citet{murphy2014socialism}. Hence, we refer to the data source that allows capturing a large number of countries exposed in the past to the communist regime in Europe.   

In this paper we exploit rich retrospective and longitudinal data on a set of later life outcomes from the Survey on Health, Ageing and retirement in Europe (SHARE), controlling for individual fixed effects and childhood conditions. Based on the life histories, we are able to identify four types of past shocks (hunger, dispossession, persecution, and exceptional stress) for individuals aged at least 50 at the time of the data collected in three waves of fieldwork spanning over years 2010-2015. The research sample comprises of individuals who never experienced international migration and lived in one of the six formerly communist and 13 non-communist countries participating in the survey, excluding Germany. By comparing the difference in differences between individuals exposed and unexposed to the shocks in the countries that did and did not underwent a regime transition, we are able 
to identify the heterogeneity of the shock effects across the formerly communist and non-communist countries. Retrospective interviews inform both about the shock occurrence and its timing, which allows us to distinguish between the shocks experienced before, during, and after the regime transition period (defined as years 1983-1995). 

We show significant and robust detrimental health and well-being effects of shocks experienced in post-communist countries, independently adding up to the separate effects of the shock and of the regime change. The shock impacts are more pronounced in the transition countries by about 8\% and 10\% in the case of hunger and persecution, respectively. We confirm relations established in the literature that hunger has a detrimental impact on well-being, and we further document that dispossession and other destabilizing events that had occurred in earlier life have similar detrimental effects.

Our results may be linked with the inefficiency of Semashko health care system in mitigating negative consequences of adverse health shocks. It seems that health care systems focused mainly on prevention with limited package of services are more vulnerable to personal deprivation than those with more extensive set of health care services, which can be applied also to current health policies. The more pronounced detrimental effects of shocks for women than men in formerly communist countries point to the limited resilience to the shocks of women burdened both with family and professional duties. Hence, in addition to policies promoting gender equality at the work place, those focused on the equality in family are needed to increase female health and well-being resilience to adverse life events.   

In the following section we discuss the studies relevant to our research question. Then we proceed to the description of the data used in the empirical analysis, which we follow with the section on the estimation strategy. After presenting the results of main analysis as well as the role of the shocks' timing and robustness tests, we proceed to the Discussion of potential mechanisms explaining the observed effects and the heterogeneity of the latter. Final section provides concluding remarks. 

\section*{\Large{\textbf{Relevant literature}}}
\label{sec:lit}

\subsection*{{\textbf{Health and well-being effects of institutional shocks}}}

The health consequences of the exposure to Soviet communism are heterogeneous in part due to their differences in economic development, alongside the effects of the transition itself. More specifically, one should distinguish Eastern European countries from countries exposed directly to direct Soviet Union rule. In the former ones, life expectancy has risen after communism due to improvement in dietary outcomes and cardiovascular health \citep{zatonski1998ecological}, while for the latter ones, evidence reveals that it was one of the few regions of the world exhibiting a reduction in life expectancy after 1991, primarily due to a rise in alcohol consumption and poor diets \citep{connor2004diets}. Although in the 1960s life expectancy was higher in former East Germany than in former West Germany, by the 1970s the numbers reversed \citep{nolte2000changing}, and life expectancy declined dramatically in 1990, especially for males. As the transformation continued, longevity had small improvements until the second half of the 2000s when it began to improve steadily.  


\subsection*{{\textbf{Long-term effect of hunger, persecution and dispossession}}}


  Early life shocks affecting food, deprivation and shelter can have long-term effects on individuals' well-being. Consistently, there is growing evidence of such effects during conflicts and periods of famines. Indeed, some studies examine the long-term consequences of childhood hunger during or after Second World War in European countries \citep{kesternich2020early}, and more generally the life satisfaction effects of exposure to hunger \citep{bertoni2015hungry}. Evidence from the great famine in China suggests that exposure to hunger early in life is found to increase the probability of being overweight, having difficulty with (Instrumental) Activities of Daily Living (ADLs/IADLs) and having depression in old age as disclosed from retrospective datasets \citep{cui2020early}. Similarly, another set of studies shows that exposure to hunger early in life worsens women's health 50 years later \citep{deng2022early}.
 
 Evidence of dispossession effects mainly refers to the role of conflict-related dispossession and indicates that dispossession among Palestine children exerts negative effects on behaviour, feelings, relationships, learning, and sense of self \citep{abdallah2022dispossession}. Similarly, persecution based on ethnicity, religion, class or political background has been documented to exert detrimental effects on the survivors of such experiences as is the case of survivors of the Cultural Revolution and class struggle in China \citep{qian2019long}, violence in Gaza \citep{hammad2020social} and religious persecution in Spain \citep{drelichman2021long}.
 


 \subsection*{{\textbf{Transition effects}}}
 
The existing literature searched for possible explanations for a health gap between post-communist and other developed countries, focusing primarily on socio-economic status (SES) and relative income during the transformation period \citep{bobak2000socioeconomic}, as well as the instability and working conditions which lead to a rise in work related stress \citep{salavecz2010work, jenkins2005mental}; the deterioration of public goods provision; and the exposure to macroeconomic volatility during the transition to capitalism \citep{guriev2009happiness}.  After the Gorbachov's Anti-Alcohol Campaign in Russia, the rise in alcohol consumption resulted in a drop of male life expectancy by about 6 years from 1989 to 1994 \citep{bhattacharya2013gorbachev}. Similarly, mass privatisation of state-owned enterprises and estates during the transition contributed to increased male mortality \citep{stuckler2009mass}. Finally, some evidence suggests that the unfolding of free markets posed some instantaneous morbidity effects, especially in rising tuberculosis and lung cancer among women increasingly targeted by tobacco companies \citep{mckee2004post}. 


\subsection*{{\textbf{Differential shock exposure}}}

However, the resilience to specific shocks differs across individuals. Attitudes and lifelong health habits during an individual's impressionable years are of especial importance as it is when preferences are shaped \citep{neundorf2013homemade}. This period is also critical for trajectories of physical health in later life \citep{leopold2018education} as well as behavioural risks undertaken habitually throughout life course \citep{jamner2003tobacco}. Regime differences might have exerted long-lasting effects especially if personal social networks and private informal relations serve as a source of access to privilege in communist societies, where opportunities for wealth accumulation are severely limited or unavailable \citep{filtzer2014privilege}. Thus, the differential roles of safety nets and informal insurance provided by family and acquaints under different politico-economic regimes \citep{costa2023comrades}, might lead to differential effects to deprivation shocks. 

\subsection*{{\textbf{Health care institutions}}}
Countries of former Soviet bloc adopted by the end of 1950s the Semashko system of health care that placed a particular emphasis on communicable disease control through mass vaccinations and malaria surveillance, the sanitary control of water supplies, hygienic waste disposal and sewage, and the pasteurization of milk \citep{rechel2014trends}. However, given that health was considered an unproductive sector, policy priority was given to other government responsibilities, hence funding for health tended to be low while the authority was heavily decentralised with limited budgetary responsibilities. Although the breadth of coverage in the other former Soviet countries was on paper `universal', the package of services covered was limited and entailed out-of-pocket and informal payments, which in some countries could make the large source of health financing \citep{lewis2007informal, ensor2004informal}. In vulnerable macroeconomic conditions, collecting mandatory insurance payments and taxes was a challenge due to tax evasion, unemployment, and the size of the informal sector \citep{rechel2014trends}. The latter led to the deterioration of infrastructure and poor qualifications of medical workers \citep{adam1991economic} continued during the transformation.

The health care systems during the transition not only faced a deep economic crisis but also a lack of coordination in the reforms applied to the fragmented elements of the public health system as well as unanticipated effects of rapid and unregulated development of private health care services \citep{millard1995changes}. Hence, the inefficiency of the system deepened despite some efforts of the government to reallocate resources, including human capital and equipment, as well as to subsidize the public sector with the introduction of out-of-pocket payments for medicines \citep{millard1995changes}. As a result of the parallel presence of private and public services, the former (often delivered after hours in public hospitals by their employees using rented public infrastructure) crowded out a substantial portion of public health care, where informal payments were not only widely accepted but also openly suggested. 

Despite certain heterogeneity in the initial systems between post-communist countries \citep{marree1997back, romaniuk2018ukrainian}, the regime change brought about comparable results in most of the formerly communist: the quality of services and access to them became substantially more unequally distributed between regions, urban and rural areas, as well as between the rich and the poor. However, governments failed to plan and instil appropriate reforms in comprehensive and accurate ways, and to regulate the vividly developing private sector. A decision to promote competition within the public sector between various suppliers contributed to the destruction of the sector \citep{millard1995changes}. Many patients were deprived of access to general practitioners, specialized medical treatments, and medicines during the first decade of the transformation. Further reforms continued over the next decades leading to more stable system, with a major role of the private sector, and improved supply of and access to health care, especially concerning the  treatment of cardiovascular diseases prevailing in the region  \citep{movsisyan2020cardiovascular}, though in certain domains such as mental health care the deterioration progressed \citep{mundt2012changes}. Hence, exposure to personal shocks under such conditions would have exacerbated its effects compared to non-communist countries. The rest of the paper will be devoted to the examination of such effects. 

\section*{\Large{\textbf{Data and measures}}}
\label{sec:data}

We use data from the Survey on Health, Ageing and Retirement in Europe (SHARE) \citep{borsch2013data}, a cross-national panel survey on health, socioeconomic status, and community and family groups of people aged 50 or over in continental Europe. In addition to a standard set of demographic controls, SHARE data include health, psychological, economic, and wealth variables. The sixth, fifth and fourth waves include nationally representative samples of nineteen countries (Austria, Sweden, the Netherlands, Spain, Italy, France, Denmark, Greece, Switzerland, Belgium, Israel, Czech Republic, Poland, Luxembourg, Hungary, Portugal, Slovenia, Estonia and Croatia) drawn from population registries or sampling. SHARE's third and seventh wave of data collection, SHARELIFE, collects detailed retrospective life accounts. The evidence in the life history includes family composition, type of home, and health status when the respondent was a child. 


Retrospective reports have been used to examine the effects of early life health \citep{haas2007long}, and more generally, retrospective data  has been shown to reliably identify important health behaviours such as past smoking \citep{kenkel2003smoke} and have been used to document the economic and health consequences of the Second World War in Europe \citep{kesternich2014effects}. One of the potential concerns is that of recollection bias and memory decay, which depends on the length of the recall and the salience of the events \citep{beckett2001quality}. In our data, recollection bias is less of a concern when some events are only a few decades ago from the time of the interview, such as the communism transition. Similarly, the events during the transitions as well as adverse and unusual events before that period, are very salient in people's lives, and they typically are very specific to a period that is unlikely to take place again. \citet{tapia2012recollection} show that the recollection of certain details of traumatic experience might be biased, but the memory sole event is not. The timing of specific shocks is critical for the assignment of the shock's occurrence to the period after, during or before the regime transformation, but the SHARELIFE used memory aids to minimize the bias, as advised in the literature  \citep{althubaiti2016information, khare2019recall}. However, even disregarding changes in political regimes, evidence suggests that retrospective data accurately measures the effects of events that have an impact on well-being. \citet{krall1988recall} show evidence of reliable self-reports of accidents and hospitalizations, which were recalled at high levels of accuracy \citep{haas2007long}. Importantly, \citet{havari2015can} examined the quality of retrospective data in SHARELIFE to find reassuring results. Hence, the rest of the study will rely on information of retrospective data along with current data from regular SHARE interviews.


We select individuals who participated in wave 4, wave 5, and wave 6 (of regular SHARE survey) and responded to the (retrospective) SHARELIFE questions in either in wave 3 or wave 7, resulting in a sample of 51364 individuals. Selective survival might affect our results, as the shocks to health might increase mortality in earlier life and thus lead to inconsistent estimates based on the sample comprised of older individuals. However, the bias is most likely to result in the underestimation of the effects, which means that our results are the lower bound of the actual impact. The proportion of individuals reporting the experience of hunger and particular stress in the past is larger in the SHARELIFE sample from 2008 than from 2017, in which newly recruited respondents from other European countries were on average older by almost 2 years (c.f. Table \ref{tab:selection}). These observations are in line with the increased mortality due to the experiences of these shocks. To address this potential identification threat, we refer to the heterogeneity analysis by birth cohort in the robustness section. Another source of selection bias might result from non-random migration, hence we exclude migrants from the analysis (11\% of the original sample), to reintroduce them again in the robustness analysis. We exclude observations on individuals whose country of residence over life course could not be uniquely assigned either to countries transitioning to free markets from communism or other. 

First, we assign individuals living for the entity of their lives in the formerly communist countries to the group of countries that underwent regime transition, and create a dummy equal to one if an individual lives in one of the following countries: Czech Republic, Estonia, Hungary, Slovenia, Poland. As far as the exogenous shocks are concerned, we use the information on the experiences of:\\
-- hunger, a dummy equal to 1 if respondent positively responded to the question ``(Looking back on your life,) was there a period during which you suffered from hunger?'';\\
-- material dispossession due to nationalization, a dummy equal to 1 if responded confirmed that ``you or your family was dispossessed of any property as a result of
war or persecution?'';\\
-- and persecutions due to political and other reasons, a dummy equal to 1 based on the question ``There are times, in which people are persecuted or discriminated against, for example because
of their political beliefs, religion, nationality, ethnicity, sexual orientation or their background. People may also be persecuted or discriminated against because of the political beliefs or the
religion of their close relatives. Have you ever been the victim of such persecution or discrimination?'';\\
-- particular stress ever experienced in the past as reported by respondents, a dummy equal to 1 if respondent confirmed that ``(Looking back on your life,) was there a distinct period during which you were under more stress compared to the rest of your life?''. \\
The timing of all the shocks except for the persecution is provided in the data. 



\begin{center}
\linespread{1.15}
\begin{table}[h!]
 
\captionof{table}{Shocks occurrence in formerly communist and other countries}
\resizebox{\textwidth}{!}{

{
\def\sym#1{\ifmmode^{#1}\else\(^{#1}\)\fi}
\begin{tabular}{l*{1}{cccccc}}
\hline
            &\multicolumn{3}{c}{Formerly communist}      &\multicolumn{3}{c}{Non-communist}                                         \\
            &       \textit{N}&  Mean&    St.D&  \textit{N}&  Mean&       St.D\\
\hline
\\
Hunger      &       15951&    0.066&    0.248&     34098&    0.059&     0.236 \\
Dispossession      &       16289&    0.124&    0.329&      34567&    0.044&    0.205\\
Persecution      &       16314&     0.075&     0.263&         34639&    0.046&    0.209 \\
Period of particular stress      &       15798&    0.451&     0.498&      33650&     0.516&    0.500    \\
\\
%Period of particular happiness   &       15726&    .4743101&    .4993555&       33504&    .4599152&    .4983981     \\
\bottomrule
\multicolumn{7}{l}{\footnotesize  \textit{Source}: SHARE (waves 4, 5, 6) and SHARELIFE (waves 3, 7), release 7.0.0.}\\
\end{tabular}
\label{tab:des}



}

}
\end{table}
\end{center}

\vspace{-1.5cm}

The shocks' prevalence in entire life course of examined respondents varies significantly between countries' experience of regime transition (c.f. Table \ref{tab:des}), especially with respect to dispossession, which concerns 12\% of individuals from formerly communist countries and only 4\% of their counter-partners in Western Europe. Similarly, proportion of individuals persecuted in Eastern Europe is about twice as big as in Western Europe, reaching 7\%. However, the risks of hunger and extraordinary periods of life with respect to stress seem to be very similar in both types of countries.



Then, we draw on several of measures of current short-term health (BMI, obesity, chronic conditions' occurrence and number), well-being and mental health (indices of quality of life and depression), current socio-economic status (wealth, education, marital status) as well as information on health (immunization, asthmna, cancerm, hospitalization) and socioeconomic status (SES index and father's presence) during childhood. Table \ref{tab:vars} provides details to the operationalization of these variables. 

\begin{landscape}
 \captionof{table}{Dependent and control variables' operationalization}
{
 \begin{adjustwidth}{-2.1cm}{}
\def\sym#1{\ifmmode^{#1}\else\(^{#1}\)\fi}
\scriptsize{
\begin{tabular}{p{0.17\linewidth}  p{0.88\linewidth}}
\hline
            \multicolumn{2}{c}{\textbf{Current short-term health}}  \\
Body Mass Index (BMI)      & a measure calculated by the SHARE team after cleaning the raw data as a ratio of self-reported weight (in kgs) to the self-reported height (in meters) squared \\ 
Obesity     &      a dummy variable equal to 1 if BMI (as operationalized above) is more than 30\\
Chronic conditions      &   a dummy equal to 1 if respondent confirmed suffering from chronic or long-term health problems in a question worded: ``By long-term we mean it has troubled you over a period of time or is likely to affect you over a period of time. Do you have any long-term health problems, illness, disability or infirmity?'' \\
Number of chronic conditions      &    a measure  calculated by the SHARE team after cleaning the raw data based on the responses to the question ``[Has a doctor ever told you that you had/Do you currently have] any of the conditions on this card? Please tell me the number or numbers of the conditions''    \\
            \multicolumn{2}{c}{\textbf{Current well-being and mental health}}  \\
CASP-12 & an index of the quality of life and well-being as defined by \citet{hyde2003measure} composed of four subscales (control, autonomy, selfrealization and pleasure) ranging from 12 to 48, where higher scores are associated with a higher quality of life\\
EURO-D & a depression scale defined by \citet{prince1999development} as a composite index of questions on mental health, ranging from 0 (no depression) to 12\\
            \multicolumn{2}{c}{\textbf{Current well-being and mental health}}  \\
Net household wealth  & a measure generated by the authors as a log of the ration of net wealth to the number of household members\\
Education  & dummies for ISCED categories 0--6 as provided by the the SHARE team after harmonization of country-specific education systems into comparable ISCED levels\\
Marital status & a dummy for being married or living in registered partnership based on the responses to the question: ``What is your marital status? 1. Married and living together with spouse, 2. Registered partnership,  3. Married, living separated from spouse, 4. Never married, 5. Divorced, 6. Widowed''\\ 
           \multicolumn{2}{c}{\textbf{ Childhood circumstances}}  \\

Immunization in early life & a dummy equal to 1 if responded confirmed that ``During your childhood, that is, from when you were born up to and including age 15, have you received any vaccinations?'' \\
Asthma & a dummy equal to 1 if respondent selected asthma responding to the question ``Did you have any of the diseases on this card during your childhood (that is, from when you were born up to and including age 15)?''\\
Cancer or malignant tumour & a dummy equal to 1 if cancer (excluding minor skin cancers) was selected in the question above\\
Hospitalisation in childhood& a dummy equal to 1 if respondent confirmed that ``During your childhood, because of a health condition, were you ever in hospital for one month or more?''\\ 
          \multicolumn{2}{c}{\textbf{ Status at 10 years of age}}  \\

Father's presence & a dummy based on the question ``Which of the people on this card did you live with at this accommodation when you were 10?'' for biological, adoptive, step, or foster father\\
Index of socio-economic status & a measure built using a principal component analysis that includes three principal areas: the number of people living in the house divided by the number of rooms, the number of books in the house, and features of the house (whether there was a bath, running cold and hot water, inside toilet, central heating, or none of these)\\
\bottomrule
\end{tabular}
\end{adjustwidth}}
\label{tab:vars}
\end{landscape}



\begin{center}
\captionof{table}{Descriptive statistics}
\label{tab:summary}
\resizebox{.8\textwidth}{!}{
{
\def\sym#1{\ifmmode^{#1}\else\(^{#1}\)\fi}
\begin{tabular}{l*{1}{cccccccccccc}}
\toprule
 & \multicolumn{5}{c}{Total}& &\multicolumn{2}{c}{Males only}& \multicolumn{2}{c}{Females only}\\
            &       \textit{N}&        Mean&          St.d.&         Min&         Max &      &        Mean&          St.d.&              Mean&          St.d.&         \\
\hline
Regime change  &       51364&    0.320&    0.467&           0&           1&       &    0.298&    0.457&                0.338&    0.473\\
\\
Age (years)         &       51356&    66.955&    10.077&          50&         105 &       &    67.016&    9.736&              66.907&    10.338\\
Married     &       51364&    0.696&    0.460&           0&           1&       &    0.790&    0.407&                0.622&    0.485\\
ISCED 0 (pre-primary) &       50551&    0.044&    0.205&           0&           1 &       &    0.039&    0.193&              0.048&    0.213      \\
ISCED 1 (primary) &       50551&    0.158&    0.365&           0&           1 &       &    0.141&    0.348&            0.172&    0.378     \\
ISCED 2 (lower secondary) &       50551&    0.190&     0.392&           0&           1&       &    0.172&    0.378&                0.203&    0.403      \\
ISCED 3 (upper secondary) &       50551&    0.338&    0.473&           0&           1  &     &    0.358&     0.479&            0.323&    0.468\\
ISCED 4 (post-secondary) &       50551&    0.046&    0.210&           0&           1 &      &    0.046&    0.209&                  0.047&    0.211\\
ISCED 5 (first stage of tertiary) &       50551&    0.215&    0.411&           0&           1 &      &    0.233&    0.422&              0.201&    0.400\\
ISCED 6 (second stage of tertiary)  &       50551&     0.008&     0.092&           0&           1&       &    0.012&    0.108&               0.006&    0.077&         \\
Net wealth (log) &       49340&    11.386&    2.417&   -29.805&    17.407 &       &    11.581&    2.205&     11.231&    2.562  \\
\\

Body Mass Index (BMI)        &       49973&    26.993&    4.671&    12.487&    74.740 &      &    27.258&    4.141&        26.781&     5.043&  \\
Obesity        &       49973&    0.225&    0.417&    0&    1 &      &    0.216    &    40.411&       0.232     &     0.423&  \\
Chronically ill    &      51364&     0.772&    0.420&           0&          1 &   &    0.763&    0.425& 0.778&    0.415\\
Number of chronic diseases    &      51232&     1.534&    1.119&           0&          3 &   &    1.480&    1.104& 1.578&    1.128\\
CASP quality of life index        &       48022&    37.398&    6.267&          12&          48&     &    37.833&    6.072&                 37.058&     6.400 \\
EURO-D depression scale      &       49022&    2.442&    2.258&           0&          12  &&    1.982&    2.032&                 2.799&    2.358\\
%Mobility limitations    &       51255&    1.707&    2.435&           0&          10&  &    1.322&    2.189&               2.009&    2.572   \\
%Limitations in daily living (GALI)        &       51260&    0.482&    0.500&           0&           1&       &    0.453&    0.498&                0.505&    0.500\\
%Self-reported health (SPHUS)       &       51251&    3.215&    1.075&           1&           5  &       &    3.166&    1.080&               3.254&    1.070\\
%Diabetes    &       51232&    0.136&    0.343&           0&           1  &       &    0.150&    0.357&                0.125&    0.330      \\
%Age of retirement    &       51363&    68.160&    9.784&          43&         104 &    &    68.180&    9.403&                68.143&    10.074\\
\\
Childhood SES: father absent&       50878&    0.119&    0.324&           0&           1&      &    0.114&    0.318&               0.124&    0.329       \\
Childhood SES: low status     &       48083&    0.202&    0.401&           0&           1  &      &    0.207&     0.405&                0.198&    0.399\\
Childhood SES: medium status     &       48083&    0.352&    0.477&           0&           1 &    &    0.364&    0.481&               0.342&    0.475      \\
Childhood SES: high status  &       50034&    0.222&    0.416&           0&           1 &       &    0.223&    0.416&                0.222&    0.416       \\
Childhood hospitalization      &       50691&    0.064&    0.245&           0&           1&     &     0.067&    0.249&            0.062&    0.242       \\
Childhood asthma    &       50472&    0.018&    0.133&           0&           1 & &    0.020&    0.140&                0.017&    0.128\\
Childhood cancer   &       50623&    0.001&    0.035&           0&           1 &       &    0.001&     0.033&              0.001&      0.037       \\
Childhood immunization   &       50287&     0.964&    0.187&           0&           1 &  &    0.966&    0.182&                0.962&    0.191\\
\\
Hunger      &       50049&      0.062&    0.240&           0&           1 &    &    0.060&    0.237&                 0.062&    0.242\\
Dispossession      &       50856&    0.070&    0.254&           0&           1 &  &    0.067&    0.250&                 0.071&    0.258      \\
Persecution    &       50953&    0.055&    0.228&           0&           1&       &    0.059&      0.234&                  0.053&    0.223  \\

Period of particular stress      &       49448&    0.495&    0.500&           0&           1 &      &    0.442&    0.497&           0.537&    0.499       \\
\\
\bottomrule
\multicolumn{11}{l}{\footnotesize  \textit{Source}: SHARE (waves 4, 5, 6) and SHARELIFE (waves 3, 7), release 7.0.0.}\\

\end{tabular}
}

%\multicolumn{11}{l}{\footnotesize  \textit{Notes}:  ISCED education levels: 0 - pre-primary, 1 - primary, 2 - lower secondary, 3 - upper secondary, 4 - post-secondary non-tertiary, 5 - first stage of tertiary, 6 - second stage of tertiary education. SES - socio-economic status.}\\}\\
\end{center}


The average age of individuals in the research sample is 67 years old, both for men and women, and two thirds of them  are married (67\%). We observe an exposure to the shocks of hunger, dispossession, and persecution at similar level of about 6\% for men and women (c.f. Table \ref{tab:summary}). About half of female respondents reports that their lives were free from an exceptionally stressful times, and in the case of men, this proportion is larger (56\%). We document differences in health between genders in line with previous studies.   About 77\%  of the sample representative for the 50+ population in Europe have at least one chronic illness, and 22\% are obese. The BMI is about 27 on average. We find relatively high scores of the CASP index of 37 average and a relatively low prevalence of depression (average EURO-D score is 2.4). Father was absent in 12\% of respondents' households when the respondent was 10 years old. Table \ref{tab:summary} provides more details on the variables and controls. 




Using SHARE retrospective data we show negative correlation between health and well-being in later life and the experience of hunger, dispossession, and persecution  in the past as well as the instability with respect to periods of particular stress in individual life history (Table \ref{tab:shocks}), in line with the existing literature. The significant association between well-being and health in later life with the experience of transition is documented in Appendix (Table \ref{tab:trans}). In particular, we document higher BMI and prevalence of obesity, lower quality of life, and more depression symptoms among those exposed to transition, both men and women. We refer to the methods delineated below to identify consistent and unbiased effects of the shocks and regimes that appeared in the past century in Europe.






\section*{\Large{\textbf{Empirical methodology}}}
\label{sec:data}

Our identification strategy relies on a natural experiment created by the rise and fall of the communist politico-economic regimes in Europe in the past century. Unlike the vast body of the literature exploring the reunification of Eastern and Western Germany \citep[e.g.]{nikolova2024echoes}, we expand the scope of countries that experienced regime change to a set of six counties and compare them with the remaining 13 non-communist countries captured in the SHARE survey. We exclude Germany due to unobserved migration between Eastern and Western Germany. Only one of the formerly communist countries in our sample belonged to USSR and in all of these countries the introduction of communist regime was imposed by external  political and military power, which supports the exogeneity of the selection into the group of countries with and without a regime change. Such a framing of empirical analysis allows us to document communist effects for the set of Central-Eastern European countries, and encompasses a recent critique of the validity of the German experiment \citep{becker2020separation}.

We employ the Mundlak approach to difference-in-differences \citep{wooldridge2021two} in our analysis of the effects of past shocks on later life health and well-being to test if they are heterogeneous for individuals born and living in formerly communist as opposed to non-communist countries. The treatment group are the individuals who have experienced a shock and have lived in a country that underwent a regime change. The control group consists of three types of individuals: 1) those born in a country with regime change who have not experienced the shock; 2) those born in a non-communist country who have experienced the shock; and 3) those born in the non-communist country who have not experienced the shock. Such an empirical strategy allows the identification of all three effects: i) of the regime change itself, ii) of the shock itself, and most importantly for answering our research question, iii) of the shock that occurred in the country that changed a regime. Aiming to explore the entire information available in the three waves of regular SHARE survey, we use panel data estimation methods and pooled ordinary least squares.  We investigate both fixed and random effects, and refer in the main analysis to the more conservative (i.e. with lower statistical significance and marginally smaller estimates of the examined effects) results obtained in the fixed effects approach. In addition, we investigate robustness to alternative estimation specifications to find reassuring results.
 
 The identification of the systematic differences between individuals living in the formerly communist countries and their Western European counterparts is crucial for our empirical strategy, as they might lead to the estimates' bias resulting from the correlation between the shocks and country type. Given that individuals in our sample are natives living in the same country, fixed effects panel data analysis removes the country effects from the estimates on the role of past shocks. In an attempt to capture the disparity in preceding conditions of life, we control for childhood circumstances, in particular childhood health and socio-economic status (SES). We examine the question of sample selection in the battery of robustness checks. 
 
 
 In order to establish the effects of the difference between the effect of a given shock on later life outcomes between the formerly communist countries and their Western European counterparts, we formally conduct the following panel data estimation:


\setcounter{equation}{0}

\begin{equation}
\label{equation:double}
\begin{split}
\mbox{\hspace*{-.6cm}\textit{Outcome}}_{i,g,t}= & \beta_{0} + \beta_{1} \mbox{\textit{Shock_{i,g} $\times$ Regime change}}_{i,g} + \beta_{2}  \mbox{\textit{Shock }}_{i,g}  \\+ \beta_{3}  \mbox{\textit{Regime change }}_{i,g} & + \beta_{4}  \mbox{\textit{Childhood conditions}}_{i,g}  + \beta_{5}  \mbox{\textit{Controls}}_{i,g,t} &  + \beta_{6}  \mbox{\textit{Individual FE}}_{i,g} + \eta_{i,g,t} \\
\end{split}
\end{equation}   

where ${\textit{Outcome}}_{i,g,t}$ is a measure of well-being at time \textit{t} of individual \textit{i} living in country \textit{g}; \textit{Shock$_{i,g}$} is a dummy for the experience of a shock to health in the past by individual \textit{i} living in country \textit{g}; \textit{Regime change$_{i,g}$} denotes living in a formerly communist country or a non-communist country; {\textit{ Childhood conditions}}$_{i,g}$ include childhood SES, and the measures of health during childhood (any hospitalization, asthma, cancer, immunization), and {\textit{ Controls}}$_{i,g,t}$ are age (2nd polynomial) and gender at time \textit{t}, and gender. The parameter of interest is the estimate on the interaction of \textit{Shock$_{i,g}$} with ${\textit{Regime change}}_{i,g}$ that denotes whether individual \textit{i} experienced the shock in the country \textit{g} that was subject to the regime transformation.


In the next step, we make a distinction between the period of stable communist regime (before 1982) and the period of regime transition (after 1982). In particular, we refer to the variable \textit{Transition period$_{i,g} $} indicating the years of the regime transition period and examine the triple interaction between \textit{Shock$_{i,g}$}, \textit{Transition period$_{i,g} $} and ${\textit{Regime change}}_{i,g}$ which value equal to one denotes the experience of a shock experienced by individual \textit{i} during the regime transformation in formerly communist country \textit{g}, as in the equation: 


\begin{equation}
\label{equation:tripple}
\begin{split}
\mbox{\hspace*{-.6cm}\textit{Outcome}}_{i,g,t}= & \delta_{0} + \delta_{1} \mbox{\textit{Shock_{i,g} $\times$ Transition period_{i,g} $\times$ Regime change}}_{i,g}   \\
& +  \delta_{2,1}\mbox{\textit{Shock_{i,g} $\times$ Transition period}}_{i,g} + \delta_{2,2}\mbox{\textit{Shock_{i,g} $\times$ Regime change}}_{i,g} \\
& + \delta_{2,3}\mbox{\textit{Transition period_{i,g} $\times$ Regime change}}_{i,g}  + \delta_{2,4}\mbox{\textit{Transition period}}_{i,g} \\
& + \delta_{2,5}  \mbox{\textit{Regime change }}_{i,g} + \delta_{3} \mbox {\textit{ Childhood conditions}}_{i,g}  \\
& + \delta_{4}  \mbox{\textit{Controls}}_{i,g,t}+ \delta_{5}  \mbox{\textit{Individual FE}}_{i,g}  + \mu_{i,g,t} \\
\end{split}
\end{equation}  

The coefficient on the triple interaction in the above specification reveals that when the shock occurred in a formerly communist country during the regime transformation it adds up the independent effects of the shocks that occurred in other times and of living in the formerly communist country on later life health. More specifically, we examine in this analysis the occurrence of the shock during a turbulent time of volatile changes brought about with the collapse of communist regime in Central-Eastern Europe, which deeply affected not only the region but also entire Europe, and might have been particularly important for later life health and well-being of individuals in the formerly communist countries.



\section*{\Large{Results}}
\label{sec:results}
\setlength\parindent{24pt}
\subsection*{Effects of regime transition and shocks}



We proceed to the main analysis as specified in equation \ref{equation:double}. In this fixed effects panel data  analysis, we examine the additional effect of shock experience in the country that underwent regime transition and control for childhood health and socio-economic status. We document significant detrimental effects of shocks experienced in the post-communist countries on long-term health measures, mental health and well-being, independently adding up to the separate effect of sole shock and of sole regime transformation in the panel estimation with individual fixed effects (c.f. Table \ref{tab:main}). 

\begin{table}[h!]
\captionof{table}{Shock effects and regime change}
\resizebox{\textwidth}{!}{
{
\def\sym#1{\ifmmode^{#1}\else\(^{#1}\)\fi}
\begin{tabular}{l*{6}{c}}
\toprule
            &\multicolumn{2}{c}{Short-term health}&\multicolumn{2}{c}{Long-term health}&\multicolumn{2}{c}{Well-being}\\
            &\multicolumn{1}{c}{BMI}&\multicolumn{1}{c}{Obesity (1,0)}&\multicolumn{1}{c}{Chronically ill (1,0)}&\multicolumn{1}{c}{No. of chronic illnesses}&\multicolumn{1}{c}{CASP index}&\multicolumn{1}{c}{Depression scale}\\
\hline 
\\
Hunger $\times$ Regime change&     1.516\sym{***}&      0.128\sym{***}&     0.072\sym{**} &      0.437\sym{***}&     -3.922\sym{***}&      0.776\sym{**} \\
            &    (0.115)         &    (0.002)         &    (0.008)         &    (0.018)         &    (0.205)         &    (0.116)         \\

\\
Regime change&     1.378\sym{***}&      0.101\sym{***}&     0.051\sym{**} &      0.190\sym{**} &     -2.166\sym{**} &     0.032         \\
            &    (0.028)         &    (0.003)         &    (0.009)         &    (0.029)         &    (0.432)         &    (0.140)         \\


Hunger&      0.663\sym{***}&     0.058\sym{***}&     0.071\sym{**} &      0.303\sym{**} &     -2.619\sym{**} &      0.861\sym{**} \\
            &    (0.044)         &    (0.006)         &    (0.010)         &    (0.043)         &    (0.344)         &    (0.109)         \\

\(N\)       &       37783         &       37783         &       38748         &       38660         &       36818         &       37467         \\

\hline
\\
Dispossession $\times$ Regime change&   1.478\sym{***}&     0.099\sym{***}&     0.058\sym{***}&      0.261\sym{***}&     -2.249\sym{**} &      0.149         \\
            &    (0.068)         &    (0.004)         &    (0.005)         &    (0.007)         &    (0.400)         &    (0.138)         \\


\\
Regime change&       1.340\sym{***}&     0.099\sym{***}&     0.046\sym{**} &      0.178\sym{**} &     -2.111\sym{**} &     0.015         \\
            &    (0.014)         &    (0.002)         &    (0.008)         &    (0.030)         &    (0.444)         &    (0.142)         \\


Dispossession&   0.147         &    0.003         &     0.054\sym{*}  &      0.172         &     -0.236         &      0.268\sym{**} \\
            &    (0.077)         &    (0.005)         &    (0.015)         &    (0.063)         &    (0.475)         &    (0.059)         \\



\(N\)       &       38081         &       38081         &       39058         &       38969         &       36987         &       37634         \\

\hline
\\
Persecution $\times$ Regime change&     1.455\sym{***}&      0.113\sym{***}&     0.078\sym{**} &      0.339\sym{***}&     -2.817\sym{**} &      0.317         \\
            &    (0.035)         &    (0.003)         &    (0.013)         &    (0.022)         &    (0.366)         &    (0.167)         \\


\\
Regime change&      1.359\sym{***}&     0.098\sym{***}&     0.045\sym{**} &      0.174\sym{**} &     -2.089\sym{**} &     0.017         \\
            &    (0.016)         &    (0.003)         &    (0.007)         &    (0.027)         &    (0.438)         &    (0.140)         \\

Persecution&      0.457\sym{***}&     0.025         &     0.085\sym{***}&      0.260\sym{**} &     -1.438\sym{**} &      0.762\sym{***}\\
            &    (0.036)         &    (0.010)         &    (0.006)         &    (0.030)         &    (0.168)         &    (0.061)         \\


\(N\)       &       38113         &       38113         &       39090         &       39000         &       37010         &       37663         \\


\hline

\\
Period of particular stress $\times$ Regime change&      1.529\sym{***}&      0.118\sym{***}&      0.115\sym{***}&      0.416\sym{***}&     -3.172\sym{**} &      0.665\sym{**} \\
            &    (0.070)         &    (0.006)         &    (0.003)         &    (0.020)         &    (0.487)         &    (0.147)         \\


\\
Regime change&     1.228\sym{***}&     0.089\sym{***}&     0.050\sym{**} &      0.160\sym{**} &     -1.820\sym{*}  &     0.042         \\
            &    (0.026)         &    (0.003)         &    (0.006)         &    (0.028)         &    (0.513)         &    (0.143)         \\


Period of particular stress &    0.026         &    0.008         &     0.063\sym{**} &      0.175\sym{***}&     -0.596\sym{**} &      0.590\sym{***}\\
            &    (0.039)         &    (0.004)         &    (0.009)         &    (0.013)         &    (0.108)         &    (0.022)         \\

\(N\)       &       37532         &       37532         &       38461         &       38378         &       36621         &       37258         \\
\hline\\
Age & YES & YES& YES& YES& YES& YES \\ 
Sex & YES & YES& YES& YES& YES& YES \\ 
Childhood SES & YES & YES& YES& YES& YES& YES  \\ 
Childhood health & YES & YES& YES& YES& YES& YES  \\ 
Individual FE & YES & YES& YES& YES& YES& YES  \\ 
\bottomrule
\multicolumn{6}{l}{\footnotesize  \textit{Source}: SHARE (waves 4, 5, 6) and SHARELIFE (waves 3, 7), release 7.0.0.}\\
\multicolumn{6}{l}{\footnotesize   \textit{Notes}: Robust standard errors clustered at individual in parentheses. \sym{*} \(p<0.10\), \sym{**} \(p<0.05\), \sym{***} \(p<0.01\).}\\
\label{tab:main}
\end{tabular}
}

\\
}
\end{table}


The effects of shocks in Western European countries confirm the  hypothesis of detrimental effects of personal deprivation (particularly hunger, dispossession, persecution) on health and well-being measures. The effects of regime change, i.e. living under Soviet communism throughout some part of life, are consistently negative for later-life health and well-being except for depression. These effects remain significant after controlling for all the personal deprivation shocks examined in our study. That might point to inefficiency of the communist health care system, limited health literacy in formerly communist countries as well as relatively poor living and work conditions in the countries that underwent the regime transition.


\subsection*{The effects of shock's timing }


 Table \ref{tab:during} presents the effects of the shocks experienced during the regime transformation (1983-1995) in the post-communist countries, as specified in equation \ref{equation:tripple}. We find little evidence that the shocks during the transformation pose significantly different impact from the shocks in post-communist counties before and after the transformation. 

\begin{table}[h!]
\captionof{table}{Shock effects during transition}
\resizebox{\textwidth}{!}{
{
\def\sym#1{\ifmmode^{#1}\else\(^{#1}\)\fi}
\begin{tabular}{l*{6}{c}}
\toprule
            &\multicolumn{2}{c}{Short-term health}&\multicolumn{2}{c}{Long-term health}&\multicolumn{2}{c}{Well-being}\\
            &\multicolumn{1}{c}{BMI}&\multicolumn{1}{c}{Obesity (1,0)}&\multicolumn{1}{c}{Chronically ill (1,0)}&\multicolumn{1}{c}{No. of chronic illnesses}&\multicolumn{1}{c}{CASP index}&\multicolumn{1}{c}{Depression scale}\\
\hline
\textit{Hunger $\times$ Transition period $\times$ Regime change}  &  & & & & &  \\ 

Total &     -0.048         &     0.049         &    0.006         &   -0.008         &    -0.030         &     -0.675\sym{**} \\
            &    (0.958)         &    (0.104)         &    (0.002)         &    (0.058)         &    (1.108)         &    (0.136)         \\


\(N\)       &       37783         &       37783         &       37783         &       35834         &       36818         &       37467         \\
\\
Males only&    -0.664         &    -0.035         &    -0.058         &      0.209         &     -3.977\sym{**} &     -0.447         \\
            &    (0.783)         &    (0.075)         &    (0.059)         &    (0.180)         &    (0.418)         &    (0.219)         \\

\(N\)       &       16640         &       16640         &       16933         &       16893         &       16038         &       16310         \\

Females only  &    0.309         &     0.097         &    -0.077\sym{**} &      0.195\sym{*}  &      1.650         &     -0.729         \\
            &    (0.885)         &    (0.108)         &    (0.013)         &    (0.046)         &    (1.448)         &    (0.274)         \\

\(N\)       &       21143         &       21143         &       21815         &       21767         &       20780         &       21157         \\




\hline

\textit{Dispossession $\times$ Transition period $\times$ Regime change}   &  & & & & &  \\
Total &       -0.503         &     0.0298         &   0.001         &     0.02113         &     -6.235         &      0.420         \\
            &    (1.529)         &    (0.137)         &    (0.006)         &    (0.038)         &    (2.544)         &    (0.221)         \\

\(N\)       &       38081         &       38081         &       38081         &       35979         &       36987         &       37634         \\
\\
Males only&  1.390\sym{**} &      0.138\sym{**} &      0.175\sym{**} &      0.141         &     -5.186\sym{**} &     0.080         \\
            &    (0.172)         &    (0.024)         &    (0.026)         &    (0.086)         &    (0.895)         &    (0.065)         \\

\(N\)       &       16826         &       16826         &       17126         &       17086         &       16136         &       16411         \\



Females only &     -1.545         &    -0.036         &    0.002         &      0.317         &     -7.148         &      0.715         \\
            &    (1.928)         &    (0.200)         &    (0.074)         &    (0.267)         &    (3.419)         &    (0.373)         \\

\(N\)       &       21255         &       21255         &       21932         &       21883         &       20851         &       21223         \\




\hline
\textit{Period of particular stress $\times$ Transition period $\times$ Regime change} & & & & & &  \\
Total &       0.342\sym{*}  &     0.020         & -0.000         &    -0.031\sym{**} &     -0.741         &     0.038         \\
            &    (0.107)         &    (0.011)         &    (0.002)         &    (0.007)         &    (0.308)         &    (0.053)         \\


\(N\)       &       37532         &       37532         &       37532         &       35651         &       36621         &       37258         \\
\\
Males only &      0.230         &     0.019         &    -0.013         &     0.049\sym{**} &     -0.542         &     0.022         \\
            &    (0.134)         &    (0.011)         &    (0.013)         &    (0.007)         &    (0.502)         &    (0.064)         \\

\(N\)       &       16520         &       16520         &       16799         &       16762         &       15941         &       16207         \\

Females only&      0.371\sym{*}  &     0.027         &    -0.012         &    -0.018         &     -0.823\sym{*}  &     0.065         \\
            &    (0.123)         &    (0.015)         &    (0.005)         &    (0.025)         &    (0.257)         &    (0.090)         \\

\(N\)       &       21012         &       21012         &       21662         &       21616         &       20680         &       21051         \\
\hline\\
Age & YES & YES& YES& YES& YES& YES \\ 
Sex & YES & YES& YES& YES& YES& YES \\ 
Childhood SES & YES & YES& YES& YES& YES& YES \\ 
Childhood health & YES & YES& YES& YES& YES& YES \\ 
Individual FE & YES & YES& YES& YES& YES& YES \\ 

\bottomrule
\multicolumn{7}{l}{\footnotesize  \textit{Source}: SHARE (waves 4, 5, 6) and SHARELIFE (waves 3, 7), release 7.0.0. }\\
\multicolumn{7}{l}{\footnotesize \textit{Notes}: Robust standard errors clustered at individual level in parentheses. Transition denotes period between 1983 and 1995. \sym{*} \(p<0.10\), \sym{**} \(p<0.05\), \sym{***} \(p<0.01\).}\\
\end{tabular}
\label{tab:during}
}
\\
}
\end{table}


 However, we find that dispossession in post-communist countries experienced by men during transformation significantly decreased the quality of later-life, increased BMI and the risks of obesity, and being chronically ill in later life, on top of the effects of sole regime change and the sole effect of the dispossession before or after the transition. While this effect is found mainly for men, we document detrimental impact of stress during the transformation on female quality of life in the future. Furthermore, we confirm existing literature contributing health deterioration to the experience of particular stress induced by the regime change with our findings on increased BMI. 

 
 We find that out of the examined shocks, hunger has the most pronounced differential impact on quality of life, reducing on average the scores at the CASP scale for individuals exposed to regime change by 10\% more than individuals living in non-communist countries. Persecution yields less damage than dispossession, as it increases the risk of being chronically ill in later life, and the effect is more pronounced (by almost 8\%) among individuals persecuted in the Soviet regimes than those persecuted in non-communist regimes. 

 
   SHARELIFE respondents were allowed to declare in addition to their exposure to personal shocks such as hunger, dispossession and persecution, whether they faced periods of particular stress. Consistently, we document evidence that such extraordinary periods experienced by individuals living in post-communist countries significantly detrimental to both their short- and long-term physical and mental health, and the role of stress in the case of obesity risk is similar to the effect of hunger in these countries. Importantly, experiencing extraordinary stressful times in post-communist countries yields effects on BMI, obesity, chronic illness and CASP of comparable sizes. We find that the exposure of self reported periods  of particular stress in post-communist countries increases depression symptoms in later life by 27\% on average. However, these figures should be interpreted with caution, given that personality traits associated with depression might be endogenous to the specific perceptions of experienced emotions. 

\subsection*{Robustness checks}
\label{sec:rob}
In this section, we test whether our results are robust using seven tests. Firstly, we check whether effects identified above using Mundlak \citep{wooldridge2021two} approach are robust to the change in the model's specification and estimation technique. In particular, we examine ordinary least squares regressions (OLS), random effects panel data estimation, and restrict the set of controls in fixed effects analysis by excluding any information on respondents' childhood circumstances with respect to their SES or health. Instead, we control for respondents' parental health-related behaviours (tobacco and alcohol consumption) and mental health problems. Secondly, we run placebo test for the shock and transition. Thirdly, we examine if the effects of countries transitioning from communism are robust to the exclusion of the South-European countries with right-wing authoritarian regimes ruling in the past (i.e. Greece, Portugal, Spain). Moreover, we investigate if the results are robust to the inclusion of time fixed effects or time trend. Then we proceed to test whether the effects on our measures of later-life health are consistent with the effects on other health measures and in various cohort groups. Furthermore, we exclude former USSR member states (Estonia, Hungary) from the research sample and test if the main results remain intact as well as if they are robust to the inclusion of migrants in the research sample.

Random effects (c.f. Table \ref{tab:RE}) and OLS analyses (c.f. Table \ref{tab:tcc} for pooled data and Table \ref{tab:w6ols} for wave 6 only) yield results consistent with the main analysis. Our additional estimates provide reassuring results that confirm the hypothesis of negative effects of health and personal shocks experienced by individuals living in previously communist countries using alternative set of controls, especially with respect to the effects of dispossession as well as extraordinary periods of stress (c.f. Table \ref{tab:rob1}). We also confirm main results on the shocks experienced during the transformation period, as documented in Table \ref{tab:rob2}. Furthermore, we reveal that the results obtained in the main analysis are not spurious because the placebo variables replacing actual shocks and transformation period, yield effects of all the shocks, the transition as well as their interaction indistinguishable from zero (c.f. Table \ref{tab:plac}). The exclusion of the South-European countries with the authoritarian regimes in the past, confirms the unique impact of communist regimes and their impact on the long lasting consequences of the shocks experiences in the past (c.f. Table \ref{tab:authorit}).

We show that the results are robust to the inclusion of time fixed effects or time trend in the analysis (Tables \ref{tab:shocks_tFE} and \ref{tab:shocks_ttrend}, respectively). The precision of the estimates increased slightly, which is mostly due to the larger number of explanatory variables, while the differences in the estimates are negligible and observed mostly only at a third decimal point level. 

Additionally, we confirm that the experience of personal shocks in formerly communist countries exerted more pronounced detrimental effects on various health measures, particularly mobility and functional status, self-reported health assessments, as well as diagnoses of diabetes and heart attacks. Individuals exposed to regime change also exhibit greater impacts in terms of mobility limitations, restrictions in daily living activities, and heightened risks of diabetes and heart attacks (c.f. Table \ref{tab:mech1}), which can be the mechanism driving the effects of such shocks on health and well-being in later life, consistently with the evidence of co-determination of mental health and physical health documented in the literature \citep{ohrnberger2017relationship}. This effect seems to be particularly strong for the role of hunger and to a smaller degree also for the periods of particular stress in one's life history.
  
However, communist regimes were not static and exhibited change. Hence in order to account for the developments of the communist regimes, we examine the heterogeneity of the effects using the research sample restricted to individuals living in formerly communist countries between the older cohorts (operationalized as being born before 1950) as opposed to younger ones (born between 1950 and 1965). Individuals who were born before 1950 were exposed to Stalinist period followed with revision of several institutions critical to the regime. Results from Table \ref{tab:coh} indeed confirm that the detrimental effects of the shocks are more pronounced for individuals belonging to older cohorts, particularly with respect to short-term health. Out of the examined shocks, only dispossession poses constant impact on long-term health and well-being across old and young cohorts. Moreover, we examine the sample excluding Estonia from the analysis to find the results presented in the Tables \ref{tab:estoniaout}. We document that the effects of the shocks do differ between communist non-member states and Western Europe significantly. Hence, the effect is not driven by the USSR only. 

Finally, we show that the exclusion of the migrants from the research sample does not affect qualitative results obtained in the main analysis (c.f. Table \ref{tab:migrants}).

\section*{\Large{Discussion}}
\label{sec:mech}

\subsection*{Mechanisms}

This paper has studied the effect of personal and health shocks in communist and non-communist regimes, including the shocks that occurred during the period of regime transition. Our additional analyses shed more light on the potential mechanisms underlying the documented effects by showing the heterogeneous effects of shocks on household net wealth, individual's retirement age, educational attainment and marital status, as well as hospitalization and out-of-pocket expenditures on health care (ambulatory and hospital).

 We have documented consistent and non-negligible effects of regime change on educational attainment, even though there is little evidence of negative effects of personal and health shocks on the probability of an individual's completed education (c.f. Table \ref{tab:mech2}). More specifically, the interaction between the shock and regime change yields independent and mostly positive effects on education. However we find negative effects on health and well-being in later life suggesting that communist education hardly protected individuals from increased health deterioration compared to Western European countries.


The above finding is consistent with the idea that neither formal education nor work experience captured by the retirement age in our analysis resulted in health improvements among older individuals in post-communist countries. However, we document that wealth deprivation and being single can be attributed to the detrimental effects of the shocks in post-communist countries on well-being and health in later life. Once more, our estimates confirm that shocks experienced in the past are detrimental for health in later life, especially if experienced by older individuals living in currently post-communist countries.

Results from Table \ref{tab:beh}  help to understand other mechanisms underlying our results, and particularly their effect on BMI, as dietary habits are largely dependent on the institutional environment. The food production under communism was deeply affected by the state of economy and not rarely rationed, which could be partially mitigated through informal ties providing access to alternative to the state-provided food supplies \citep{turnock1996agriculture}. Moreover, we document significant role of the experience of hunger in the past on the increase of consumption of all examined foods (dairy, eggs, fish or meat, and greens). Importantly, we find that all other shocks from the past (dispossession, persecution as well as periods of particular stress), are associated with more frequent consumption of dairy and greens. Taking into account that these shocks are associated also with larger BMI and increased likelihood of obesity, we might credit these effects to behavioural risks related to food (over)consumption.

Some evidence suggests that the organisation of the health system played a role too. We find evidence of detrimental effects of persecution and dispossession on hospitalization but not on the out-of-pocket expenditures on health in later life, as shown in Table \ref{tab:hexp}. These effects are more pronounced for individuals living in countries that transitioned to free markets from communism. Moreover, we find differential effects of experience of hunger in the past which significantly increases the probability of hospitalization in later life only in the formerly communist countries. These findings seem to be related to inefficiency of health care systems and lack of proper coordination of health care delivery relying heavily on the hospitals rather than ambulatory care \citep{sowa2016selected}.


Another mechanism refers to changes in household composition which can help accommodate personal and health shocks. Indeed, the association between past health shocks and the reduced likelihood of having a partner needs additional analysis for detailed interpretation, particularly with respect to the sequence of events. However, our results are in line with two bodies of literature: on the protective role of partner's (and in turn, children's) presence to health deterioration \citep{wood2009effects} and on the selection into marriage based on the increased health capital \citep{kiecolt2001marriage}.


Similarly,  the detrimental effects of hunger might be credited to physical inactivity most times accompanied by certain periods of vigorous activity \citep{westerterp2018changes}. The imbalance concerning various forms of physical activity might have contributed to deteriorate health in the long term, as any other deep irregularities relevant to health \citep{bell2017late, li2015variability}.  However, we do not find any relevant differences in unhealthy behaviours as nicotine consumption, between the shock effects experienced by individuals living in countries that changed regimes or not.

\subsection*{Heterogeneity}

When we explore gender effect, we find that the detrimental effects of shocks in post-communist countries on later life health and well-being are more pronounced for women than for men, as shown in Table \ref{tab:h_female}. We document more pronounced effects for women, consistently across the sole effect of each shock, exposure to communist regime, as well as their interaction, in line with the notion of worse health on average among women than men in post-communist countries (despite greater longevity). The observed gender heterogeneity is also in line with excess male mortality accompanied with more healthy life years in post-communist countries \citep{kingkade1992sex}.


The shocks that happened to individuals living in rural areas for whom we can reasonably assume they have lived in rural areas most of their lives in countries that experienced regime transition yield more pronounced effects on well-being and short-term health than same shocks in same countries but in urban areas, as shown in Table \ref{area_h}. We find consistently stronger detrimental effects of instability in stress levels in rural than in urban areas of countries exposed to communism. Also the direct impact of shocks is on average more harmful in rural than in urban areas. That is most likely a result of inequalities in the accessibility as well as in the quality of health care services throughout respondents life course.

We showed that the shocks occurring in the period of regime transition in Central and Eastern Europe do not differ from the shocks that occurred before 1982 or after 1995. However, the analysis of the cohort heterogeneity suggests that the detrimental health effects of the shocks, especially in the formerly communist countries, are more pronounced for individuals from the older cohorts (i.e. those born before 1950). These results are in line with the unique severity and cruelty of persecution, dispossession, hunger, and stress that occurred during the times of totalitarian and authoritarian regimes' raise in Europe in the first half of the previous century \citep{szalontai2002dynamic, karlsson2008crimes, morrock2014psychology, longerich2010holocaust}. The younger cohorts were free form the risk of being exposed to the most extreme shocks, and this is reflected in the heterogeneity documented in Table \ref{rev_coh}.

\section*{\Large{Conclusion}}
\label{sec:con}

This paper reports evidence of the detrimental effects on health and well-being in later life due to exposure to personal and health shocks across different regimes. It also highlights the specific impacts of regime change, drawing on longitudinal and retrospective data from the SHARE survey. We provide robust evidence of the negative consequences of adverse life events, such as hardship, hunger, and deprivation, over the life course, and the extent to which these effects are more pronounced in countries that experienced socio-political regime transformations. Importantly, our findings show that the transition from Soviet communism to capitalism in Europe amplified the negative effects of health shocks, particularly on physical health and overall well-being.

Consistently, we document that all shocks experienced by individuals living in formerly communist countries have negative effects on health and well-being, in addition to the smaller but still significant negative impact of the transition shock itself. We find that the differential effect of exposure to hunger affects quality of life indicators in old age  (reducing the CASP index for individuals exposed to regime change by 10\% more than for individuals living in non-communist countries). In contrast, persecution yields less detrimental effects than dispossession overall, but increases the prevalence of chronic illness in later life by almost 8\%  in the post-communist countries. Extraordinary periods experienced by individuals living in post-communist countries are significantly detrimental for short- and long-term physical and particularly mental health. Period of particular stress reported by respondents in post-communist countries increases depression symptoms in later-life by 27\% on average.  

 An additional finding is that these effects differ by gender. That is, all these effects documented for post-communist countries are  more pronounced for women than men. This can be explained by the fact that in countries exposed to Soviet communism women were more likely to be in employment, but at the same time communist societies kept some traditional gender norms, so it did not reduce household-related inequalities. Women in the Soviet bloc had to work long hours in physically demanding jobs, and also care for their household at home and their health, and well-being needs would take less of a priority compared to that of children and male counterparts \citep{harden2002beyond}. Furthermore, the health systems in communist countries were based on prevention \citep{sigerist2017essence} where the profile patient would be mainly male, and hence more likely to overlook female health care needs in comparison to Western countries. Finally, communist countries placed less weight to reproductive services and generally health care catered mainly for women. In addition, women are more prone to face mental disorders \citep{cockerham2006psychological}. These effects explain the gender heterogeneity in the detrimental effects of shocks experienced in the past by men and women from the Soviet bloc.

Another important finding is that these shocks led to increased food consumption, which, in turn, raised BMI and the risk of obesity, particularly among women. This, combined with the traditional age gap in marriages (making widowhood more likely among older women) and the substantial role of wealth accumulation for health and well-being in later life, helps explain not only how shocks and the Soviet regime affected health, but also why the impact differed significantly between genders. 

It is worth noting that increased educational attainment in post-communist countries did not provide protection against greater health deterioration compared to Western European countries. This is unsurprising, given the specific characteristics of the Soviet education system, which was free, public, centralized, and lacked a dynamic labor market, compounded by an inefficient healthcare system. Moreover, shocks experienced before, during, and after the transformation had a similar impact on later-life health. The Soviet period was far from stable, and periods of particular stress that led to destabilizing shocks were not confined to the transition period. In fact, despite the instability, volatility, and uncertainty during the transformation, only dispossession (for men) and exceptional levels of stress (for women) during the regime change had a more pronounced impact on later-life health than similar shocks under communism. This suggests that the inefficiency of the Semashko healthcare system may have been more detrimental to life-course health trajectories influenced by adverse shocks than the fragmented public health system and the unregulated rise of private healthcare services during the regime transformation in formerly communist Central and Eastern Europe.

Finally, our result suggest that the magnitude of the difference between the shocks' effects in communist and non-communist regime is larger if the research sample includes USSR member state. These results should be treated with caution, as the SHARE longitudinal data cover only one USSR member state, namely Estonia, hence we cannot generalize the result on the heterogeneity between all member and non-member states in the Soviet bloc. More studies are needed to confirm that the more pronounced difference in the effect of shocks in the Soviet Union than in other Soviet bloc countries are universal when comparing them to similar shocks that occurred in Western Europe. Further research of the heterogeneous shock effects across regimes should investigate the variation in the communist institution within the Soviet bloc countries. 


\normalsize



\newpage


\bibliography{HyBib.bib}



\newpage
\appendix{\Large{\textbf{Appendix}}}

\setcounter{page}{1}
\setcounter{table}{0} \renewcommand{\thetable}{A.\arabic{table}}
\onehalfspacing


\captionof{table}{Health shocks in SHARELIFE waves}
\label{tab:selection}
\resizebox{\textwidth}{!}{
{
\def\sym#1{\ifmmode^{#1}\else\(^{#1}\)\fi}
\begin{tabular}{l*{1}{cccccc}}
\toprule
 & \multicolumn{3}{c}{2008-2009 (wave 3)}& \multicolumn{3}{c}{2017 (wave 7)}\\
            &       \textit{N}&        Mean&          St.d.&    \textit{N}&        Mean&          St.d.\\
\hline
Age &  28471  &  66.005  &  10.051 &  76490  &  67.980    & 10.032 \\
Hunger      &    28351 &  0.079  &  0.270 & 60980 &0.050   &  0.218  \\
Dispossession      & 28351 &    0.042   & 0.202 &  61901 & 0.075  &  0.263\\
Persecution    &      28351 & 0.045   & 0.208 &  62032 &  0.053   & 0.224 \\
Period of particular stress      & 27765&    0.506 &   0.500 &  60462 &  0.469   & 0.499 \\
\bottomrule
\multicolumn{11}{l}{\footnotesize  \textit{Source}: SHARELIFE (waves 3, 7), release 7.0.0.}\\
\end{tabular}
}
\\
}\\
\newpage
\captionof{table}{Shock effects adjusted for age}
\resizebox{\textwidth}{!}{
{
\def\sym#1{\ifmmode^{#1}\else\(^{#1}\)\fi}
\begin{tabular}{l*{6}{c}}
\toprule
            &\multicolumn{2}{c}{Short-term health}&\multicolumn{2}{c}{Long-term health}&\multicolumn{2}{c}{Well-being}\\
            &\multicolumn{1}{c}{BMI}&\multicolumn{1}{c}{Obesity (1,0)}&\multicolumn{1}{c}{Chronically ill (1,0)}&\multicolumn{1}{c}{No. of chronic illnesses}&\multicolumn{1}{c}{CASP index}&\multicolumn{1}{c}{Depression scale}\\
\hline
\textit{Hunger}  &  & & & & &  \\ 
Total &      0.484\sym{***}&     0.050\sym{***}&   0.052\sym{***}&      0.292\sym{***}&     -2.807\sym{***}&      0.996\sym{***}\\
            &    (0.116)         &    (0.010)         &    (0.008)         &    (0.026)        &    (0.193)         &    (0.069)         \\
\(N\)       &       40387         &       40387         &       41493         &       41395         &       39236         &       39985         \\
Males only    &      0.187         &     0.019         &      0.033\sym{***}&      0.199\sym{***}&     -1.708\sym{***}&      0.525\sym{***}\\
            &    (0.153)         &    (0.015)         &  (0.012)         &    (0.037)        &    (0.261)         &    (0.092)         \\

\(N\)       &       16640         &       16640         &       18110         &       18067     &       16038         &       16310         \\
Females only  &      0.648\sym{***}&     0.067\sym{***}&     0.068\sym{***}&      0.364\sym{***}&     -2.747\sym{***}&      1.071\sym{***}\\
            &    (0.180)         &    (0.015)         &    (0.010)         &    (0.033)             &    (0.251)         &    (0.098)         \\
\(N\)       &       21143         &       21143         &      23383         &       23328         &       20780         &       21157         \\
\hline
\textit{Dispossession }  &  & & & & &  \\ 

Total &      0.711\sym{***}&     0.040\sym{***}&   0.052\sym{***}&      0.192\sym{***}&     -0.712\sym{***}&      0.230\sym{***}\\
            &    (0.093)         &    (0.009)         &    (0.008)         &    (0.026)          &    (0.157)         &    (0.051)         \\
\(N\)       &       40992         &       40992         &       42143         &       42044        &       39595         &       40345         \\

Males only   &      0.272\sym{*}  &     0.021         &     0.054\sym{***}&      0.183\sym{***}&     -0.819\sym{***}&      0.263\sym{***}\\
            &    (0.144)         &    (0.014)         &    (0.013)         &    (0.036)          &    (0.230)         &    (0.077)         \\
\(N\)       &       16826         &       16826         &     18490         &       18447         &       16136         &       16411         \\
Females only   &      0.562\sym{***}&     0.025\sym{*}  &    0.050\sym{***}&      0.197\sym{***}&     -0.520\sym{***}&     0.094         \\
            &    (0.151)         &    (0.014)         &   (0.010)         &    (0.035)           &    (0.188)         &    (0.071)         \\
\(N\)       &       21255         &       21255         &      23653         &       23597            &       20851         &       21223         \\

\hline
\textit{Persecution }  &  & & & & &  \\ 
Total &      0.493\sym{***}&     0.037\sym{***}&    0.065\sym{***}&      0.265\sym{***}      &     -1.412\sym{***}&      0.5237\sym{***}\\
            &    (0.134)         &    (0.011)         &  (0.009)         &    (0.026)     &    (0.171)         &    (0.066)         \\
\(N\)       &       41040         &       41040         &          42196         &       42096                 &       39633         &       40390         \\
Males only &      0.297\sym{*}  &     0.0444\sym{***}&     0.0601\sym{***}&      0.248\sym{***}     &     -1.467\sym{***}&      0.454\sym{***}\\
            &    (0.171)         &    (0.017)         &  (0.013)         &    (0.038)        &    (0.234)         &    (0.083)         \\
\(N\)       &       16838         &       16838         &     18505         &       18461          &       16145         &       16422         \\
Females only  &      0.806\sym{***}&     0.037\sym{**} &    0.070\sym{***}&      0.282\sym{***}         &     -1.497\sym{***}&      0.570\sym{***}\\
            &    (0.197)         &    (0.015)         &    (0.012)         &    (0.035)         &    (0.233)         &    (0.089)         \\
\(N\)       &       21275         &       21275         &      23691         &       23635          &       20865         &       21241         \\
\hline
%\textit{Period of particular happiness }  &  & & & & &  \\ 
%Total &     0.09341\sym{*}  &     0.01057\sym{***}&    0.04272\sym{***}&      0.1394\sym{***}&     -1.2679\sym{***}&      0.5216\sym{***}\\
%            &    (0.0480)         &    (0.0040)         &   (0.0044)         &    (0.0117)         &    (0.0758)         &    (0.0264)         \\
%\(N\)       &       39804         &       39804         &        40836         &       40748                &       38778         &       39498         \\
%Males only &     0.08943         &     0.01435\sym{**} &    0.03061\sym{***}&      0.1027\sym{***} &     -1.1314\sym{***}&      0.4330\sym{***}\\
 %           &    (0.0607)         &    (0.0061)         &   (0.0065)         &    (0.0168)        &    (0.1026)         &    (0.0343)         \\
%\(N\)       &       16454         &       16454         &       17803         &       17764        &       15889         &       16146         \\
%Females only &      0.1652\sym{**} &     0.01368\sym{**} &    0.05197\sym{***}&      0.1676\sym{***}&     -1.4830\sym{***}&      0.6193\sym{***}\\
 %           &    (0.0663)         &    (0.0057)         &      (0.0055)         &    (0.0150)        &    (0.0969)         &    (0.0348)         \\
%\(N\)       &       20939         &       20939         &      23033         &       22984         &       20613         &       20983         \\


\hline
\textit{Period of particular stress }   &  & & & & &  \\ 
Total &     0.016         &    0.008\sym{*}  &     0.060\sym{***}&      0.193\sym{***}     &     -0.698\sym{***}&      0.567\sym{***}\\
            &    (0.047)         &    (0.004)         &    (0.004)         &    (0.011)          &    (0.069)         &    (0.025)         \\
\(N\)       &       39977         &       39977         &      41022         &       40932          &       38934         &       39664         \\

Males only    &     0.03147         &    0.009433         &  0.05204\sym{***}&      0.161\sym{***}&     -0.662\sym{***}&      0.483\sym{***}\\
            &    (0.065)         &    (0.007)         &   (0.007)         &    (0.017)        &    (0.097)         &    (0.032)         \\
\(N\)       &       16520         &       16520         &     17891         &       17851      &       15941         &       16207         \\
Females only   &      0.104         &     0.014\sym{**} &  0.066\sym{***}&      0.217\sym{***}&     -0.895\sym{***}&      0.690\sym{***}\\
            &    (0.066)         &    (0.006)         &     (0.006)         &    (0.015)            &    (0.095)         &    (0.034)         \\
\(N\)       &       21012         &       21012         &       23131         &       23081          &       20680         &       21051         \\
\hline
\\
Age & YES & YES& YES& YES& YES& YES \\ 
Sex & YES & YES& YES& YES& YES& YES \\ 
Childhood SES & NO  & NO & NO & NO & NO & NO \\ 
Childhood health & NO & NO & NO & NO & NO & NO \\ 
Individual FE & NO & NO & NO & NO & NO & NO \\ 
\bottomrule
\multicolumn{7}{l}{\footnotesize  \textit{Source}: SHARE (waves 4, 5, 6) and SHARELIFE (waves 3, 7), release 7.0.0.}\\
\multicolumn{7}{l}{\footnotesize \textit{Notes}: Robust standard errors clustered at individual level in parentheses. \sym{*} \(p<0.10\), \sym{**} \(p<0.05\), \sym{***} \(p<0.01\).}\\
\label{tab:shocks}
\end{tabular}
}
\\
}

\captionof{table}{Regime change effects adjusted for age}
\resizebox{\textwidth}{!}{
{
\def\sym#1{\ifmmode^{#1}\else\(^{#1}\)\fi}
\begin{tabular}{l*{6}{c}}
\toprule
            &\multicolumn{2}{c}{Short-term health}&\multicolumn{2}{c}{Long-term health}&\multicolumn{2}{c}{Well-being}\\
            &\multicolumn{1}{c}{BMI}&\multicolumn{1}{c}{Obesity (1,0)}&\multicolumn{1}{c}{Chronically ill (1,0)}&\multicolumn{1}{c}{No. of chronic illnesses}&\multicolumn{1}{c}{CASP index}&\multicolumn{1}{c}{Depression scale}\\
\hline
\textit{Regime change} &  & & & & &  \\ 
Total  &      1.396\sym{***}&     0.098\sym{***}&       0.062\sym{***}&      0.207\sym{***}&     -1.565\sym{***}&      0.198\sym{***}\\
            &    (0.072)         &    (0.006)         &     (0.006)         &    (0.018)     &    (0.180)         &    (0.043)         \\
\(N\)       &       41258         &       41258         &       41258         &         42443         &       42339         &       40547         \\
Males only  &      0.984\sym{***}&     0.088\sym{***}&    0.057\sym{***}&      0.183\sym{***}&     -1.722\sym{***}&      0.242\sym{***}\\
            &    (0.082)         &    (0.007)         &   (0.008)         &    (0.022)        &    (0.188)         &    (0.045)         \\
\(N\)       &       18271         &       18271         &     18640         &       18592       &       17375         &       17700         \\
Females only &      1.694\sym{***}&      0.105\sym{***}&     0.065\sym{***}&      0.224\sym{***}&     -1.444\sym{***}&      0.164\sym{***}\\
            &    (0.096)         &    (0.008)         &     (0.007)         &    (0.021)          &    (0.196)         &    (0.052)         \\
\(N\)       &       22987         &       22987         &        23803         &       23747              &       22403         &       22847         \\
\hline
Age & YES & YES& YES& YES& YES& YES \\ 
Sex & YES & YES& YES& YES& YES& YES \\ 
Childhood SES & NO  & NO & NO & NO & NO & NO \\ 
Childhood health & NO & NO & NO & NO & NO & NO \\ 
Individual FE & NO & NO & NO & NO & NO & NO \\ 
\bottomrule
\multicolumn{7}{l}{\footnotesize  \textit{Source}: SHARE (waves 4, 5, 6) and SHARELIFE (waves 3, 7), release 7.0.0.}\\
\multicolumn{7}{l}{\footnotesize \textit{Notes}: Robust standard errors clustered at individual level in parentheses. \sym{*} \(p<0.10\), \sym{**} \(p<0.05\), \sym{***} \(p<0.01\).}\\
\label{tab:trans}
\end{tabular}
}

\\
}


\newpage
\captionof{table}{Shock and regime change effects in random effects panel estimation}
\resizebox{\textwidth}{!}{
{
\def\sym#1{\ifmmode^{#1}\else\(^{#1}\)\fi}
\begin{tabular}{l*{6}{c}}
\toprule
            &\multicolumn{2}{c}{Short-term health}&\multicolumn{2}{c}{Long-term health}&\multicolumn{2}{c}{Well-being}\\
            &\multicolumn{1}{c}{BMI}&\multicolumn{1}{c}{Obesity (1,0)}&\multicolumn{1}{c}{Chronically ill (1,0)}&\multicolumn{1}{c}{No. of chronic illnesses}&\multicolumn{1}{c}{CASP index}&\multicolumn{1}{c}{Depression scale}\\
\hline 
\\
Hunger $\times$ Regime change&     1.514\sym{***}&      0.129\sym{***}&     0.074\sym{**} &      0.447\sym{***}&     -4.059\sym{***}&      0.817\sym{**} \\
            &    (0.144)         &    (0.004)         &    (0.011)         &    (0.028)         &    (0.324)         &    (0.168)         \\

\\
Regime change&     1.378\sym{***}&      0.102\sym{***}&     0.052\sym{**} &      0.198\sym{***} &     -2.281\sym{***} &     0.065         \\
            &    (0.034)         &    (0.003)         &    (0.011)         &    (0.037)         &    (0.559)         &    (0.182)         \\


Hunger&      0.644\sym{***}&     0.058\sym{***}&     0.072\sym{**} &      0.307\sym{***} &     -2.622\sym{***} &      0.878\sym{***} \\
            &    (0.045)         &    (0.006)         &    (0.011)         &    (0.047)         &    (0.338)         &    (0.102)         \\

\(N\)       &       37783         &       37783         &       38748         &       38660         &       36818         &       37467         \\

\hline
\\
Dispossession $\times$ Regime change&   1.486\sym{***}&     0.100\sym{***}&     0.059\sym{***}&      0.266\sym{***}&     -2.250\sym{***} &      0.171         \\
            &    (0.068)         &    (0.004)         &    (0.005)         &    (0.007)         &    (0.400)         &    (0.138)         \\


\\
Regime change&       1.340\sym{***}&     0.100\sym{***}&     0.049\sym{***} &      0.186\sym{***} &     -2.230\sym{***} &     0.050         \\
            &    (0.030)         &    (0.002)         &    (0.010)         &    (0.039)         &    (0.574)         &    (0.185)         \\


Dispossession&   0.151\sym{*}         &    0.004         &     0.053\sym{***}  &      0.171\sym{***}         &     -0.229         &      0.261\sym{***} \\
            &    (0.078)         &    (0.005)         &    (0.015)         &    (0.062)         &    (0.474)         &    (0.068)         \\



\(N\)       &       38081         &       38081         &       39058         &       38969         &       36987         &       37634         \\

\hline
\\
Persecution $\times$ Regime change&     1.467\sym{***}&      0.114\sym{***}&     0.078\sym{***} &      0.343\sym{***}&     -2.900\sym{***} &      0.332\sym{*}         \\
            &    (0.037)         &    (0.003)         &    (0.013)         &    (0.025)         &    (0.452)         &    (0.188)         \\


\\
Regime change&      1.358\sym{***}&     0.100\sym{***}&     0.047\sym{***} &      0.182\sym{***} &     -2.211\sym{***} &     0.053         \\
            &    (0.029)         &    (0.003)         &    (0.010)         &    (0.036)         &    (0.571)         &    (0.183)         \\

Persecution&      0.459\sym{***}&     0.025\sym{**}         &     0.085\sym{***}&      0.260\sym{***} &     -1.443\sym{***} &      0.761\sym{***}\\
            &    (0.037)         &    (0.010)         &    (0.006)         &    (0.030)         &    (0.176)         &    (0.062)         \\


\(N\)       &       38113         &       38113         &       39090         &       39000         &       37010         &       37663         \\


\hline

\\
Period of particular stress $\times$ Regime change&      1.529\sym{***}&      0.120\sym{***}&      0.117\sym{***}&      0.424\sym{***}&     -3.293\sym{***} &      0.700\sym{***} \\
            &    (0.044)         &    (0.006)         &    (0.005)         &    (0.029)         &    (0.620)         &    (0.192)         \\


\\
Regime change&     1.226\sym{***}&     0.091\sym{***}&     0.052\sym{***} &      0.168\sym{***} &     -1.930\sym{*}  &     0.076         \\
            &    (0.051)         &    (0.001)         &    (0.008)         &    (0.036)         &    (0.632)         &    (0.184)         \\


Period of particular stress &    0.022         &    0.008\sym{**}         &     0.063\sym{***} &      0.175\sym{***}&     -0.594\sym{***} &      0.593\sym{***}\\
            &    (0.041)         &    (0.004)         &    (0.009)         &    (0.013)         &    (0.104)         &    (0.021)         \\

\(N\)       &       37532         &       37532         &       38461         &       38378         &       36621         &       37258         \\
\hline
Age & YES & YES& YES& YES& YES& YES \\ 
Sex & YES & YES& YES& YES& YES& YES \\ 
Childhood SES & YES & YES& YES& YES& YES& YES  \\ 
Childhood health & YES & YES& YES& YES& YES& YES  \\ 
Random effects & YES & YES& YES& YES& YES& YES  \\ 
\bottomrule
\multicolumn{6}{l}{\footnotesize  \textit{Source}: SHARE (waves 4, 5, 6) and SHARELIFE (waves 3, 7), release 7.0.0.}\\
\multicolumn{6}{l}{\footnotesize   \textit{Notes}: Robust standard errors clustered at individual in parentheses. \sym{*} \(p<0.10\), \sym{**} \(p<0.05\), \sym{***} \(p<0.01\).}\\
\label{tab:RE}
\end{tabular}
}

\\
}\\




\newpage
\captionof{table}{Shock and regime change effects in pooled cross-section estimation with region-specific time trends}
\resizebox{\textwidth}{!}{
{
\def\sym#1{\ifmmode^{#1}\else\(^{#1}\)\fi}
\begin{tabular}{l*{6}{c}}
\toprule
            &\multicolumn{2}{c}{Short-term health}&\multicolumn{2}{c}{Long-term health}&\multicolumn{2}{c}{Well-being}\\
            &\multicolumn{1}{c}{BMI}&\multicolumn{1}{c}{Obesity (1,0)}&\multicolumn{1}{c}{Chronically ill (1,0)}&\multicolumn{1}{c}{No. of chronic illnesses}&\multicolumn{1}{c}{CASP index}&\multicolumn{1}{c}{Depression scale}\\
\hline 
\hline 

\\
Hunger $\times$ Regime change&      1.059\sym{***}&      0.102\sym{***}&   0.089\sym{***}         &  0.484\sym{***}         &     -4.881\sym{***}&      1.014\sym{***}\\
            &    (0.219)         &    (0.021)         &    (0.015)         &    (0.053)         &    (0.363)         &    (0.123)         \\
\\
Regime change&      0.921\sym{***}&      0.076\sym{***}&   0.066\sym{***}&     0.232\sym{***}&     -3.128\sym{***}&     0.290\sym{***}         \\
            &    (0.128)         &    (0.011)         &    (0.010)         &    (0.028)         &    (0.240)         &    (0.055)         \\
 
Hunger&      0.529\sym{***}&     0.025\sym{***}&    0.061\sym{***}         &    0.266\sym{***}&     -1.877\sym{***}&      0.769\sym{***}\\
            &    (0.151)         &    (0.013)         &    (0.011)         &    (0.034)         &    (0.225)         &    (0.090)         \\
\(N\)       &       37783         &       37783         &       37783         &       35834         &       36818         &       37467         \\

\hline
\\
Dispossession $\times$ Regime change&      1.020\sym{***}&      0.072\sym{***}&    0.073\sym{***}&     0.300\sym{***}&     -3.390\sym{***}&      0.428\sym{**} \\
            &    (0.160)         &    (0.016)         &    (0.013)         &    (0.042)         &    (0.270)         &    (0.074)         \\
\\
Regime change&      0.904\sym{***}&      0.074\sym{***}&   0.061\sym{***}&     0.216\sym{***}&     -3.055\sym{***}&     0.263\sym{***}         \\
            &    (0.126)         &    (0.011)         &    (0.009)         &    (0.028)         &    (0.244)         &    (0.057)         \\
            
Dispossession&      0.031         &    0.006         &    0.037\sym{**}         &    0.122\sym{**}         &     0.012         &      0.176\sym{*}\\
            &    (0.192)         &    (0.018)         &    (0.017)         &    (0.052)         &    (0.280)         &    (0.095)         \\

\(N\)       &       38081         &       38081         &       38081         &       35979         &       36987         &       37634         \\

\hline
\\

Persecution $\times$ Regime change&      1.004\sym{***}&      0.087\sym{***}&   0.088\sym{**} &     0.364\sym{***}&     -3.670\sym{***}&      0.544\sym{***}\\
            &    (0.218)         &    (0.019)         &    (0.015)         &    (0.044)         &    (0.327)         &    (0.097)         \\
\\

Regime change&      0.921\sym{***}&     0.073\sym{***}&   0.060\sym{***}&     0.211\sym{***}&     -3.071\sym{***}&     0.264\sym{***}         \\
            &    (0.125)         &    (0.011)         &    (0.010)         &    (0.028)         &    (0.244)         &    (0.057)         \\
Persecution&      0.451\sym{**} &     0.021  &    0.078\sym{***}         &    0.245\sym{***}         &     -1.759\sym{***}&      0.755\sym{***}\\
            &    (0.195)         &    (0.015)         &    (0.014)         &    (0.038)         &    (0.252)         &    (0.085)         \\

\(N\)       &       38113         &       38113         &       38113         &       36006         &       37010         &       37663         \\

\hline
\\
Period of particular stress $\times$ Regime change&      1.071\sym{***}&      0.091\sym{***}&   0.124\sym{***}&     0.434\sym{***}&     -4.161\sym{***}&      0.871\sym{***}\\
            &    (0.134)         &    (0.012)         &    (0.011)         &    (0.031)         &    (0.255)         &    (0.064)         \\
\\
Regime change&      0.755\sym{***}&     0.069\sym{***}&   0.060\sym{**} &     0.183\sym{***}&     -3.060\sym{***}&     0.280\sym{***}         \\
            &    (0.139)         &    (0.012)         &    (0.011)         &    (0.031)         &    (0.261)         &    (0.064)         \\

Period of particular stress &     0.039         &    0.005         &  0.060\sym{***}         &    0.169\sym{**} &     -1.079\sym{***}&      0.616\sym{***}\\
            &    (0.058)         &    (0.005)         &    (0.006)         &    (0.014)         &    (0.075)         &    (0.030)         \\
\(N\)       &       37532         &       37532         &       37532         &       35651         &       36621         &       37258         \\
\hline\\
Age & YES & YES& YES& YES& YES& YES \\ 
Gender & YES & YES& YES& YES& YES& YES \\ 
Childhood SES & YES & YES& YES& YES& YES& YES  \\ 
Childhood health & YES & YES& YES& YES& YES& YES  \\
Region-specific time trend & YES & YES& YES& YES& YES& YES  \\ 
\bottomrule
\multicolumn{67}{l}{\footnotesize  \textit{Source}: SHARE (waves 4, 5, 6) and SHARELIFE (waves 3, 7), release 7.0.0.}\\
\multicolumn{6}{l}{\footnotesize   \textit{Notes}: Robust standard errors in parentheses. \sym{*} \(p<0.10\), \sym{**} \(p<0.05\), \sym{***} \(p<0.01\).}\\
\label{tab:tcc}
\end{tabular}
}

\\
}\\
\scriptsize{Regions: West - France, Belgium, Luxemburg, the Netherlands; North - Sweden, Denmark, Switzerland; South - Italy, Spain, Portugal, Greece, Croatia; Central (reference group) - Austria, Czech Republic, Estonia, Latvia, Lithuania, Slovakia, Poland.}




\newpage
\captionof{table}{Shock and regime change effects in cross-section estimation using most recent SHARE wave}
\resizebox{\textwidth}{!}{
{
\def\sym#1{\ifmmode^{#1}\else\(^{#1}\)\fi}
\begin{tabular}{l*{6}{c}}
\toprule
            &\multicolumn{2}{c}{Short-term health}&\multicolumn{2}{c}{Long-term health}&\multicolumn{2}{c}{Well-being}\\
            &\multicolumn{1}{c}{BMI}&\multicolumn{1}{c}{Obesity (1,0)}&\multicolumn{1}{c}{Chronically ill (1,0)}&\multicolumn{1}{c}{No. of chronic illnesses}&\multicolumn{1}{c}{CASP index}&\multicolumn{1}{c}{Depression scale}\\
\hline 
\hline 
\\

Hunger $\times$ Regime change&      1.579\sym{***}&      0.127\sym{***}&   0.069\sym{***}         &  0.423\sym{***}         &     -3.818\sym{***}&      0.692\sym{***}\\
            &    (0.214)         &    (0.022)         &    (0.016)         &    (0.056)         &    (0.372)         &    (0.136)         \\
\\
Regime change&      1.354\sym{***}&      0.100\sym{***}&   0.043\sym{***}&     0.167\sym{***}&     -1.830\sym{***}&     -0.076         \\
            &    (0.112)         &    (0.010)         &    (0.009)         &    (0.028)         &    (0.177)         &    (0.048)         \\

Hunger&      0.653\sym{***}&     0.064\sym{***}&    0.065\sym{***}         &    0.270\sym{***}&     -2.950\sym{***}&      0.961\sym{***}\\
            &    (0.175)         &    (0.015)         &    (0.013)         &    (0.041)         &    (0.284)         &    (0.105)         \\
\(N\)       &       29726         &       29726         &       30306         &       30271         &       28993         &       29376         \\
\hline
\\

Dispossession $\times$ Regime change&      1.419\sym{***}&      0.095\sym{***}&    0.054\sym{***}&     0.261\sym{***}&     -1.947\sym{***}&      0.044 \\
            &    (0.147)         &    (0.015)         &    (0.013)         &    (0.045)         &    (0.226)         &    (0.069)         \\
\\

Regime change&      1.326\sym{***}&      0.098\sym{***}&   0.040\sym{***}&     0.153\sym{***}&     -1.762\sym{***}&     -0.095\sym{*}         \\
            &    (0.113)         &    (0.010)         &    (0.009)         &    (0.027)         &    (0.183)         &    (0.051)         \\

Dispossession&      0.089         &    0.001         &    0.041\sym{**}         &    0.128\sym{**}         &     0.093         &      0.314\sym{***}\\
            &    (0.213)         &    (0.019)         &    (0.019)         &    (0.056)         &    (0.321)         &    (0.110)         \\

\(N\)       &       29975         &       29975         &       30564         &       30527         &       29120         &       29497         \\


\hline
\\

Persecution $\times$ Regime change&      1.426\sym{***}&      0.115\sym{***}&   0.070\sym{***} &     0.325\sym{***}&     -2.531\sym{***}&      0.198\sym{***}\\
            &    (0.232)         &    (0.020)         &    (0.015)         &    (0.047)         &    (0.275)         &    (0.095)         \\
\\
Regime change&      1.344\sym{***}&     0.097\sym{***}&   0.040\sym{***}&     0.152\sym{***}&     -1.747\sym{***}&     -0.092\sym{*}         \\
            &    (0.110)         &    (0.009)         &    (0.009)         &    (0.027)         &    (0.183)         &    (0.050)         \\

Persecution&      0.467\sym{**} &     0.033\sym{*}  &    0.080\sym{***}         &    0.235\sym{***}         &     -1.336\sym{***}&      0.747\sym{***}\\
            &    (0.212)         &    (0.017)         &    (0.016)         &    (0.043)         &    (0.296)         &    (0.101)         \\

\(N\)       &       30001         &       30001         &       30590         &       30553         &       29138         &       29519         \\

\hline
\\

Period of particular stress $\times$ Regime change&      1.498\sym{***}&      0.116\sym{***}&   0.113\sym{***}&     0.400\sym{***}&     -2.291\sym{***}&      0.555\sym{***}\\
            &    (0.125)         &    (0.011)         &    (0.010)         &    (0.031)         &    (0.202)         &    (0.059)         \\
\\
Regime change&      1.240\sym{***}&     0.091\sym{***}&   0.046\sym{***} &     0.138\sym{***}&     -1.423\sym{***}&     -0.069         \\
            &    (0.123)         &    (0.011)         &    (0.010)         &    (0.031)         &    (0.202)         &    (0.055)         \\

Period of particular stress &     0.051         &    0.010\sym{*}        &  0.071\sym{***}         &    0.186\sym{***} &     -0.523\sym{***}&      0.608\sym{***}\\
            &    (0.067)         &    (0.006)         &    (0.007)         &    (0.017)         &    (0.096)         &    (0.034)         \\
\(N\)       &       29561         &       29561         &       30128         &       30094         &       28860         &       29236         \\
\hline
Age & YES & YES& YES& YES& YES& YES \\ 
Gender & YES & YES& YES& YES& YES& YES \\ 
Childhood SES & YES & YES& YES& YES& YES& YES  \\ 
Childhood health & YES & YES& YES& YES& YES& YES  \\ 
\bottomrule
\multicolumn{6}{l}{\footnotesize  \textit{Source}: SHARE (wave 6) and SHARELIFE (waves 3, 7), release 7.0.0.}\\
\multicolumn{6}{l}{\footnotesize   \textit{Notes}: Robust standard errors in parentheses. \sym{*} \(p<0.10\), \sym{**} \(p<0.05\), \sym{***} \(p<0.01\).}\\
\label{tab:w6ols}
\end{tabular}
}

\\
}\\

\newpage
\captionof{table}{Robustness of shock effects and regime change}
\resizebox{\textwidth}{!}{
{
\def\sym#1{\ifmmode^{#1}\else\(^{#1}\)\fi}
\begin{tabular}{l*{6}{c}}
\toprule
            &\multicolumn{2}{c}{Short-term health}&\multicolumn{2}{c}{Long-term health}&\multicolumn{2}{c}{Well-being}\\
            &\multicolumn{1}{c}{BMI}&\multicolumn{1}{c}{Obesity (1,0)}&\multicolumn{1}{c}{Chronically ill (1,0)}&\multicolumn{1}{c}{No. of chronic illnesses}&\multicolumn{1}{c}{CASP index}&\multicolumn{1}{c}{Depression scale}\\
\hline 
\\
Hunger $\times$ Regime change&      1.119         &      0.114         &     0.058         &      0.273         &           -         &           -        \\
            &    (0.859)         &    (0.110)         &    (0.025)         &    (0.120)         &         -         &         -         \\

\\
Regime change&         1.023\sym{*}  &     0.083\sym{**} &     0.071         &      0.312\sym{*}  &      2.763\sym{***}&     -1.001\sym{***}\\
            &    (0.264)         &    (0.013)         &    (0.026)         &    (0.083)         &    (0.869)         &    (0.364)         \\




Hunger&     0.585\sym{*}  &     0.0412\sym{***}&     0.074\sym{***}&      0.303\sym{**} &     -3.826\sym{***}&      1.118\sym{***}\\
            &    (0.170)         &    (0.002)         &    (0.006)         &    (0.070)         &    (0.362)         &    (0.124)         \\


\(N\)       &        7045         &        7045         &        7522         &        7473         &        6681         &        6959         \\

\hline
\\
Dispossession $\times$ Regime change&     1.707\sym{*}  &      0.127\sym{*}  &     0.090         &      0.358\sym{**} &           -         &           -         \\
            &    (0.473)         &    (0.030)         &    (0.056)         &    (0.038)         &       -         &        -         \\




\\
Regime change&       0.930\sym{*}  &     0.079\sym{*}  &     0.064\sym{*}  &      0.285\sym{*}  &      0.648         &     -0.514         \\
            &    (0.293)         &    (0.020)         &    (0.022)         &    (0.081)         &    (0.813)         &    (0.372)         \\




Dispossession&     0.181         &     0.023         &      0.103\sym{*}  &      0.2721         &     -1.152\sym{*}  &      0.245         \\
            &    (0.195)         &    (0.038)         &    (0.035)         &    (0.144)         &    (0.592)         &    (0.220)         \\


\(N\)       &        7044         &        7044         &        7519         &        7470         &        6679         &        6956         \\



\hline
\\
Persecution $\times$ Regime change&       0.804         &     0.074\sym{*}  &     0.049         &      0.142         &          -         &           -         \\
            &    (0.684)         &    (0.021)         &    (0.070)         &    (0.151)         &         -         &         -        \\


\\
Regime change&         1.008\sym{*}  &     0.083\sym{*}  &     0.070\sym{*}  &      0.313\sym{*}  &      1.422\sym{*}  &     -0.595\sym{*}  \\
            &    (0.255)         &    (0.020)         &    (0.020)         &    (0.074)         &    (0.812)         &    (0.312)         \\

Persecution&       -0.031         &    -0.011         &     0.085\sym{**} &      0.317\sym{***}&     -2.285\sym{***}&      0.707\sym{***}\\
            &    (0.152)         &    (0.005)         &    (0.012)         &    (0.022)         &    (0.526)         &    (0.174)         \\

\(N\)       &        7043         &        7043         &        7518         &        7469         &        6680         &        6957         \\


\hline
%\\
%Period of particular happiness $\times$ Regime change&      1.3379\sym{**} &     0.08594\sym{***}&     0.09345\sym{*}  &      0.3764\sym{**} &          -         &           -        \\
%            &    (0.1843)         &    (0.0031)         &    (0.0236)         &    (0.0484)         &         -        &         -        \\

%\\
%Regime change&        0.6084         &     0.06628         &     0.07388\sym{*}  &      0.3188         &     0.05749         &    -0.03904\sym{*}  \\
%            &    (0.3679)         &    (0.0263)         &    (0.0240)         &    (0.1123)         &    (0.5187)         &    (0.0120)         \\

%Period of particular happiness &     -0.1043         &   -0.009621         &     0.03855\sym{**} &      0.1207\sym{**} &     -1.4847\sym{*}  &      0.5679\sym{**} \\
 %           &    (0.0983)         &    (0.0097)         &    (0.0042)         &    (0.0182)         &    (0.3866)         &    (0.1169)         \\

%\(N\)       &        6889         &        6889         &        7326         &        7282         &        6585         &        6853         \\

%\hline
\\
Period of particular stress $\times$ Regime change&    1.208\sym{*}  &      0.106\sym{**} &      0.118\sym{**} &      0.412\sym{**} &           -         &           -         \\
            &    (0.311)         &    (0.022)         &    (0.027)         &    (0.053)         &         -         &         -         \\


\\
Regime change&     0.905         &     0.070\sym{*}  &     0.068         &      0.334\sym{*}  &      0.636         &     -0.362         \\
            &    (0.343)         &    (0.018)         &    (0.029)         &    (0.093)         &    (0.646)         &    (0.152)         \\


Period of particular stress &      0.052         &    0.007\sym{**} &     0.045\sym{**} &      0.137\sym{***}&     -0.549         &      0.420\sym{**} \\
            &    (0.018)         &    (0.001)         &    (0.009)         &    (0.008)         &    (0.225)         &    (0.048)         \\


\(N\)       &        6911         &        6911         &        7349         &        7305         &        6599         &        6870         \\

\hline
Age & YES & YES& YES& YES& YES& YES \\ 
Gender & YES & YES& YES& YES& YES& YES \\ 
Childhood SES & NO & NO & NO & NO & NO & NO  \\ 
Childhood health & NO & NO & NO & NO & NO & NO  \\ 
Individual FE & YES & YES& YES& YES& YES& YES \\ 
Parent's features & YES & YES& YES& YES& YES& YES \\ 
\bottomrule
\multicolumn{6}{l}{\footnotesize  \textit{Source}: SHARE (waves 4, 5, 6) and SHARELIFE (waves 3, 7), release 7.0.0.}\\
\multicolumn{6}{l}{\footnotesize   \textit{Notes}: Robust standard errors clustered at individual level in parentheses. \sym{*} \(p<0.10\), \sym{**} \(p<0.05\), \sym{***} \(p<0.01\). }\\
\label{tab:rob1}
\end{tabular}
}
\\
}\\

Parent's features include: smoking, alcohol consumption and mental health problems

\newpage

\captionof{table}{Robustness of shock effects by their time of occurrence and regime change}
\resizebox{\textwidth}{!}{
{
\def\sym#1{\ifmmode^{#1}\else\(^{#1}\)\fi}
\begin{tabular}{l*{6}{c}}
\toprule
            &\multicolumn{2}{c}{Short-term health}&\multicolumn{2}{c}{Long-term health}&\multicolumn{2}{c}{Well-being}\\
            &\multicolumn{1}{c}{BMI}&\multicolumn{1}{c}{Obesity (1,0)}&\multicolumn{1}{c}{Chronically ill (1,0)}&\multicolumn{1}{c}{No. of chronic illnesses}&\multicolumn{1}{c}{CASP index}&\multicolumn{1}{c}{Depression scale}\\
\hline
\textit{Hunger $\times$ Transition period $\times$ Regime change}  &  & & & & &  \\ 

Total &       0.412         &     0.051         &     -0.131         &      0.134         &      1.941         &     -1.535\sym{**} \\
            &    (5.936)         &    (0.497)         &    (0.081)         &    (0.269)         &    (5.0450)         &    (0.267)         \\


\(N\)       &        7045         &        7045         &        7522         &        7473         &        6681         &        6959         \\
Males only&    -4.937         &     -0.567         &     -0.408         &     -0.560         &     -6.928         &     -4.431\sym{*}  \\
            &    (2.384)         &    (0.246)         &    (0.253)         &    (0.567)         &    (5.022)         &    (1.437)         \\


\(N\)       &        3259         &        3259         &        3444         &        3424         &        3085         &        3201         \\

Females only  &      2.521         &      0.321         &    -0.042\sym{**} &      0.370         &      3.556         &     -0.602         \\
            &    (6.664)         &    (0.557)         &    (0.006)         &    (0.257)         &    (6.718)         &    (1.896)         \\

\(N\)       &        3786         &        3786         &        4078         &        4049         &        3595         &        3757         \\




\hline

\textit{Dispossession $\times$ Transition period $\times$ Regime change}^{a)}  &  & & & & &  \\
Total &   -0.620         &    0.004         &     0.059         &      0.296         &     -4.853         &      0.206         \\
            &    (1.287)         &    (0.120)         &    (0.058)         &    (0.195)         &    (2.449)         &    (0.253)         \\

\(N\)       &       38081         &       38081         &       38081         &       35979         &       36987         &       37634         \\
Males only&   1.486\sym{**} &      0.137\sym{*}  &      0.160\sym{**} &      0.109         &     -4.780\sym{**} &     -0.118         \\
            &    (0.216)         &    (0.036)         &    (0.023)         &    (0.103)         &    (0.831)         &    (0.107)         \\


\(N\)       &       18134         &       18134         &       18490         &       18447         &       17282         &       17598         \\



Females only &    -1.565         &    -0.062         &     0.038         &      0.461         &     -5.178         &      0.452         \\
            &    (1.663)         &    (0.179)         &    (0.055)         &    (0.232)         &    (3.137)         &    (0.418)         \\

\(N\)       &       22858         &       22858         &       23653         &       23597         &       22313         &       22747         \\




\hline
%\textit{Period of particular happiness $\times$ Transition period $\times$ Regime change} &  & & & & &  \\
%Total&      -1.1122         &    -0.07689         &    -0.02308         &     -0.1096         &     -0.6512         &     -0.1194         \\
 %           &    (0.5905)         &    (0.0382)         &    (0.0398)         &    (0.0519)         &    (0.6600)         &    (0.2752)         \\



%\(N\)       &        6889         &        6889         &        7326         &        7282         &        6584         &        6852         \\
%Males only &      -2.1713         &     -0.1779         &    -0.03585         &     -0.2722         &     -0.3214         &     -0.3311         \\
 %           &    (1.2018)         &    (0.0985)         &    (0.0544)         &    (0.1035)         &    (1.0396)         &    (0.4248)         \\

%\(N\)       &        3184         &        3184         &        3352         &        3333         &        3028         &        3139         \\

%Females only &       -0.3735         &   -0.008348         &    -0.01492         &    0.009982         &     -1.0006         &     0.06477         \\
 %           &    (0.1538)         &    (0.0065)         &    (0.0405)         &    (0.0281)         &    (0.8543)         &    (0.3647)         \\

%\(N\)       &        3705         &        3705         &        3974         &        3949         &        3556         &        3713         \\


%\hline
\textit{Period of particular stress $\times$ Transition period $\times$ Regime change} & & & & & &  \\
Total &        0.605         &  -0.000         &     0.014         &     0.048         &     -0.527         &      0.337\sym{**} \\
            &    (0.339)         &    (0.037)         &    (0.038)         &    (0.046)         &    (0.462)         &    (0.061)         \\


\(N\)       &        6911         &        6911         &        7349         &        7305         &        6598         &        6869         \\
Males only &        0.544         &    -0.041         &    -0.045         &    -0.094         &      0.723         &      0.450         \\
            &    (0.316)         &    (0.025)         &    (0.073)         &    (0.214)         &    (0.726)         &    (0.215)         \\


\(N\)       &        3193         &        3193         &        3362         &        3343         &        3035         &        3148         \\

Females only&       0.663         &     0.034         &     0.053\sym{*}  &      0.133         &     -1.430\sym{*}  &      0.273         \\
            &    (0.372)         &    (0.083)         &    (0.014)         &    (0.084)         &    (0.476)         &    (0.254)         \\

\(N\)       &        3718         &        3718         &        3987         &        3962         &        3563         &        3721         \\



\hline
Age  & YES & YES& YES& YES& YES& YES \\ 
Gender & YES & YES& YES& YES& YES& YES \\ 
Childhood SES & NO & NO & NO & NO & NO & NO  \\ 
Childhood health & NO & NO & NO & NO & NO & NO  \\ 
Individual FE & YES & YES& YES& YES& YES& YES \\ 

\bottomrule
\multicolumn{7}{l}{\footnotesize  \textit{Source}: SHARE (waves 4, 5, 6) and SHARELIFE (waves 3, 7), release 7.0.0. }\\
\multicolumn{7}{l}{\footnotesize \textit{Notes}: Robust standard errors clustered at individual level in parentheses. Transition denotes period between 1983 and 1995. \sym{*} \(p<0.10\), \sym{**} \(p<0.05\), \sym{***} \(p<0.01\).}\\

\end{tabular}
\label{tab:rob2}
}
\\
}
\newpage

\captionof{table}{Placebo test for shock effects and for regime change}
\resizebox{\textwidth}{!}{
{
\def\sym#1{\ifmmode^{#1}\else\(^{#1}\)\fi}
\begin{tabular}{l*{6}{c}}
\toprule
            &\multicolumn{2}{c}{Short-term health}&\multicolumn{2}{c}{Long-term health}&\multicolumn{2}{c}{Well-being}\\
            &\multicolumn{1}{c}{BMI}&\multicolumn{1}{c}{Obesity (1,0)}&\multicolumn{1}{c}{Chronically ill (1,0)}&\multicolumn{1}{c}{No. of chronic illnesses}&\multicolumn{1}{c}{CASP index}&\multicolumn{1}{c}{Depression scale}\\
\hline 
\\ Total
\\
Placebo shock $\times$ Regime change&    0.064         &    0.001         &    0.008         &     0.011         &    -0.098         &    -0.045         \\
            &    (0.028)         &    (0.002)         &    (0.004)         &    (0.009)         &    (0.062)         &    (0.025)         \\



\\
Regime change&   -0.038         &   -0.001         &    0.005         &   -0.002         &    -0.0404         &     0.021         \\
            &    (0.019)         &    (0.003)         &    (0.004)         &    (0.012)         &    (0.051)         &    (0.011)         \\


Placebo shock &   -0.062         &   0.000         &    0.002         &    0.003151         &    -0.080         &     0.015         \\
            &    (0.066)         &    (0.003)         &    (0.003)         &    (0.005)         &    (0.097)         &    (0.013)         \\

\(N\)       &       38152         &       38152         &       39132         &       39042         &       37041         &       37694         \\

\hline
\\Men
\\

Placebo shock $\times$ Regime change&      0.150\sym{*}  &    0.004         &    0.006         &    0.006         &     0.033         &    -0.016         \\
            &    (0.048)         &    (0.007)         &    (0.005)         &    (0.014)         &    (0.096)         &    (0.018)         \\

\\
Regime change&     0.074         &    0.003         &   -0.009         &    -0.01231         &     0.040         &     0.016         \\
            &    (0.077)         &    (0.010)         &    (0.006)         &    (0.022)         &    (0.123)         &    (0.026)         \\



Placebo shock&     -0.047         &   -0.003         &   -0.009         &    0.003740         &    -0.038         &     0.039         \\
            &    (0.070)         &    (0.008)         &    (0.010)         &    (0.032)         &    (0.084)         &    (0.017)         \\

\(N\)       &       16860         &       16860         &       17161         &       17120         &       16163         &       16440         \\


\hline
\\Women
\\

Placebo shock $\times$ Regime change&     -0.007         &  -0.001         &     0.01058         &     0.014         &     -0.205\sym{*}  &    -0.067         \\
            &    (0.067)         &    (0.006)         &    (0.005)         &    (0.012)         &    (0.053)         &    (0.031)         \\

\\
Regime change&     -0.125         &   -0.005         &     0.016\sym{**} &    0.006385         &     -0.108         &     0.026         \\
            &    (0.078)         &    (0.008)         &    (0.003)         &    (0.005)         &    (0.085)         &    (0.028)         \\


Placebo shock &    -0.077         &    0.002         &     0.010         &    0.002         &     -0.110         &   -0.005         \\
            &    (0.153)         &    (0.010)         &    (0.004)         &    (0.021)         &    (0.112)         &    (0.024)         \\


\(N\)       &       21292         &       21292         &       21971         &       21922         &       20878         &       21254         \\



\hline
Age & YES & YES& YES& YES& YES& YES \\ 
Gender & YES & YES& YES& YES& YES& YES \\ 
Childhood SES & YES & YES& YES& YES& YES& YES  \\ 
Childhood health & YES & YES& YES& YES& YES& YES  \\ 
Individual FE & YES & YES& YES& YES& YES& YES  \\ 
\bottomrule
\multicolumn{67}{l}{\footnotesize  \textit{Source}: SHARE (waves 4, 5, 6) and SHARELIFE (waves 3, 7), release 7.0.0.}\\
\multicolumn{6}{l}{\footnotesize   \textit{Notes}: Robust standard errors clustered at individual level in parentheses. \sym{*} \(p<0.10\), \sym{**} \(p<0.05\), \sym{***} \(p<0.01\).}\\
\label{tab:plac}
\end{tabular}
}

\\
}

\newpage
\captionof{table}{Shock effects excluding South-European countries with authoritarian past}
\resizebox{\textwidth}{!}{
{
\def\sym#1{\ifmmode^{#1}\else\(^{#1}\)\fi}
\begin{tabular}{l*{6}{c}}
\toprule
            &\multicolumn{2}{c}{Short-term health}&\multicolumn{2}{c}{Long-term health}&\multicolumn{2}{c}{Well-being}\\
            &\multicolumn{1}{c}{BMI}&\multicolumn{1}{c}{Obesity (1,0)}&\multicolumn{1}{c}{Chronically ill (1,0)}&\multicolumn{1}{c}{No. of chronic illnesses}&\multicolumn{1}{c}{CASP index}&\multicolumn{1}{c}{Depression scale}\\
\hline 
\\
Hunger $\times$ Regime change&    1.727\sym{***}&      0.139\sym{***}&     0.093\sym{***}&      0.514\sym{***}&     -4.989\sym{***}&      0.908\sym{**} \\
            &    (0.138)         &    (0.002)         &    (0.006)         &    (0.007)         &    (0.171)         &    (0.107)         \\
\\
Regime change&          1.560\sym{***}&      0.109\sym{***}&     0.068\sym{***}&      0.253\sym{***}&     -3.150\sym{**} &      0.131         \\
            &    (0.008)         &    (0.003)         &    (0.006)         &    (0.020)         &    (0.325)         &    (0.139)         \\

Hunger&        0.461\sym{**} &     0.045\sym{***}&     0.079\sym{**} &      0.326\sym{**} &     -2.070\sym{**} &      0.813\sym{**} \\
            &    (0.064)         &    (0.002)         &    (0.011)         &    (0.049)         &    (0.347)         &    (0.165)         \\

\(N\)       &       33102         &       33102         &       33792         &       33711         &       32237         &       32823         \\

\hline
\\
Dispossession $\times$ Regime change&        1.677\sym{***}&      0.109\sym{***}&     0.078\sym{***}&      0.330\sym{***}&     -3.282\sym{***}&      0.267         \\
            &    (0.042)         &    (0.003)         &    (0.003)         &    (0.012)         &    (0.298)         &    (0.135)         \\

\\
Regime change&    1.533\sym{***}&      0.108\sym{***}&     0.065\sym{***}&      0.244\sym{***}&     -3.146\sym{**} &      0.127         \\
            &    (0.016)         &    (0.003)         &    (0.005)         &    (0.021)         &    (0.332)         &    (0.139)         \\

Dispossession&       0.155         &   -0.002         &     0.064\sym{**} &      0.225\sym{*}  &     -0.578         &      0.381\sym{**} \\
            &    (0.080)         &    (0.006)         &    (0.014)         &    (0.065)         &    (0.475)         &    (0.079)         \\

\(N\)       &       33354         &       33354         &       34051         &       33969         &       32372         &       32955         \\

\hline
\\
Persecution $\times$ Regime change&   1.647\sym{***}&      0.122\sym{***}&     0.095\sym{**} &      0.398\sym{***}&     -3.765\sym{***}&      0.415         \\
            &    (0.029)         &    (0.004)         &    (0.011)         &    (0.017)         &    (0.263)         &    (0.167)         \\

\\
Regime change&       1.556\sym{***}&      0.108\sym{***}&     0.064\sym{***}&      0.241\sym{***}&     -3.138\sym{**} &      0.130         \\
            &    (0.018)         &    (0.003)         &    (0.004)         &    (0.017)         &    (0.328)         &    (0.138)         \\

Persecution&     0.535\sym{**} &     0.029         &      0.102\sym{***}&      0.317\sym{***}&     -1.861\sym{***}&      0.844\sym{***}\\
            &    (0.123)         &    (0.017)         &    (0.001)         &    (0.020)         &    (0.182)         &    (0.054)         \\


\(N\)       &       38113         &       38113         &       38113         &       36006         &       37010         &       37663         \\

\hline
\\

Period of particular stress $\times$ Regime change&      1.745\sym{***}&      0.130\sym{***}&      0.144\sym{***}&      0.510\sym{***}&     -4.431\sym{***}&      0.822\sym{**} \\
            &    (0.070)         &    (0.007)         &    (0.003)         &    (0.007)         &    (0.391)         &    (0.149)         \\

\\
Regime change&        1.447\sym{***}&      0.101\sym{***}&     0.080\sym{***}&      0.255\sym{***}&     -3.096\sym{**} &      0.200         \\
            &    (0.054)         &    (0.005)         &    (0.001)         &    (0.015)         &    (0.416)         &    (0.146)         \\

Period of particular stress &       0.067         &     0.010         &     0.079\sym{**} &      0.218\sym{***}&     -1.013\sym{**} &      0.6452\sym{***}\\
            &    (0.055)         &    (0.005)         &    (0.012)         &    (0.018)         &    (0.136)         &    (0.007)         \\

\(N\)       &       32897         &       32897         &       33562         &       33485         &       32075         &       32651         \\

\hline
Age  & YES & YES& YES& YES& YES& YES \\ 
Gender & YES & YES& YES& YES& YES& YES \\ 
Childhood SES & YES & YES& YES& YES& YES& YES  \\ 
Childhood health & YES & YES& YES& YES& YES& YES  \\ 
\bottomrule
\multicolumn{67}{l}{\footnotesize  \textit{Source}: SHARE (waves 4, 5, 6) and SHARELIFE (waves 3, 7), release 7.0.0.}\\
\multicolumn{6}{l}{\footnotesize   \textit{Notes}: Sample excluding non-communist authoritarian countries (Greece, Portugal, Spain). Robust standard errors clustered at individual level in parentheses. \sym{*} \(p<0.10\), \sym{**} \(p<0.05\), \sym{***} \(p<0.01\).}\\
\label{tab:authorit}
\end{tabular}
}

\\
}




\newpage
\captionof{table}{Robustness of shock effects to the inclusion of time fixed effects}
\resizebox{\textwidth}{!}{
{
\def\sym#1{\ifmmode^{#1}\else\(^{#1}\)\fi}
\begin{tabular}{l*{6}{c}}
\toprule
            &\multicolumn{2}{c}{Short-term health}&\multicolumn{2}{c}{Long-term health}&\multicolumn{2}{c}{Well-being}\\
            &\multicolumn{1}{c}{BMI}&\multicolumn{1}{c}{Obesity (1,0)}&\multicolumn{1}{c}{Chronically ill (1,0)}&\multicolumn{1}{c}{No. of chronic illnesses}&\multicolumn{1}{c}{CASP index}&\multicolumn{1}{c}{Depression scale}\\
\hline 
\\
Hunger $\times$ Regime change&     1.533\sym{***}&      0.127\sym{***}&     0.073\sym{**} &      0.445\sym{***}&     -3.891\sym{***}&      0.713\sym{***} \\
            &    (0.093)         &    (0.004)         &    (0.010)         &    (0.028)         &    (0.216)         &    (0.037)         \\

\\
Regime change&     1.393\sym{***}&      0.101\sym{***}&     0.051\sym{**} &      0.196\sym{**} &     -2.148\sym{**} &     -0.007         \\
            &    (0.047)         &    (0.002)         &    (0.010)         &    (0.037)         &    (0.423)         &    (0.093)         \\


Hunger&      0.662\sym{***}&     0.058\sym{***}&     0.071\sym{**} &      0.303\sym{**} &     -2.621\sym{**} &      0.865\sym{**} \\
            &    (0.045)         &    (0.006)         &    (0.009)         &    (0.042)         &    (0.343)         &    (0.107)         \\

\(N\)       &       37783         &       37783         &       38748         &       38660         &       36818         &       37467         \\

\hline
\\
Dispossession $\times$ Regime change&   1.487\sym{***}&     0.098\sym{***}&     0.059\sym{**}&      0.264\sym{***}&     -2.231\sym{**} &      0.116         \\
            &    (0.080)         &    (0.003)         &    (0.006)         &    (0.007)         &    (0.391)         &    (0.094)         \\


\\
Regime change&       1.355\sym{***}&     0.098\sym{***}&     0.047\sym{**} &      0.182\sym{**} &     -2.087\sym{**} &     -0.027         \\
            &    (0.034)         &    (0.002)         &    (0.009)         &    (0.037)         &    (0.429)         &    (0.091)         \\


Dispossession&   0.147         &    0.003         &     0.054\sym{*}  &      0.172         &     -0.236         &      0.266\sym{**} \\
            &    (0.077)         &    (0.005)         &    (0.015)         &    (0.063)         &    (0.475)         &    (0.061)         \\



\(N\)       &       38081         &       38081         &       39058         &       38969         &       36987         &       37634         \\

\hline
\\
Persecution $\times$ Regime change&     1.467\sym{***}&      0.112\sym{***}&     0.078\sym{**} &      0.342\sym{***}&     -2.800\sym{**} &      0.285         \\
            &    (0.052)         &    (0.003)         &    (0.013)         &    (0.025)         &    (0.352)         &    (0.121)         \\


\\
Regime change&      1.374\sym{***}&     0.098\sym{***}&     0.046\sym{**} &      0.178\sym{**} &     -2.064\sym{**} &     -0.026         \\
            &    (0.036)         &    (0.002)         &    (0.009)         &    (0.033)         &    (0.422)         &    (0.090)         \\

Persecution&      0.457\sym{***}&     0.025         &     0.085\sym{***}&      0.260\sym{**} &     -1.437\sym{**} &      0.761\sym{***}\\
            &    (0.037)         &    (0.010)         &    (0.006)         &    (0.030)         &    (0.168)         &    (0.061)         \\


\(N\)       &       38113         &       38113         &       39090         &       39000         &       37010         &       37663         \\


\hline

\\
Period of particular stress $\times$ Regime change&      1.541\sym{***}&      0.118\sym{***}&      0.116\sym{***}&      0.420\sym{***}&     -3.148\sym{**} &      0.623\sym{**} \\
            &    (0.086)         &    (0.006)         &    (0.004)         &    (0.026)         &    (0.471)         &    (0.091)         \\


\\
Regime change&     1.240\sym{***}&     0.089\sym{***}&     0.051\sym{**} &      0.164\sym{**} &     -1.795\sym{*}  &     -0.001         \\
            &    (0.010)         &    (0.003)         &    (0.007)         &    (0.034)         &    (0.500)         &    (0.093)         \\


Period of particular stress &    0.026         &    0.008         &     0.063\sym{**} &      0.175\sym{***}&     -0.596\sym{**} &      0.590\sym{***}\\
            &    (0.039)         &    (0.004)         &    (0.009)         &    (0.013)         &    (0.108)         &    (0.022)         \\

\(N\)       &       37532         &       37532         &       38461         &       38378         &       36621         &       37258         \\
\hline\\
Age & YES & YES& YES& YES& YES& YES \\ 
Sex & YES & YES& YES& YES& YES& YES \\ 
Childhood SES & YES & YES& YES& YES& YES& YES  \\ 
Childhood health & YES & YES& YES& YES& YES& YES  \\ 
Individual FE & YES & YES& YES& YES& YES& YES  \\ 
Time FE & YES & YES& YES& YES& YES& YES  \\ 

\bottomrule
\multicolumn{6}{l}{\footnotesize  \textit{Source}: SHARE (waves 4, 5, 6) and SHARELIFE (waves 3, 7), release 7.0.0.}\\
\multicolumn{6}{l}{\footnotesize   \textit{Notes}: Robust standard errors clustered at individual in parentheses. \sym{*} \(p<0.10\), \sym{**} \(p<0.05\), \sym{***} \(p<0.01\).}\\
\label{tab:shocks_tFE}
\end{tabular}
}

\\
}\\


\newpage
\captionof{table}{Robustness of shock effects to the inclusion of time trend}
\resizebox{\textwidth}{!}{
{
\def\sym#1{\ifmmode^{#1}\else\(^{#1}\)\fi}
\begin{tabular}{l*{6}{c}}
\toprule
            &\multicolumn{2}{c}{Short-term health}&\multicolumn{2}{c}{Long-term health}&\multicolumn{2}{c}{Well-being}\\
            &\multicolumn{1}{c}{BMI}&\multicolumn{1}{c}{Obesity (1,0)}&\multicolumn{1}{c}{Chronically ill (1,0)}&\multicolumn{1}{c}{No. of chronic illnesses}&\multicolumn{1}{c}{CASP index}&\multicolumn{1}{c}{Depression scale}\\
\hline 
\\
Hunger $\times$ Regime change&     1.546\sym{***}&      0.128\sym{***}&     0.074\sym{**} &      0.448\sym{***}&     -3.896\sym{***}&      0.714\sym{***} \\
            &    (0.075)         &    (0.002)         &    (0.011)         &    (0.033)         &    (0.216)         &    (0.038)         \\

\\
Regime change&     1.398\sym{***}&      0.101\sym{***}&     0.051\sym{**} &      0.197\sym{**} &     -2.145\sym{**} &     -0.006         \\
            &    (0.055)         &    (0.003)         &    (0.011)         &    (0.038)         &    (0.425)         &    (0.094)         \\


Hunger&      0.661\sym{***}&     0.058\sym{***}&     0.071\sym{**} &      0.303\sym{**} &     -2.621\sym{**} &      0.865\sym{**} \\
            &    (0.045)         &    (0.006)         &    (0.009)         &    (0.042)         &    (0.343)         &    (0.107)         \\

\(N\)       &       37783         &       37783         &       38748         &       38660         &       36818         &       37467         \\

\hline
\\
Dispossession $\times$ Regime change&   1.494\sym{***}&     0.100\sym{***}&     0.059\sym{**}&      0.266\sym{***}&     -2.231\sym{**} &      0.117         \\
            &    (0.091)         &    (0.004)         &    (0.007)         &    (0.008)         &    (0.391)         &    (0.096)         \\


\\
Regime change&       1.362\sym{***}&     0.099\sym{***}&     0.047\sym{**} &      0.183\sym{**} &     -2.087\sym{**} &     -0.026         \\
            &    (0.044)         &    (0.003)         &    (0.010)         &    (0.039)         &    (0.430)         &    (0.092)         \\


Dispossession&   0.147         &    0.003         &     0.054\sym{*}  &      0.172         &     -0.236         &      0.266\sym{**} \\
            &    (0.077)         &    (0.005)         &    (0.015)         &    (0.063)         &    (0.475)         &    (0.061)         \\



\(N\)       &       38081         &       38081         &       39058         &       38969         &       36987         &       37634         \\

\hline
\\
Persecution $\times$ Regime change&     1.471\sym{***}&      0.113\sym{***}&     0.078\sym{**} &      0.343\sym{***}&     -2.800\sym{**} &      0.286         \\
            &    (0.058)         &    (0.003)         &    (0.013)         &    (0.030)         &    (0.168)         &    (0.061)         \\


\\
Regime change&      1.382\sym{***}&     0.099\sym{***}&     0.047\sym{**} &      0.180\sym{**} &     -2.064\sym{**} &     -0.025         \\
            &    (0.047)         &    (0.003)         &    (0.009)         &    (0.036)         &    (0.423)         &    (0.091)         \\

Persecution&      0.457\sym{***}&     0.025         &     0.085\sym{***}&      0.260\sym{**} &     -1.437\sym{**} &      0.761\sym{***}\\
            &    (0.037)         &    (0.010)         &    (0.006)         &    (0.030)         &    (0.168)         &    (0.061)         \\


\(N\)       &       38113         &       38113         &       39090         &       39000         &       37010         &       37663         \\


\hline

\\
Period of particular stress $\times$ Regime change&      1.547\sym{***}&      0.118\sym{***}&      0.116\sym{***}&      0.421\sym{***}&     -3.150\sym{**} &      0.623\sym{**} \\
            &    (0.094)         &    (0.006)         &    (0.004)         &    (0.028)         &    (0.473)         &    (0.092)         \\


\\
Regime change&     1.246\sym{***}&     0.089\sym{***}&     0.051\sym{**} &      0.165\sym{**} &     -1.800\sym{*}  &     -0.001         \\
            &    (0.006)         &    (0.008)         &    (0.006)         &    (0.035)         &    (0.502)         &    (0.094)         \\


Period of particular stress &    0.026         &    0.008         &     0.063\sym{**} &      0.175\sym{***}&     -0.596\sym{**} &      0.590\sym{***}\\
            &    (0.039)         &    (0.004)         &    (0.009)         &    (0.013)         &    (0.108)         &    (0.022)         \\

\(N\)       &       37532         &       37532         &       38461         &       38378         &       36621         &       37258         \\
\hline\\
Age  & YES & YES& YES& YES& YES& YES \\ 
Sex & YES & YES& YES& YES& YES& YES \\ 
Childhood SES & YES & YES& YES& YES& YES& YES  \\ 
Childhood health & YES & YES& YES& YES& YES& YES  \\ 
Individual FE & YES & YES& YES& YES& YES& YES  \\ 
Time trend & YES & YES& YES& YES& YES& YES  \\ 

\bottomrule
\multicolumn{6}{l}{\footnotesize  \textit{Source}: SHARE (waves 4, 5, 6) and SHARELIFE (waves 3, 7), release 7.0.0.}\\
\multicolumn{6}{l}{\footnotesize   \textit{Notes}: Robust standard errors clustered at individual in parentheses. \sym{*} \(p<0.10\), \sym{**} \(p<0.05\), \sym{***} \(p<0.01\).}\\
\label{tab:shocks_ttrend}
\end{tabular}
}

\\
}\\


\newpage
\captionof{table}{Shock effects on other measures of later-life health}
\resizebox{\textwidth}{!}{
{
\def\sym#1{\ifmmode^{#1}\else\(^{#1}\)\fi}
\begin{tabular}{l*{5}{c}}
\toprule
            &\multicolumn{1}{c}{Mobility limitations}&\multicolumn{1}{c}{GALI (1,0)}&\multicolumn{1}{c}{SPHUS scale}&\multicolumn{1}{c}{Diabetes}&\multicolumn{1}{c}{Heart attack}\\
\hline
Hunger $\times$ Regime change &       1.058\sym{***}&      0.196\sym{***}&      0.629\sym{***}&     0.075\sym{**} &      0.107\sym{**} \\
            &    (0.028)         &    (0.000)         &    (0.013)         &    (0.011)         &    (0.017)         \\

            \\
Regime change&        0.301\sym{*}  &     0.098\sym{***}&      0.426\sym{***}&     0.044\sym{***}&     0.035\sym{**} \\
            &    (0.100)         &    (0.009)         &    (0.036)         &    (0.001)         &    (0.006)         \\
Hunger&        0.875\sym{***}&      0.125\sym{***}&      0.367\sym{**} &     0.051\sym{**} &     0.043\sym{***}\\
            &    (0.030)         &    (0.007)         &    (0.043)         &    (0.010)         &    (0.001)         \\

\(N\)       &       38676         &       38683         &       38682         &       38660         &       38660         \\

\hline

Dispossession $\times$ Regime change &      0.302\sym{**} &      0.106\sym{***}&      0.458\sym{***}&     0.048\sym{**} &     0.043\sym{**} \\
            &    (0.061)         &    (0.006)         &    (0.020)         &    (0.009)         &    (0.008)         \\
 \\
 Regime change &    0.294\sym{*}  &     0.095\sym{**} &      0.411\sym{***}&     0.042\sym{***}&     0.035\sym{*}  \\
            &    (0.098)         &    (0.010)         &    (0.037)         &    (0.002)         &    (0.008)         \\

Dispossession &   0.095         &     0.049         &   0.001         &    0.009         &     0.037         \\
            &    (0.057)         &    (0.017)         &    (0.067)         &    (0.012)         &    (0.018)         \\

\(N\)       &       38984         &       38991         &       38990         &       38969         &       38969         \\

\hline
Persecution $\times$ Regime change &    0.316\sym{*}  &      0.142\sym{***}&      0.453\sym{***}&     0.054\sym{***}&     0.050\sym{*}  \\
            &    (0.090)         &    (0.012)         &    (0.044)         &    (0.003)         &    (0.017)         \\

\\
Regime change &    0.305\sym{*}  &     0.094\sym{***}&      0.418\sym{***}&     0.041\sym{***}&     0.034\sym{**} \\
            &    (0.098)         &    (0.008)         &    (0.034)         &    (0.001)         &    (0.007)         \\


Persecution&       0.468\sym{**} &      0.121\sym{***}&      0.168\sym{**} &   -0.001         &     0.028\sym{**} \\
            &    (0.103)         &    (0.008)         &    (0.026)         &    (0.006)         &    (0.004)         \\
            \(N\)       &       39016         &       39023         &       39022         &       39000         &       39000         \\

\hline
Period of particular stress $\times$ Regime change &  0.690\sym{**} &      0.176\sym{***}&      0.551\sym{***}&     0.051\sym{***}&     0.068\sym{**} \\
            &    (0.087)         &    (0.006)         &    (0.040)         &    (0.001)         &    (0.008)         \\


\\
Regime change&    0.252\sym{*}  &      0.104\sym{***}&      0.423\sym{***}&     0.032\sym{***}&     0.030\sym{**} \\
            &    (0.084)         &    (0.009)         &    (0.038)         &    (0.002)         &    (0.006)         \\


Period of particular stress &        0.303\sym{**} &     0.080\sym{***}&      0.120\sym{***}&   -0.006         &     0.021\sym{**} \\
            &    (0.045)         &    (0.006)         &    (0.001)         &    (0.003)         &    (0.003)         \\

\(N\)       &       38395         &       38401         &       38401         &       38378         &       38378         \\

\bottomrule
\multicolumn{6}{l}{\footnotesize  \textit{Source}: SHARE (waves 4, 5, 6) and SHARELIFE (waves 3, 7), release 7.0.0.}\\
\multicolumn{6}{l}{\footnotesize \textit{Notes}: Robust standard errors clustered at individual level in parentheses. \sym{*} \(p<0.10\), \sym{**} \(p<0.05\), \sym{***} \(p<0.01\).}\\
\end{tabular}
\label{tab:mech1}
}
\\
}\\

\newpage
\captionof{table}{Shock effects in formerly communist countries by old versus young cohorts}
\resizebox{\textwidth}{!}{
{
\def\sym#1{\ifmmode^{#1}\else\(^{#1}\)\fi}
\begin{tabular}{l*{6}{c}}
\toprule
            &\multicolumn{2}{c}{Short-term health}&\multicolumn{2}{c}{Long-term health}&\multicolumn{2}{c}{Well-being}\\
            &\multicolumn{1}{c}{BMI}&\multicolumn{1}{c}{Obesity (1,0)}&\multicolumn{1}{c}{Chronically ill (1,0)}&\multicolumn{1}{c}{No. of chronic illnesses}&\multicolumn{1}{c}{CASP index}&\multicolumn{1}{c}{Depression scale}\\
\hline 
\\
Hunger $\times$ Old cohorts&   0.751\sym{***}&     0.094\sym{**} &     0.023         &      0.232\sym{**} &     -0.731         &      0.428\sym{*}  \\
            &    (0.075)         &    (0.019)         &    (0.018)         &    (0.047)         &    (0.359)         &    (0.108)         \\
\\
\(N\)       &       12563         &       12563         &       12805         &       12779         &       12251         &       12433         \\

\hline
\\
Dispossession $\times$ Old cohort&      0.503\sym{*}  &     0.052\sym{*}  &     0.021         &      0.126         &      0.126\sym{*}  &   -0.009         \\
            &    (0.159)         &    (0.012)         &    (0.014)         &    (0.053)         &    (0.031)         &    (0.044)         \\
\\
\(N\)       &       12720         &       12720         &       12966         &       12939         &       12350         &       12532         \\

\hline
\\
Persecution $\times$ Old cohort&        0.460\sym{*}  &     0.071\sym{**} &     0.030\sym{***}&      0.154\sym{**} &      0.202         &    -0.085\sym{***}\\
            &    (0.143)         &    (0.007)         &    (0.003)         &    (0.017)         &    (0.145)         &    (0.004)         \\

\\

\(N\)       &       12730         &       12730         &       12974         &       12947         &       12355         &       12539         \\


\hline
\\
Period of particular stress $\times$ Old cohort&   0.656\sym{**} &     0.088\sym{**} &     0.071\sym{**} &      0.288\sym{**} &     -1.229\sym{***}&      0.552\sym{***}\\
            &    (0.092)         &    (0.010)         &    (0.015)         &    (0.041)         &    (0.039)         &    (0.044)         \\
\\

\(N\)       &       12512         &       12512         &       12745         &       12721         &       12208         &       12389         \\
\hline
Age  & YES & YES& YES& YES& YES& YES \\ 
Sex & YES & YES& YES& YES& YES& YES \\ 
Childhood SES & YES & YES& YES& YES& YES& YES  \\ 
Childhood health & YES & YES& YES& YES& YES& YES  \\ 
Individual FE & YES & YES& YES& YES& YES& YES  \\ 
\bottomrule
\multicolumn{6}{l}{\footnotesize  \textit{Source}: SHARE (waves 4, 5, 6) and SHARELIFE (waves 3, 7), release 7.0.0.}\\
\multicolumn{6}{l}{\footnotesize   \textit{Notes}: Robust standard errors clustered at individual level in parentheses. \sym{*} \(p<0.10\), \sym{**} \(p<0.05\), \sym{***} \(p<0.01\).}\\
\label{tab:coh}
\end{tabular}
}

\\
}\\

\scriptsize{Old cohort comprises of individuals born before 1950. The research sample is limited to transition countries only.}




\newpage
\captionof{table}{Shock and regime change effects excluding former USSR states}
\resizebox{\textwidth}{!}{
{
\def\sym#1{\ifmmode^{#1}\else\(^{#1}\)\fi}
\begin{tabular}{l*{6}{c}}
\toprule
            &\multicolumn{2}{c}{Short-term health}&\multicolumn{2}{c}{Long-term health}&\multicolumn{2}{c}{Well-being}\\
            &\multicolumn{1}{c}{BMI}&\multicolumn{1}{c}{Obesity (1,0)}&\multicolumn{1}{c}{Chronically ill (1,0)}&\multicolumn{1}{c}{No. of chronic illnesses}&\multicolumn{1}{c}{CASP index}&\multicolumn{1}{c}{Depression scale}\\
\hline 
\\
Hunger $\times$ Regime change&         1.357\sym{**} &      0.121\sym{***}&     0.069\sym{**} &      0.412\sym{***}&     -3.835\sym{***}&      0.803\sym{**} \\
            &    (0.185)         &    (0.010)         &    (0.010)         &    (0.019)         &    (0.188)         &    (0.116)         \\


\\
Regime change&       1.385\sym{***}&     0.100\sym{***}&     0.055\sym{**} &      0.209\sym{**} &     -2.093\sym{**} &     0.013         \\
            &    (0.047)         &    (0.004)         &    (0.009)         &    (0.032)         &    (0.449)         &    (0.146)         \\



Hunger&     0.648\sym{***}&     0.057\sym{***}&     0.070\sym{**} &      0.303\sym{**} &     -2.674\sym{**} &      0.867\sym{**} \\
            &    (0.048)         &    (0.005)         &    (0.010)         &    (0.044)         &    (0.351)         &    (0.110)         \\



\(N\)       &       34584         &       34584         &       35513         &       35433         &       33768         &       34368         \\


\hline
\\
Dispossession $\times$ Regime change&      1.617\sym{**} &     0.096\sym{**} &     0.078\sym{***}&      0.341\sym{***}&     -1.998\sym{*}  &      0.116         \\
            &    (0.166)         &    (0.014)         &    (0.007)         &    (0.030)         &    (0.550)         &    (0.186)         \\

\\
Regime change&     1.328\sym{***}&     0.098\sym{***}&     0.049\sym{**} &      0.187\sym{**} &     -2.061\sym{**} &   0.001         \\
            &    (0.019)         &    (0.003)         &    (0.010)         &    (0.035)         &    (0.443)         &    (0.142)         \\




Dispossession&   0.139         &    0.003         &     0.053\sym{*}  &      0.171         &     -0.257         &      0.272\sym{**} \\
            &    (0.080)         &    (0.005)         &    (0.015)         &    (0.063)         &    (0.466)         &    (0.060)         \\




\(N\)       &       34808         &       34808         &       35748         &       35667         &       33899         &       34497         \\


\hline
\\
Persecution $\times$ Regime change&      1.462\sym{***}&      0.110\sym{***}&     0.084\sym{**} &      0.343\sym{***}&     -2.759\sym{**} &      0.229         \\
            &    (0.112)         &    (0.003)         &    (0.017)         &    (0.033)         &    (0.415)         &    (0.205)         \\




\\
Regime change&      1.361\sym{***}&     0.097\sym{***}&     0.049\sym{**} &      0.192\sym{**} &     -2.013\sym{**} &    0.007         \\
            &    (0.023)         &    (0.004)         &    (0.008)         &    (0.030)         &    (0.452)         &    (0.141)         \\


Persecution&          0.455\sym{***}&     0.025         &     0.084\sym{***}&      0.259\sym{**} &     -1.442\sym{**} &      0.765\sym{***}\\
            &    (0.033)         &    (0.010)         &    (0.007)         &    (0.031)         &    (0.164)         &    (0.060)         \\




\(N\)       &       34835         &       34835         &       35777         &       35695         &       33920         &       34523         \\


\hline
%\\
%Period of particular happiness $\times$ Regime change&    1.4925\sym{***}&      0.1106\sym{***}&     0.08567\sym{**} &      0.3178\sym{**} &     -3.2040\sym{**} &      0.4924\sym{*}  \\
%            &    (0.0275)         &    (0.0035)         &    (0.0099)         &    (0.0426)         &    (0.4375)         &    (0.1326)         \\
%\\

%Regime change&        1.1668\sym{***}&     0.08861\sym{***}&     0.04464\sym{**} &      0.1528\sym{**} &     -2.2706\sym{**} &     0.08606         \\
%            &    (0.0198)         &    (0.0016)         &    (0.0081)         &    (0.0280)         &    (0.5170)         &    (0.1835)         \\



%Period of particular happiness &     -0.02392         &    0.004639         &     0.04260\sym{***}&      0.1205\sym{**} &     -1.3797\sym{**} &      0.5904\sym{***}\\
%            &    (0.0321)         &    (0.0033)         &    (0.0017)         &    (0.0127)         &    (0.1400)         &    (0.0406)         \\


%\(N\)       &       37393         &       37393         &       38315         &       38233         &       36502         &       37129         \\

%\hline
\\
Period of particular stress $\times$ Regime change&           1.4971\sym{***}&      0.1136\sym{***}&      0.1172\sym{***}&      0.4239\sym{***}&     -3.0742\sym{**} &      0.6255\sym{*}  \\
            &    (0.1028)         &    (0.0095)         &    (0.0035)         &    (0.0217)         &    (0.5211)         &    (0.1637)         \\
\\
Regime change&  

 1.2674\sym{***}&     0.09175\sym{***}&     0.05593\sym{**} &      0.1836\sym{**} &     -1.7421\sym{*}  &     0.03800         \\
            &    (0.0344)         &    (0.0040)         &    (0.0073)         &    (0.0310)         &    (0.5138)         &    (0.1383)         \\



Period of particular stress &   0.06316\sym{**} &      0.1748\sym{***}&     -0.5749\sym{**} &      0.5873\sym{***}\\
            &    (0.0427)         &    (0.0039)         &    (0.0093)         &    (0.0131)         &    (0.1140)         &    (0.0212)         \\


\(N\)       &       34354         &       34354         &       35250         &       35174         &       33588         &       34177         \\



\hline\\
Age  & YES & YES& YES& YES& YES& YES \\ 
Sex & YES & YES& YES& YES& YES& YES \\ 
Childhood SES & YES & YES& YES& YES& YES& YES  \\ 
Childhood health & YES & YES& YES& YES& YES& YES  \\ 
Individual FE & YES & YES& YES& YES& YES& YES  \\ 
\bottomrule
\multicolumn{6}{l}{\footnotesize  \textit{Source}: SHARE (waves 4, 5, 6) and SHARELIFE (waves 3, 7), release 7.0.0.}\\
\multicolumn{6}{l}{\footnotesize   \textit{Notes}: Robust standard errors clustered at individual level in parentheses. \sym{*} \(p<0.10\), \sym{**} \(p<0.05\), \sym{***} \(p<0.01\).}\\
\label{tab:estoniaout}
\end{tabular}
}

\\
}\\



\newpage
\captionof{table}{Shock and regime change effects including migrants}
\resizebox{\textwidth}{!}{
{
\def\sym#1{\ifmmode^{#1}\else\(^{#1}\)\fi}
\begin{tabular}{l*{6}{c}}
\toprule
            &\multicolumn{2}{c}{Short-term health}&\multicolumn{2}{c}{Long-term health}&\multicolumn{2}{c}{Well-being}\\
            &\multicolumn{1}{c}{BMI}&\multicolumn{1}{c}{Obesity (1,0)}&\multicolumn{1}{c}{Chronically ill (1,0)}&\multicolumn{1}{c}{No. of chronic illnesses}&\multicolumn{1}{c}{CASP index}&\multicolumn{1}{c}{Depression scale}\\
\hline 
\\
Hunger $\times$ Regime change&        1.663\sym{***}&      0.130\sym{***}&     0.078\sym{***}&      0.443\sym{***}&     -4.290\sym{***}&      0.812\sym{***}\\
            &    (0.055)         &    (0.003)         &    (0.003)         &    (0.012)         &    (0.218)         &    (0.071)         \\

\\
Regime change&       1.362\sym{***}&     0.099\sym{***}&     0.048\sym{**} &      0.180\sym{**} &     -2.147\sym{**} &     0.020         \\
            &    (0.031)         &    (0.004)         &    (0.007)         &    (0.022)         &    (0.385)         &    (0.128)         \\


Hunger&       0.626\sym{***}&     0.054\sym{**} &     0.073\sym{***}&      0.328\sym{***}&     -2.290\sym{***}&      0.834\sym{***}\\
            &    (0.035)         &    (0.007)         &    (0.004)         &    (0.021)         &    (0.149)         &    (0.053)         \\


\(N\)       &       42042         &       42042         &       43100         &       43000         &       40901         &       41644         \\

\hline
\\
Dispossession $\times$ Regime change&     0.039         &    -0.012         &     0.052\sym{***}&      0.175\sym{***}&    -0.089         &      0.190\sym{**} \\
            &    (0.079)         &    (0.008)         &    (0.005)         &    (0.003)         &    (0.299)         &    (0.044)         \\


\\
Regime change&       1.314\sym{***}&     0.096\sym{***}&     0.043\sym{**} &      0.165\sym{**} &     -2.096\sym{**} &    0.007         \\
            &    (0.020)         &    (0.004)         &    (0.006)         &    (0.023)         &    (0.401)         &    (0.129)         \\



Dispossession& 0.03880         &    -0.012         &     0.052\sym{***}&      0.175\sym{***}&    -0.089         &      0.190\sym{**} \\
            &    (0.079)         &    (0.008)         &    (0.005)         &    (0.003)         &    (0.299)         &    (0.044)         \\



\(N\)       &       42390         &       42390         &       43463         &       43359         &       41087         &       41832         \\

\hline
\\
Persecution $\times$ Regime change&      1.361\sym{***}&     0.099\sym{***}&     0.081\sym{**} &      0.319\sym{***}&     -2.812\sym{**} &      0.319         \\
            &    (0.050)         &    (0.001)         &    (0.009)         &    (0.015)         &    (0.359)         &    (0.179)         \\



\\
Regime change&       1.356\sym{***}&     0.098\sym{***}&     0.043\sym{**} &      0.166\sym{**} &     -2.110\sym{**} &     0.014         \\
            &    (0.023)         &    (0.004)         &    (0.005)         &    (0.020)         &    (0.382)         &    (0.123)         \\

Persecution&         0.395\sym{*}  &     0.015         &     0.075\sym{***}&      0.252\sym{***}&     -1.349\sym{***}&      0.681\sym{***}\\
            &    (0.107)         &    (0.013)         &    (0.001)         &    (0.016)         &    (0.106)         &    (0.033)         \\



\(N\)       &       42433         &       42433         &       43507         &       43402         &       41124         &       41874         \\


\hline
%\\
%Period of particular happiness $\times$ Regime change&    1.4925\sym{***}&      0.1106\sym{***}&     0.08567\sym{**} &      0.3178\sym{**} &     -3.2040\sym{**} &      0.4924\sym{*}  \\
%            &    (0.0275)         &    (0.0035)         &    (0.0099)         &    (0.0426)         &    (0.4375)         &    (0.1326)         \\
%\\

%Regime change&        1.1668\sym{***}&     0.08861\sym{***}&     0.04464\sym{**} &      0.1528\sym{**} &     -2.2706\sym{**} &     0.08606         \\
%            &    (0.0198)         &    (0.0016)         &    (0.0081)         &    (0.0280)         &    (0.5170)         &    (0.1835)         \\



%Period of particular happiness &     -0.02392         &    0.004639         &     0.04260\sym{***}&      0.1205\sym{**} &     -1.3797\sym{**} &      0.5904\sym{***}\\
%            &    (0.0321)         &    (0.0033)         &    (0.0017)         &    (0.0127)         &    (0.1400)         &    (0.0406)         \\


%\(N\)       &       37393         &       37393         &       38315         &       38233         &       36502         &       37129         \\

%\hline
\\
Period of particular stress $\times$ Regime change&        1.540\sym{***}&      0.116\sym{***}&      0.113\sym{***}&      0.410\sym{***}&     -3.158\sym{**} &      0.664\sym{**} \\
            &    (0.074)         &    (0.007)         &    (0.000)         &    (0.011)         &    (0.423)         &    (0.125)         \\




\\
Regime change&     1.540\sym{***}&      0.116\sym{***}&      0.113\sym{***}&      0.410\sym{***}&     -3.158\sym{**} &      0.664\sym{**} \\
            &    (0.074)         &    (0.007)         &    (0.000)         &    (0.011)         &    (0.423)         &    (0.125)         \\



Period of particular stress &    0.034         &    0.006         &     0.062\sym{**} &      0.177\sym{**} &     -0.579\sym{**} &      0.598\sym{***}\\
            &    (0.051)         &    (0.004)         &    (0.010)         &    (0.018)         &    (0.096)         &    (0.025)         \\

\(N\)       &       41747         &       41747         &       42766         &       42670         &       40670         &       41400         \\


\hline\\
Age & YES & YES& YES& YES& YES& YES \\ 
Sex & YES & YES& YES& YES& YES& YES \\ 
Childhood SES & YES & YES& YES& YES& YES& YES  \\ 
Childhood health & YES & YES& YES& YES& YES& YES  \\ 
Individual FE & YES & YES& YES& YES& YES& YES  \\ 
\bottomrule
\multicolumn{6}{l}{\footnotesize  \textit{Source}: SHARE (waves 4, 5, 6) and SHARELIFE (waves 3, 7), release 7.0.0.}\\
\multicolumn{6}{l}{\footnotesize   \textit{Notes}: Robust standard errors clustered at individual level in parentheses. \sym{*} \(p<0.10\), \sym{**} \(p<0.05\), \sym{***} \(p<0.01\).}\\
\label{tab:migrants}
\end{tabular}
}

\\
}\\



\newpage

\captionof{table}{Shock effects on socio-economic factors}
\resizebox{0.9\textwidth}{!}{
{
\def\sym#1{\ifmmode^{#1}\else\(^{#1}\)\fi}
\begin{tabular}{l*{4}{c}}
\toprule
            &\multicolumn{1}{c}{Net wealth in logs}&\multicolumn{1}{c}{Retirement age}&\multicolumn{1}{c}{Education}&\multicolumn{1}{c}{Married}\\
\hline
Hunger $\times$ Regime change&        -1.732\sym{***}&    -0.060         &   -0.010         &     -0.124\sym{***}\\
            &    (0.045)         &    (0.196)         &    (0.049)         &    (0.008)         \\

\\

Regime change &      -1.489\sym{***}&    0.008         &      0.190\sym{**} &    -0.061\sym{***}\\
            &    (0.051)         &    (0.152)         &    (0.035)         &    (0.005)         \\

Hunger&         -0.516\sym{**} &     -0.616\sym{**} &     -0.417\sym{**} &    -0.064\sym{**} \\
            &    (0.062)         &    (0.080)         &    (0.062)         &    (0.007)         \\


\(N\)       &       37369         &       38747         &       38184         &       38748         \\


\hline
Dispossession $\times$ Regime change &    -1.389\sym{***}&      0.249         &      0.455\sym{**} &    -0.082\sym{**} \\
            &    (0.053)         &    (0.152)         &    (0.083)         &    (0.010)         \\

            \\
Regime change&     -1.484\sym{***}&     0.0134         &      0.178\sym{**} &    -0.060\sym{***}\\
            &    (0.048)         &    (0.159)         &    (0.030)         &    (0.006)         \\


Dispossession &        0.401\sym{*}  &      0.387         &      0.430\sym{***}&    -0.059\sym{**} \\
            &    (0.114)         &    (0.228)         &    (0.007)         &    (0.010)         \\


\(N\)       &       37667         &       39057         &       38492         &       39058         \\

\hline

Persecution $\times$ Regime change  &  -1.409\sym{***}&      0.499         &      0.494\sym{**} &    -0.089\sym{**} \\
            &    (0.124)         &    (0.251)         &    (0.050)         &    (0.011)         \\


\\
Regime change&       -1.490\sym{***}&    -0.020         &      0.184\sym{**} &    -0.061\sym{***}\\
            &    (0.042)         &    (0.138)         &    (0.040)         &    (0.005)         \\



Persecution &      -0.035         &    0.006         &      0.434\sym{*}  &    -0.098\sym{**} \\
            &    (0.073)         &    (0.077)         &    (0.104)         &    (0.010)         \\



\(N\)       &       37699         &       39089         &       38519         &       39090         \\


\hline

Period of particular stress $\times$ Regime change &        -1.460\sym{***}&     -0.113         &      0.423\sym{**} &     -0.136\sym{***}\\
            &    (0.040)         &    (0.099)         &    (0.063)         &    (0.003)         \\


\\
Regime change&       -1.423\sym{***}&    -0.033         &      0.340\sym{***}&    -0.060\sym{***}\\
            &    (0.050)         &    (0.073)         &    (0.030)         &    (0.005)         \\

Period of particular stress&    0.076\sym{**} &     -0.201         &      0.350\sym{***}&    -0.065\sym{***}\\
            &    (0.012)         &    (0.139)         &    (0.016)         &    (0.006)         \\

\(N\)       &       37093         &       38460         &       37910         &       38461         \\

\bottomrule
\multicolumn{5}{l}{\footnotesize  \textit{Source}: SHARE (waves 4, 5, 6) and SHARELIFE (waves 3, 7), release 7.0.0.}\\
\multicolumn{5}{l}{\footnotesize \textit{Notes}: Robust standard errors clustered at individual level in parentheses.  \sym{*} \(p<0.10\), \sym{**} \(p<0.05\), \sym{***} \(p<0.01\).}\\
\label{tab:mech2}
\end{tabular}
}
\\
}\\
\scriptsize{Transition denotes period between 1983 and 1995.}





\newpage
\begin{landscape}    
\captionof{table}{Shock effects on health-related behaviours in later life}
\resizebox{1.05\textwidth}{!}{
{
\def\sym#1{\ifmmode^{#1}\else\(^{#1}\)\fi}
\begin{tabular}{l*{9}{c}}
\toprule
            &\multicolumn{1}{c}{Currently smoking}&\multicolumn{1}{c}{Total years smoking}&\multicolumn{1}{c}{Vigorous activity (scale)}&\multicolumn{1}{c}{Mediocre activity (scale)}&\multicolumn{1}{c}{Any physical activity}&\multicolumn{1}{c}{Dairy }&\multicolumn{1}{c}{Eggs}&\multicolumn{1}{c}{Fish or meat}&\multicolumn{1}{c}{Fruits or vegetables}\\
\hline
Hunger $\times$ Regime change &     0.0206         &      0.609         &      0.249\sym{***}&      0.126         &    -0.035\sym{*}  &      0.573\sym{**} &      0.534\sym{*}  &      0.455\sym{**} &      0.473\sym{*}  \\
            &    (0.017)         &    (0.356)         &    (0.024)         &    (0.051)         &    (0.011)         &    (0.086)         &    (0.161)         &    (0.100)         &    (0.137)         \\
\\
Hunger&     0.052         &     -0.389         &      0.188\sym{***}&      0.159\sym{*}  &    -0.056\sym{**} &   -0.004         &    -0.059         &      0.104         &     0.069         \\
            &    (0.026)         &    (0.371)         &    (0.014)         &    (0.038)         &    (0.006)         &    (0.008)         &    (0.060)         &    (0.063)         &    (0.090)         \\

Regime change&     0.050\sym{**} &     -0.498         &      0.127\sym{*}  &   -0.002         &   -0.001         &      0.504\sym{**} &      0.376         &      0.255         &      0.320\sym{*}  \\
            &    (0.009)         &    (0.240)         &    (0.041)         &    (0.063)         &    (0.014)         &    (0.110)         &    (0.219)         &    (0.165)         &    (0.077)         \\

\(N\)       &        9872         &       38748         &       38625         &       38628         &       38633         &       14659         &       14653         &       14659         &       14663         \\
\hline

Regime change $\times$ Dispossession&    -0.011         &      0.356         &     0.083         &     -0.101         &     0.030         &      0.417\sym{**} &      0.420         &      0.341         &      0.259\sym{*}  \\
            &    (0.007)         &    (0.440)         &    (0.050)         &    (0.055)         &    (0.011)         &    (0.095)         &    (0.242)         &    (0.157)         &    (0.077)         \\
[1em]
Dispossession &   -0.003         &    -0.077         &    0.007         &    -0.049         &    0.008         &   -0.004         &     0.015         &     0.016         &    -0.020         \\
            &    (0.032)         &    (0.272)         &    (0.100)         &    (0.043)         &    (0.012)         &    (0.079)         &    (0.043)         &    (0.115)         &    (0.023)         \\

Regime change&     0.052\sym{**} &     -0.534         &      0.130\sym{*}  &    0.009         &   -0.005         &      0.513\sym{**} &      0.381         &      0.252         &      0.326\sym{*}  \\
            &    (0.009)         &    (0.274)         &    (0.041)         &    (0.063)         &    (0.014)         &    (0.116)         &    (0.213)         &    (0.162)         &    (0.085)         \\

\(N\)       &        9931         &       39058         &       38934         &       38936         &       38941         &       14764         &       14758         &       14764         &       14768         \\
\hline

Regime change $\times$ Persecution&     0.053         &     -0.272\sym{*}  &      0.132         &    -0.081         &     0.029         &      0.365\sym{**} &      0.464         &      0.365         &      0.327\sym{**} \\
            &    (0.038)         &    (0.086)         &    (0.063)         &    (0.066)         &    (0.013)         &    (0.067)         &    (0.206)         &    (0.142)         &    (0.074)         \\
\\
Persecution &     0.022         &      0.138         &     0.072        &   -0.002         &   -0.005         &     -0.106         &    -0.017         &     -0.113         &    -0.034         \\
            &    (0.074)         &    (0.535)         &    (0.040)         &    (0.053)         &    (0.009)         &    (0.063)         &    (0.028)         &    (0.121)         &    (0.038)         \\

Regime change&     0.045\sym{*}  &     -0.433         &      0.126\sym{*}  &    0.005         &   -0.004         &      0.514\sym{*}  &      0.375         &      0.244         &      0.318\sym{*}  \\
            &    (0.013)         &    (0.303)         &    (0.0385)         &    (0.063)         &    (0.014)         &    (0.121)         &    (0.216)         &    (0.166)         &    (0.084)         \\
\\
\(N\)       &        9934         &       39090         &       38965         &       38968         &       38973         &       14767         &       14761         &       14767         &       14771         \\
\hline

Regime change $\times$ Period of particular stress&     0.043         &     -0.787         &      0.144         &    -0.024         &  -0.001         &      0.478\sym{**} &      0.449         &      0.261         &      0.312\sym{*}  \\
            &    (0.025)         &    (0.295)         &    (0.050)         &    (0.066)         &    (0.014)         &    (0.106)         &    (0.225)         &    (0.182)         &    (0.096)         \\
\\
Period of particular stress&   -0.001         &     -0.163         &    -0.019\sym{**} &    -0.052\sym{**} &    0.006         &    -0.031         &     0.060\sym{*}  &    -0.039         &    -0.021         \\
            &    (0.014)         &    (0.272)         &    (0.004)         &    (0.010)         &    (0.003)         &    (0.050)         &    (0.016)         &    (0.030)         &    (0.011)         \\

Regime change&     0.048\sym{*}  &     -0.301         &     0.088         &    -0.040         &    0.007         &      0.492\sym{*}  &      0.393         &      0.227         &      0.306\sym{**} \\
            &    (0.012)         &    (0.420)         &    (0.037)         &    (0.064)         &    (0.012)         &    (0.143)         &    (0.214)         &    (0.162)         &    (0.069)         \\
\(N\)       &        9768         &       38461         &       38348         &       38351         &       38356         &       14534         &       14529         &       14535         &       14536         \\
\hline

%Regime change $\times$ Period of particular happiness&     0.06846\sym{**} &     -1.0409         &      0.1676\sym{*}  &     0.01612         &   -0.001558         &      0.5045\sym{**} &      0.4245         &      0.2991         &      0.3484\sym{*}  \\
 %           &    (0.0157)         &    (0.5734)         &    (0.0541)         &    (0.0546)         &    (0.0092)         &    (0.1136)         &    (0.2134)         &    (0.1514)         &    (0.0906)         \\
%\\
%Period of particular happiness&    0.005584         &     -0.6222\sym{*}  &     0.09793\sym{**} &     0.04907         &   -0.009529         &    0.009732         &     0.01420         &     0.06883\sym{**} &     0.01266         \\
 %           &    (0.0054)         &    (0.1584)         &    (0.0215)         &    (0.0212)         &    (0.0061)         &    (0.0095)         &    (0.0297)         &    (0.0147)         &    (0.0100)         \\

%Regime change&     0.03418\sym{*}  &     -0.3537         &      0.1713\sym{*}  &     0.01628         &   -0.005894         &      0.5136\sym{**} &      0.3642         &      0.2863         &      0.3044\sym{*}  \\
 %           &    (0.0081)         &    (0.5854)         &    (0.0426)         &    (0.0775)         &    (0.0188)         &    (0.1020)         &    (0.2257)         &    (0.1633)         &    (0.0767)         \\
%\(N\)       &        9725         &       38315         &       38203         &       38207         &       38211         &       14470         &       14465         &       14471         &       14472         \\
%\hline
Age  & YES & YES& YES& YES& YES& YES& YES& YES& YES \\ 
Sex & YES & YES& YES& YES& YES& YES & YES& YES& YES\\ 
Childhood SES & YES & YES& YES& YES& YES& YES & YES& YES& YES \\ 
Childhood health & YES & YES& YES& YES& YES& YES & YES& YES& YES \\ 
Individual FE & YES & YES& YES& YES& YES& YES & YES& YES& YES \\ 
\bottomrule
\multicolumn{6}{l}{\footnotesize  \textit{Source}: SHARE (waves 4, 5, 6) and SHARELIFE (waves 3, 7), release 7.0.0.}\\
\multicolumn{6}{l}{\footnotesize   \textit{Notes}: Robust standard errors clustered at individual level in parentheses. \sym{*} \(p<0.10\), \sym{**} \(p<0.05\), \sym{***} \(p<0.01\).}\\
\label{tab:beh}
\end{tabular}
}\\
}
\end{landscape}





\newpage
\captionof{table}{Shock effects on hospitalization and out-of-pocket expenditures}
\resizebox{0.9\textwidth}{!}{
{
\def\sym#1{\ifmmode^{#1}\else\(^{#1}\)\fi}
\begin{tabular}{l*{2}{c}}

\toprule
             &\multicolumn{1}{c}{Hospitalization (0-1)}&\multicolumn{1}{c}{Out-of-pocket health care expenditures (0-1)}\\

\hline 
Hunger $\times$ Regime change&     0.045\sym{*}  &    -0.029         \\
            &    (0.014)         &    (0.051)         \\
            \\
Regime change&     0.021\sym{**} &    -0.043         \\
            &    (0.004)         &    (0.058)         \\


Hunger&    0.017         &    -0.026         \\
            &    (0.013)         &    (0.043)         \\
\(N\)         &       38737         &       34466         \\

\hline
Dispossession $\times$ Regime change&       0.033\sym{**} &    -0.018         \\
            &    (0.007)         &    (0.065)         \\
\\
Regime change&      0.021\sym{**} &    -0.044         \\
            &    (0.004)         &    (0.059)         \\
Dispossession&      0.056\sym{*}  &     0.083\sym{**} \\
            &    (0.017)         &    (0.005)         \\

\(N\)        &       39047         &       34746         \\

\hline
Persecution $\times$ Regime change&     0.029\sym{**} &   -0.006         \\
            &    (0.005)         &    (0.077)         \\
\\
Regime change&      0.022\sym{**} &    -0.044         \\
            &    (0.005)         &    (0.057)         \\
Persecution&      0.054\sym{*}  &     0.087         \\
            &    (0.018)         &    (0.014)         \\

\(N\)       &       39079         &       34775         \\

\hline
Period of particular stress $\times$ Regime change&      0.048\sym{*}  &     0.010         \\
            &    (0.012)         &    (0.057)         \\
            \\
Regime change&    0.029\sym{*}  &    -0.018         \\
            &    (0.008)         &    (0.059)         \\
Period of particular stress &      0.030         &     0.071\sym{**} \\
            &    (0.011)         &    (0.004)         \\
\(N\)      &       38451         &       34246 \\


%Age (linear) & YES & YES \\ 
%Gender & YES & YES \\ 
%Childhood SES & YES & YES \\ 
%Childhood health & YES & YES\\ 
%Individual FE & YES & YES\\ 
\bottomrule
\multicolumn{3}{l}{\footnotesize  \textit{Source}: SHARE (waves 4, 5, 6) and SHARELIFE (waves 3, 7), release 7.0.0.}\\
\multicolumn{3}{l}{\footnotesize   \textit{Notes}: Robust standard errors clustered at individual level in parentheses. \sym{*} \(p<0.10\), \sym{**} \(p<0.05\), \sym{***} \(p<0.01\).}\\
\label{tab:hexp}
\end{tabular}
}

\\
}\\




\newpage
\captionof{table}{Shock and regime change effects controlling for age, childhood SES and health by sex}
\resizebox{\textwidth}{!}{
{
\def\sym#1{\ifmmode^{#1}\else\(^{#1}\)\fi}
\begin{tabular}{l*{6}{c}}
\toprule
            &\multicolumn{2}{c}{Short-term health}&\multicolumn{2}{c}{Long-term health}&\multicolumn{2}{c}{Well-being}\\
            &\multicolumn{1}{c}{BMI}&\multicolumn{1}{c}{Obesity (1,0)}&\multicolumn{1}{c}{Chronically ill (1,0)}&\multicolumn{1}{c}{No. of chronic illnesses}&\multicolumn{1}{c}{CASP index}&\multicolumn{1}{c}{Depression scale}\\
\hline 
\\
Hunger $\times$ Regime change $\times$ Female&          1.343\sym{***}&      0.167\sym{***}&     0.097\sym{**} &      0.603\sym{**} &     -5.186\sym{***}&      1.797\sym{**} \\
            &    (0.063)         &    (0.001)         &    (0.015)         &    (0.068)         &    (0.292)         &    (0.335)         \\

\\
Regime change $\times$ Female &     0.855\sym{***}&      0.107\sym{***}&     0.065\sym{**} &      0.282\sym{**} &     -2.725\sym{**} &      0.776\sym{**} \\
            &    (0.030)         &    (0.002)         &    (0.009)         &    (0.033)         &    (0.380)         &    (0.147)         \\


Hunger $\times$ Female&        0.206         &     0.089\sym{**} &      0.113\sym{***}&      0.491\sym{***}&     -3.787\sym{**} &      1.935\sym{***}\\
            &    (0.152)         &    (0.009)         &    (0.001)         &    (0.019)         &    (0.546)         &    (0.159)         \\

\(N\)       &       37783         &       37783         &       38748         &       38660         &       36818         &       37467         \\

\hline
\\
Dispossession $\times$ Regime change $\times$ Female&        1.125\sym{**} &      0.115\sym{***}&     0.076\sym{***}&      0.365\sym{***}&     -2.802\sym{**} &      0.882\sym{**} \\
            &    (0.127)         &    (0.010)         &    (0.008)         &    (0.012)         &    (0.325)         &    (0.114)         \\


\\
Regime change $\times$ Female&       0.831\sym{***}&      0.108\sym{***}&     0.063\sym{**} &      0.274\sym{**} &     -2.745\sym{**} &      0.790\sym{**} \\
            &    (0.027)         &    (0.004)         &    (0.008)         &    (0.035)         &    (0.396)         &    (0.166)         \\


Dispossession $\times$ Female&       -0.745\sym{**} &   -0.002         &     0.060         &      0.255\sym{*}  &     -0.896         &      1.016\sym{**} \\
            &    (0.166)         &    (0.009)         &    (0.021)         &    (0.071)         &    (0.410)         &    (0.112)         \\


\(N\)       &       38081         &       38081         &       39058         &       38969         &       36987         &       37634         \\

\hline
\\
Persecution $\times$ Regime change $\times$ Female&      1.022\sym{***}&      0.108\sym{***}&     0.096\sym{**} &      0.451\sym{***}&     -3.497\sym{***}&      1.108\sym{***}\\
            &    (0.042)         &    (0.001)         &    (0.012)         &    (0.016)         &    (0.183)         &    (0.095)         \\


\\
Regime change $\times$ Female&         0.843\sym{***}&      0.108\sym{***}&     0.062\sym{**} &      0.271\sym{**} &     -2.714\sym{**} &      0.785\sym{**} \\
            &    (0.021)         &    (0.002)         &    (0.007)         &    (0.033)         &    (0.387)         &    (0.162)         \\

Persecution $\times$ Female&      0.096         &     0.043         &      0.113\sym{**} &      0.392\sym{***}&     -2.269\sym{**} &      1.6456\sym{***}\\
            &    (0.052)         &    (0.018)         &    (0.012)         &    (0.022)         &    (0.339)         &    (0.071)         \\


\(N\)       &       38113         &       38113         &       38113         &       36006         &       37010         &       37663         \\

\hline
%\\
%Period of particular happiness $\times$ Regime change $\times$ Female&          1.0174\sym{***}&      0.1242\sym{***}&     0.09738\sym{***}&      0.4030\sym{**} &     -3.6708\sym{**} &      1.1916\sym{**} \\
 %           &    (0.0364)         &    (0.0035)         &    (0.0084)         &    (0.0472)         &    (0.3931)         &    (0.1519)         \\

%\\
%Regime change $\times$ Female&         0.6829\sym{***}&     0.09800\sym{***}&     0.04989\sym{*}  &      0.2145\sym{**} &     -2.6353\sym{**} &      0.7332\sym{*}  \\
 %           &    (0.0244)         &    (0.0070)         &    (0.0126)         &    (0.0353)         &    (0.4423)         &    (0.1861)         \\
%Period of particular happiness $\times$ Female&     -0.7812\sym{***}&    0.008203         &     0.05572\sym{***}&      0.2009\sym{***}&     -1.9750\sym{***}&      1.3454\sym{***}\\
 %           &    (0.0176)         &    (0.0030)         &    (0.0049)         &    (0.0109)         &    (0.1210)         &    (0.0115)         \\
%\(N\)       &       37393         &       37393         &       38315         &       38233         &       36502         &       37129         \\
%\hline
\\
Period of particular stress $\times$ Regime change $\times$ Female&    1.084\sym{***}&      0.136\sym{***}&      0.126\sym{***}&      0.501\sym{***}&     -3.758\sym{**} &      1.357\sym{**} \\
            &    (0.058)         &    (0.003)         &    (0.007)         &    (0.038)         &    (0.471)         &    (0.157)         \\


\\
Regime change $\times$ Female&       0.736\sym{***}&     0.096\sym{***}&     0.057\sym{**} &      0.220\sym{**} &     -2.264\sym{**} &      0.669\sym{**} \\
            &    (0.065)         &    (0.009)         &    (0.008)         &    (0.030)         &    (0.414)         &    (0.153)         \\


Period of particular stress $\times$ Female &       -0.734\sym{***}&     0.012\sym{**} &     0.074\sym{**} &      0.249\sym{***}&     -1.272\sym{***}&      1.340\sym{***}\\
            &    (0.003)         &    (0.003)         &    (0.009)         &    (0.015)         &    (0.093)         &    (0.059)         \\


\(N\)       &       37532         &       37532         &       38461         &       38378         &       36621         &       37258         \\


\hline
Age & YES & YES& YES& YES& YES& YES \\ 
Gender & YES & YES& YES& YES& YES& YES \\ 
Childhood SES & YES & YES& YES& YES& YES& YES  \\ 
Childhood health & YES & YES& YES& YES& YES& YES  \\ 
Individual FE & YES & YES& YES& YES& YES& YES  \\ 
\bottomrule
\multicolumn{7}{l}{\footnotesize  \textit{Source}: SHARE (waves 4, 5, 6) and SHARELIFE (waves 3, 7), release 7.0.0.}\\
\multicolumn{7}{l}{\footnotesize   \textit{Notes}: Robust standard errors clustered at individual level in parentheses. \sym{*} \(p<0.10\), \sym{**} \(p<0.05\), \sym{***} \(p<0.01\).}\\
\label{tab:h_female}
\end{tabular}
}
\\
}\\


\newpage
\captionof{table}{Shock and regime change effects controlling for age, childhood SES and health by area}
\label{area_h}
\resizebox{\textwidth}{!}{
{
\def\sym#1{\ifmmode^{#1}\else\(^{#1}\)\fi}
\begin{tabular}{l*{6}{c}}
\toprule
            &\multicolumn{2}{c}{Short-term health}&\multicolumn{2}{c}{Long-term health}&\multicolumn{2}{c}{Well-being}\\
            &\multicolumn{1}{c}{BMI}&\multicolumn{1}{c}{Obesity (1,0)}&\multicolumn{1}{c}{Chronically ill (1,0)}&\multicolumn{1}{c}{No. of chronic illnesses}&\multicolumn{1}{c}{CASP index}&\multicolumn{1}{c}{Depression scale}\\
\hline 
\\
Hunger $\times$ Regime change $\times$ Rural&       1.730\sym{***}&      0.140\sym{***}&     0.075\sym{***}&      0.452\sym{***}&     -3.680\sym{**} &      0.870\sym{**} \\
            &    (0.145)         &    (0.004)         &    (0.003)         &    (0.022)         &    (0.374)         &    (0.179)         \\


\\
Regime change $\times$ Rural &         1.640\sym{***}&      0.123\sym{***}&     0.057\sym{**} &      0.211\sym{**} &     -2.065\sym{**} &      0.146         \\
            &    (0.080)         &    (0.008)         &    (0.010)         &    (0.032)         &    (0.465)         &    (0.157)         \\



Hunger $\times$ Rural&       0.577         &     0.063\sym{**} &     0.066\sym{**} &      0.319\sym{**} &     -3.330\sym{***}&      1.027\sym{***}\\
            &    (0.201)         &    (0.014)         &    (0.011)         &    (0.042)         &    (0.208)         &    (0.099)         \\


\(N\)       &       37754         &       37754         &       38718         &       38630         &       36794         &       37438         \\

\hline
\\
Dispossession $\times$ Regime change $\times$ Rural&      1.819\sym{***}&      0.123\sym{***}&     0.047\sym{**} &      0.241\sym{**} &     -2.362\sym{**} &      0.2016         \\
            &    (0.127)         &    (0.010)         &    (0.005)         &    (0.030)         &    (0.423)         &    (0.134)         \\



\\
Regime change $\times$ Rural&       1.581\sym{***}&      0.119\sym{***}&     0.056\sym{**} &      0.204\sym{**} &     -1.982\sym{*}  &      0.136         \\
            &    (0.055)         &    (0.007)         &    (0.011)         &    (0.037)         &    (0.495)         &    (0.165)         \\


Dispossession $\times$ Rural&       0.073         &   -0.010         &     0.034         &      0.165         &     -0.165         &      0.426\sym{**} \\
            &    (0.296)         &    (0.006)         &    (0.035)         &    (0.100)         &    (0.545)         &    (0.079)         \\


\(N\)       &       38052         &       38052         &       39029         &       38940         &       36963         &       37606         \\

\hline
\\
Persecution $\times$ Regime change $\times$ Rural&         1.730\sym{***}&      0.134\sym{***}&     0.065\sym{**} &      0.294\sym{***}&     -2.464\sym{**} &      0.280         \\
            &    (0.131)         &    (0.009)         &    (0.009)         &    (0.006)         &    (0.443)         &    (0.224)         \\


\\
Regime change $\times$ Rural&        1.609\sym{***}&      0.119\sym{***}&     0.055\sym{**} &      0.201\sym{**} &     -2.035\sym{*}  &      0.154         \\
            &    (0.060)         &    (0.007)         &    (0.009)         &    (0.031)         &    (0.479)         &    (0.157)         \\


Persecution $\times$ Rural&        0.342\sym{*}  &     0.024\sym{*}  &     0.032\sym{*}  &      0.153\sym{*}  &     -0.265         &      0.488\sym{**} \\
            &    (0.085)         &    (0.008)         &    (0.010)         &    (0.050)         &    (0.224)         &    (0.109)         \\


\(N\)       &       38084         &       38084         &       39060         &       38970         &       36986         &       37634         \\

\hline
%\\
%Period of particular happiness $\times$ Regime change $\times$ Rural&          1.7710\sym{***}&      0.1394\sym{***}&     0.09267\sym{**} &      0.3446\sym{**} &     -3.1607\sym{**} &      0.7101\sym{*}  \\
 %           &    (0.1485)         &    (0.0122)         &    (0.0143)         &    (0.0573)         &    (0.5676)         &    (0.1810)         \\

%\\
%Regime change $\times$ Rural&        1.4237\sym{***}&      0.1072\sym{***}&     0.04944\sym{**} &      0.1726\sym{**} &     -2.1185\sym{*}  &      0.1820         \\
 %           &    (0.0589)         &    (0.0016)         &    (0.0077)         &    (0.0294)         &    (0.5492)         &    (0.1905)         \\


%Period of particular happiness$\times$ Rural &    0.1143         &     0.02493\sym{**} &     0.03845\sym{***}&      0.1054\sym{***}&     -0.9590\sym{**} &      0.6226\sym{***}\\
 %           &    (0.0610)         &    (0.0054)         &    (0.0029)         &    (0.0039)         &    (0.1602)         &    (0.0240)         \\

%\(N\)       &       37367         &       37367         &       38289         &       38207         &       36479         &       37104         \\

%\hline
\\
Period of particular stress $\times$ Regime change $\times$ Rural&     1.734\sym{***}&      0.144\sym{***}&      0.118\sym{***}&      0.443\sym{***}&     -3.050\sym{**} &      0.845\sym{**} \\
            &    (0.107)         &    (0.013)         &    (0.006)         &    (0.030)         &    (0.536)         &    (0.173)         \\


\\
Regime change $\times$ Rural&          1.571\sym{***}&      0.114\sym{***}&     0.058\sym{**} &      0.173\sym{**} &     -1.610\sym{*}  &     0.094         \\
            &    (0.044)         &    (0.003)         &    (0.007)         &    (0.025)         &    (0.543)         &    (0.161)         \\



Period of particular stress $\times$ Rural &       0.163         &     0.027\sym{*}  &     0.055\sym{**} &      0.152\sym{**} &     -0.128         &      0.606\sym{***}\\
            &    (0.065)         &    (0.007)         &    (0.013)         &    (0.018)         &    (0.130)         &    (0.036)         \\

\(N\)       &       37504         &       37504         &       38433         &       38350         &       36598         &       37231         \\


\hline
Age  & YES & YES& YES& YES& YES& YES \\ 
Gender & YES & YES& YES& YES& YES& YES \\ 
Childhood SES & YES & YES& YES& YES& YES& YES  \\ 
Childhood health & YES & YES& YES& YES& YES& YES  \\
Individual FE & YES & YES& YES& YES& YES& YES  \\ 
\bottomrule
\multicolumn{7}{l}{\footnotesize  \textit{Source}: SHARE (waves 4, 5, 6) and SHARELIFE (waves 3, 7), release 7.0.0.}\\
\multicolumn{7}{l}{\footnotesize   \textit{Notes}: Robust standard errors clustered at individual level in parentheses. \sym{*} \(p<0.10\), \sym{**} \(p<0.05\), \sym{***} \(p<0.01\).}\\
\end{tabular}
}
\\
}\\


\newpage
\captionof{table}{Shock and regime change effects controlling for age, childhood SES and health by cohort group}
\label{rev_coh}
\resizebox{\textwidth}{!}{
{
\def\sym#1{\ifmmode^{#1}\else\(^{#1}\)\fi}
\begin{tabular}{l*{6}{c}}
\toprule
            &\multicolumn{2}{c}{Short-term health}&\multicolumn{2}{c}{Long-term health}&\multicolumn{2}{c}{Well-being}\\
            &\multicolumn{1}{c}{BMI}&\multicolumn{1}{c}{Obesity (1,0)}&\multicolumn{1}{c}{Chronically ill (1,0)}&\multicolumn{1}{c}{No. of chronic illnesses}&\multicolumn{1}{c}{CASP index}&\multicolumn{1}{c}{Depression scale}\\
\hline 
\\
Hunger $\times$ Regime change $\times$ Old cohorts&     1.711\sym{***}&      0.133\sym{***}&     0.064\sym{**} &      0.400\sym{***}&     -2.844\sym{***}&      0.514\sym{**} \\
            &    (0.046)         &    (0.013)         &    (0.004)         &    (0.009)         &    (0.260)         &    (0.028)         \\

\\
Regime change $\times$ Old cohorts&     1.575\sym{***}&      0.123\sym{***}&     0.060\sym{**} &      0.245\sym{***} &     -1.946\sym{**} &     0.018         \\
            &    (0.022)         &    (0.009)         &    (0.003)         &    (0.004)         &    (0.396)         &    (0.093)         \\


Hunger $\times$ Old cohorts&      0.821\sym{***}&     0.067\sym{***}&     0.064\sym{**} &      0.298\sym{**} &     -1.553\sym{**} &      0.591\sym{**} \\
            &    (0.047)         &    (0.008)         &    (0.005)         &    (0.039)         &    (0.378)         &    (0.162)         \\

\(N\)       &       37783         &       37783         &       38748         &       38660         &       36818         &       37467         \\

\hline
\\
Dispossession $\times$ Regime change $\times$ Old cohorts&   1.569\sym{***}&     0.105\sym{***}&     0.060\sym{***}&      0.301\sym{***}&     -1.931\sym{**} &      0.066         \\
            &    (0.117)         &    (0.006)         &    (0.008)         &    (0.033)         &    (0.351)         &    (0.084)         \\


\\
Regime change $\times$ Old cohorts&       1.552\sym{***}&     0.122\sym{***}&     0.057\sym{**} &      0.223\sym{**} &     -1.814\sym{**} &     -0.020         \\
            &    (0.028)         &    (0.008)         &    (0.003)         &    (0.005)         &    (0.403)         &    (0.091)         \\


Dispossession $\times$ Old cohorts&   0.250\sym{***}         &    0.016         &     0.065\sym{**}  &      0.231\sym{**}         &     0.331         &      0.167 \\
            &    (0.019)         &    (0.017)         &    (0.011)         &    (0.032)         &    (0.465)         &    (0.172)         \\



\(N\)       &       38081         &       38081         &       39058         &       38969         &       36987         &       37634         \\

\hline
\\
Persecution $\times$ Regime change $\times$ Old cohorts&     1.550\sym{***}&      0.126\sym{***}&     0.078\sym{***} &      0.350\sym{***}&     -1.943\sym{**} &      0.021         \\
            &    (0.083)         &    (0.011)         &    (0.007)         &    (0.011)         &    (0.349)         &    (0.111)         \\


\\
Regime change $\times$ Old cohorts&      1.295\sym{***}&     0.093\sym{***}&     0.048\sym{**} &      0.172\sym{**} &     -1.542\sym{**} &     -0.075         \\
            &    (0.029)         &    (0.009)         &    (0.004)         &    (0.007)         &    (0.374)         &    (0.107)         \\

Persecution $\times$ Old cohorts&      0.138&     -0.011         &     0.056\sym{**}&      0.265\sym{**} &     -0.295 &      0.412\sym{*}\\
            &    (0.138)         &    (0.011)         &    (0.009)         &    (0.031)         &    (0.506)         &    (0.132)         \\


\(N\)       &       38113         &       38113         &       39090         &       39000         &       37010         &       37663         \\


\hline

\\
Period of particular stress $\times$ Regime change $\times$ Old cohorts&      1.828\sym{***}&      0.149\sym{***}&      0.124\sym{***}&      0.456\sym{***}&     -2.910\sym{**} &      0.716\sym{**} \\
            &    (0.108)         &    (0.012)         &    (0.003)         &    (0.011)         &    (0.446)         &    (0.092)         \\


\\
Regime change $\times$ Old cohorts&     1.481\sym{***}&     0.112\sym{***}&     0.082\sym{***} &      0.238\sym{***} &     -1.661\sym{*}  &     0.087         \\
            &    (0.076)         &    (0.010)         &    (0.007)         &    (0.014)         &    (0.391)         &    (0.104)         \\


Period of particular stress $\times$ Old cohorts &    0.150         &    0.023         &     0.074\sym{**} &      0.219\sym{***}&     0.211 &      0.483\sym{**}\\
            &    (0.082)         &    (0.011)         &    (0.014)         &    (0.036)         &    (0.075)         &    (0.088)         \\

\(N\)       &       37532         &       37532         &       38461         &       38378         &       36621         &       37258         \\
\hline\\
Age & YES & YES& YES& YES& YES& YES \\ 
Sex & YES & YES& YES& YES& YES& YES \\ 
Childhood SES & YES & YES& YES& YES& YES& YES  \\ 
Childhood health & YES & YES& YES& YES& YES& YES  \\ 
Individual FE & YES & YES& YES& YES& YES& YES  \\ 
\bottomrule
\multicolumn{6}{l}{\footnotesize  \textit{Source}: SHARE (waves 4, 5, 6) and SHARELIFE (waves 3, 7), release 7.0.0.}\\
\multicolumn{6}{l}{\footnotesize   \textit{Notes}:  Robust standard errors clustered at individual in parentheses. \sym{*} \(p<0.10\), \sym{**} \(p<0.05\), \sym{***} \(p<0.01\).}\\
\end{tabular}
}

\\
}\\

\scriptsize{Old cohorts include individuals born before 1950.}

\end{document}




